\documentclass[
  captions=tableheading,
]{scrartcl}

\usepackage{scrhack}

\usepackage[aux]{rerunfilecheck}

\usepackage{fontspec}

\usepackage[ngerman]{babel}

\usepackage{amsmath}
\usepackage{amssymb}
\usepackage{mathtools}

\usepackage[
  math-style=ISO,
  bold-style=ISO,
  sans-style=italic,
  nabla=upright,
  partial=upright,
]{unicode-math}

\usepackage[
  locale=DE,
  separate-uncertainty=true,
  per-mode=reciprocal,
  output-decimal-marker=.,
]{siunitx}

\usepackage{float}
\floatplacement{figure}{htbp}
\floatplacement{table}{htbp}

\usepackage[
  labelfont=bf,
  font=small,
  width=0.9\textwidth,
]{caption}

\usepackage{graphicx}
\usepackage{booktabs}

\usepackage[
  unicode,
]{hyperref}
\usepackage{bookmark}


\begin{document}

\section{Messwerte und Auswertung}

Die Messdaten befinden sich in Tabelle~\ref{tab:data}.
Die Ergbenisse des Fits an die Funktion
\begin{equation}
  U(t) = a \sin(b t + c) + d
\end{equation}
sind
\begin{align}
  a &= \input{loesung-a.tex} \\
  b &= \input{loesung-b.tex} \\
  c &= \input{loesung-c.tex} \\
  d &= \input{loesung-d.tex} .
\end{align}
Die Messdaten und das Ergebnis des Fits sind in Abbildung~\ref{fig:plot} geplottet.

\begin{table}
  \centering
  \caption{Messdaten.}
  \label{tab:data}
  \input{loesung-table.tex}
\end{table}

\begin{figure}
  \centering
  \includegraphics{loesung-plot.pdf}
  \caption{Messdaten und Fitergebnis.}
  \label{fig:plot}
\end{figure}

\end{document}
