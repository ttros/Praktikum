\section{Durchführung}
\label{sec:Durchführung}

Für den Versuch wird eine Kupfer-Röntgenröhre verwendet, deren Röntgenstrahlung wird an einem
LiF-Kristall gebrochen und mit einem Geiger-Müller-Zählrohr gemessen.
Die Röntgenröhre erzeugt Elektronen mit einer Energie von $\qty{35}{\kilo\electronvolt}$ mit einem
Emmissionsstrom von $\qty{1}{\milli\ampere}$.
Der verwendete LiF-Kristall besitzt eine Gitterkonstante von $d=\qty{201.4}{\pico\metre}$.

\subsection{Vorbereitung}
\label{sec:Vorbereitung}

Zur Vorbereitung auf den Versuch sollen Literaturwerte zu den $K_{\symup{\alpha}}$ und $K_{\symup{\beta}}$
Linien von Kupfer recherchiert werden. Zusätlich wurde aus diesen Werten mithilfe von \autoref{eq:Energie}
und \autoref{eq:Bragg} der zugehörige Braggwinkel berechnet.
\begin{align*}
    K_{\symup{\alpha}} &= \qty{8}{\kilo\electronvolt} \\
    \theta_{\symup{\alpha}} &= \qty{22.66}{\degree}\\
    K_{\symup{\beta}} &= \qty{8.95}{\kilo\electronvolt}\\
    \theta_{\symup{\beta}} &= \qty{20.14}{\degree}
\end{align*}

Für die Auswertung der Absorptionsmessung werden Literaturwerte für die $K$-Kanten verschiedener Elemente benötigt.
Auch hier werden direkt mit \autoref{eq:Energie} und \autoref{eq:Bragg} der Braggwinkel sowie mit \autoref{eq:sigma_K} die
Abschirmkonstante berechnet, alle Werte finden sich in \autoref{tab:theoriewerte}.

\begin{table} [H]
    \centering
    \caption{Literaturwerte für die Absorptionsenergie $E_{\symup{K}}$, sowie errechnete Werte für
            $\theta_{\symup{K}}$ und $\sigma_{\symup{K}}$. Dabei ist $Z$ die Ordnungszahl des jeweiligen Elements \cite{Ekabs}}
    \label{tab:theoriewerte}
    \begin{tabular}{l c S S S}
        \toprule
         & Z & {$E_{\symup{K}}^{\symup{Lit}} \mathbin{/} \unit{\kilo\electronvolt}$} & 
         {$\Theta_{\symup{K}}^{\symup{Lit}} \mathbin{/} \unit{\degree}$} & {$\sigma_{\symup{K}}$} \\
        \midrule
        Zink         & 30 &  9.65 & 18.60 & 3.56 \\
        Gallium      & 31 & 10.37 & 17.29 & 3.61 \\
        Germanium    & 32 & 11.10 & 16.20 & 3.68 \\
        Brom         & 35 & 13.47 & 13.23 & 3.85 \\
        Rubidium     & 37 & 15.20 & 11.70 & 3.94 \\
        Strontium    & 38 & 16.10 & 11.04 & 4.00 \\
        Zirconium    & 40 & 17.99 &  9.86 & 4.10 \\
    \bottomrule
    \end{tabular}
\end{table}

\subsection{Messung}

Ziel der ersten Messung ist es, die Braggbedingung zu prüfen. Dafür wird der Li-F Kristall auf einen festen Winkel
von $\theta=\qty{14}{\degree}$ eingestellt. Mithilfe des Geiger-Müller-Zählrohrs wird nun die Intensität der Röntgenstrahlung
in einem Bereich zwischen $\qty{26}{\degree}$ und $\qty{30}{\degree}$ gemessen, dabei beträgt der Winkelzuwachs
$\symup{\Delta\alpha}=\qty{0.1}{\degree}$ mit einer Integrationszeit von $\Delta t=\qty{5}{\second}$.

Als nächstes soll das Emmisionsspektrum der Cu-Röntgenröhre untersucht werden. Dafür wird der Winkel des LiF-Kristall an den des
Geiger-Müller-Zählrohrs gekoppelt, damit die Braggbedingung zu jedem Zeitpunkt erfüllt ist.
Es wird in einem Bereich zwischen $\qty{4}{\degree}$ und $\qty{26}{\degree}$ gemessen, diesmal mit
$\symup{\Delta\alpha}=\qty{0.2}{\degree}$ und wie vorher $\Delta t=\qty{5}{\second}$.

Direkt dannach soll der Bereich um die $K_{\symup{\alpha}}$ und $K_{\symup{\beta}}$ Kanten genauer untersucht werden.
Dafür wird in einem möglichst kleinem Winkelbereich diese Stelle erneut vermessen, um für eine gute Auflösung
zu sorgen wird $\symup{\Delta\alpha}=\qty{0.1}{\degree}$ gewählt, $\Delta t=\qty{5}{\second}$ bleibt gleich.

Die weiteren Messungen dienen zur Untersuchung des Absorptionsspektrums und werden mit fünf verschiedenen
Elementen durchgeführt.
Vor jeder Messung wird eine dünne Schicht des Absorbers direkt vor dem Geiger-Müller-Zählrohr befestigt.
Anschließend wird in einem geeigneten Messbereich, der um den in \autoref{tab:theoriewerte} berechneten Braggwinkel liegt,
eine Messung mit einem Winkelzuwachs von $\symup{\Delta\alpha}=\qty{0.1}{\degree}$ gestartet.
Die Integrationszeit wird hier auf $\Delta t=\qty{20}{\second}$ eingestellt.

Alle Messwerte werden digital erfasst und zur Auswertung in \textit{.txt} Dateien abgespeichert.