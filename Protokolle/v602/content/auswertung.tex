\section{Auswertung}
\label{sec:Auswertung}

\subsection{Überprüfung der Bragg-Bedingung}
Zur Überprüfung der Bragg-Bedingung werden die Messwerte graphisch in \autoref{fig:Bragg} dargestellt und es wird das Maximum bei
$\theta_{\symup{exp}}=\qty{13.85}{\degree}$ abgelesen. Da der theoretische Bragg-Winkel bei $\theta_{\symup{theorie}} = \qty{14}{\degree}$
liegt, beläuft sich die relative Abweichung hier auf rund $\symup{\Delta}\theta = \qty{1,1}{\percent}$.

\begin{figure}[H]
  \centering
  \includegraphics[height=8cm]{build/Bragg.pdf}
  \caption{Überprüfung der Bragg-Bedingung liefert experimentellen Bragg-Winkel von $\theta_{\symup{exp}}=\qty{13.85}{\degree}.$}
  \label{fig:Bragg}
\end{figure}

\subsection{Emissionsspektrum einer Cu-Röhre}
Die Daten zu der breiten Messreihe des Emissionsspektrum werden in \autoref{fig:Emissionsspektrum} visualisiert. Der Bremsberg, sowie die 
$K_{\alpha}$- und $K_{\beta}$-Linien sind deutlich zu erkennen und lassen sich zu
\begin{align*}
    K_{\alpha} &= \qty{22,4}{\degree} \\
    K_{\beta} &= \qty{20,2}{\degree}
\end{align*}
bestimmen. Der Grenzwinkel $\theta_{\symup{Grenz}}$ ist nicht besonders deutlich zu sehen, kann aber auf etwa
\begin{equation*}
  \theta_{\symup{Grenz}} = \qty{5,2}{\degree}
\end{equation*}
geschätzt werden. Daraus folgt mit \eqref{eq:Energie} und \eqref{eq:Bragg} eine minimale Wellenlänge von 
\begin{equation*}
  \lambda_{\symup{min}} = \qty{36.51}{\pico\metre}
\end{equation*}
und eine maximale Energie von
\begin{equation*}
  E_{\symup{max}} = \qty{33.96}{\kilo\electronvolt}.
\end{equation*}


\begin{figure}[H]
  \centering
  \includegraphics[height=8cm]{build/Emissionsspektrum.pdf}
  \caption{Emissionsspektrum der Cu-Röhre mit Markierungen des Bremsbergs, sowie der $K_{\alpha}$- und $K_{\beta}$-Linien.}
  \label{fig:Emissionsspektrum}
\end{figure}

In \autoref{fig:Detailspektrum} sind die Daten der Aufzeichung des Detailspektrums dargestellt. Damit das Auflösungsvermögen
der Apperatur bestimmt werden kann, werden die Halbwertsbreiten der $K_{\alpha}$- und $K_{\beta}$-Linien in den Plot eingezeichnet 
und die Schnittpunkte mit der Messwertkurve bestimmt. So ergeben sich für jeden Peak zwei Winkel, denen über \eqref{eq:Energie} und \eqref{eq:Bragg}
Energien zugeordnet werden können. Die Energiedifferenz dieser Energien $\symup{\Delta}E$ wird verwendet, um über
\begin{equation*}
  A = \frac{E}{\symup{\Delta}E}
\end{equation*}
das Auflösungsvermögen zu berechnen. 

\begin{figure}[H]
  \centering
  \includegraphics[height=8cm]{build/Detailspektrum.pdf}
  \caption{Detailspektrum der Cu-Röhre mit eingezeichneten Halbwertsbreiten der $K_{\alpha}$- und $K_{\beta}$-Linien.}
  \label{fig:Detailspektrum}
\end{figure}

\begin{table}
  \centering
  \caption{Darstellung des Auflösungsvermögens.}
  \label{tab:Auflösungsvermögen}
  \begin{tabular}{c S[table-format=1.2] S[table-format=1.2] S[table-format=2.1]}
    \toprule
    {K-Linie} & {$E \mathbin{/} \unit{\kilo\electronvolt}$} & {$\symup{\Delta}E \mathbin{/} \unit{\kilo\electronvolt}$} &%
    {$A$}\\
    \midrule
    $\alpha$  & 8,04 & 0,17 & 52,3 \\
    $\beta$   & 8,87 & 0,21 & 42,4 \\
    \bottomrule
  \end{tabular}
\end{table}

In einem nächsten Schritt werden mit \eqref{eq:Sigma_Kupfer}, \eqref{eq:Sigma_Kupfer2}, und \eqref{eq:Sigma_Kupfer3}, sowie den
Energien der K-Linien aus \autoref{tab:Auflösungsvermögen} die Abschirmkonstanten berechnet. 

\begin{table}
  \centering
  \caption{Darstellung der Abschirmkonstanten. Die Berechnung erfolgt sowohl für die Energien aus \autoref{tab:Auflösungsvermögen}, %
  als auch für Theoriewerte.}
  \label{tab:Abschirmkonstanten}
  \begin{tabular}{c S[table-format=2.2] S[table-format=2.2] S[table-format=2.2]}
    \toprule
    {$\sigma_{\symup{n}}$} & {experimentell} & {theoretisch} & {relative Abweichung}\\
    \midrule
    $\sigma_1$  & {}    & 3.29  & {}        \\
    $\sigma_2$  & 12.30 & 11.95 & 2,91\,\%  \\
    $\sigma_3$  & 20.16 & 23.99 & 15,94\,\% \\
    \bottomrule
  \end{tabular}
\end{table}

\subsection{Absorptionsspektrum verschiedener Elemente}
\subsubsection{Bestimmung des Bragg-Winkels, Absorptionsenergie, und Abschirmkonstanten.}
Die Daten aus den Messungen der Absorptionsspektren der verschiedenen Absorber werden graphisch dargestellt und es wird die
Absorptionsenergie bestimmt. Dies geschieht indem zunächst die Messwerte vor dem Wendepunkt und nach dem Mittelpunkt einzeln gemittelt
werden und dann wiederrum der Mittelwert dieser beiden Mittelwerte als horizontale Linie in den Plot eingezeichnet wird. Die Schnittstelle
der Messwertkurve mit dieser Linie ist dann der Braggwinkel $\theta_{\symup{K}}$ zur Absorptionsenergie der K-Kante.
Aus der Absorptionsenergie lässt sich mit \eqref{eq:sigma_K} außerdem die Abschirmkonstante berechnen. Dies wird für fünf verschiedene
Absorber durchgeführt, die Plots für Brom, Gallium, Strontium, und Zirconium sind im Anhang \ref{sec:Anhang} zu finden.

\begin{figure}[H]
  \centering
  \includegraphics[height=8cm]{build/Zn.pdf}
  \caption{Bestimmung des Bragg-Winkels des Zink-Absorbers.}
  \label{fig:Zn}
\end{figure}

In den nachfolgenden Tabellen \ref{tab:theta_K}, \ref{tab:E_K}, und \ref{tab:sigma_K} sind die berechneten Werte, sowie die relativen 
Abweichungen zu den Theoriewerten aufgeführt.

\begin{table}[H]
  \centering
  \caption{Bragg-Winkel der verschiedenen Absorber.}
  \label{tab:theta_K}
  \begin{tabular}{c S S S}
    \toprule
    {Element} & {$\theta_{\symup{K}}^{\symup{lit}} \mathbin{/} \unit{\degree}$} & %
    {$\theta_{\symup{K}}^{\symup{exp}}  \mathbin{/} \unit{\degree}$} & %
    {$\symup{\Delta}\theta_{\symup{K}} \mathbin{/} \unit{\percent}$}\\
    \midrule
    Zn & 18.60 & 18.60 & 0.00 \\
    Br & 13.23 & 13.18 & 0.38 \\
    Ga & 17.29 & 17.32 & 0.17 \\
    Sr & 11.04 & 11.02 & 0.18 \\
    Zr &  9.86 &  9.90 & 0.41 \\
    \bottomrule
  \end{tabular}
\end{table}

\begin{table}[H]
  \centering
  \caption{Absorptionsenergie der verschiedenen Absorber.}
  \label{tab:E_K}
  \begin{tabular}{c S S S}
    \toprule
    {Element} & {$E_{\symup{K}}^{\symup{lit}} \mathbin{/} \unit{\kilo\electronvolt}$} & %
    {$E_{\symup{K}}^{\symup{exp}}  \mathbin{/} \unit{\kilo\electronvolt}$} & %
    {$\symup{\Delta}E_{\symup{K}} \mathbin{/} \unit{\percent}$}\\
    \midrule
    Zn &  9.65 &  9.65 & 0.01 \\
    Br & 13.47 & 13.50 & 0.22 \\
    Ga & 10.37 & 10.34 & 0.30 \\
    Sr & 16.10 & 16.10 & 0.02 \\
    Zr & 17.99 & 17.90 & 0.48 \\
    \bottomrule
  \end{tabular}
\end{table}

\begin{table}[H]
  \centering
  \caption{Abschirmkonstanten der verschiedenen Absorber.}
  \label{tab:sigma_K}
  \begin{tabular}{c S S S}
    \toprule
    {Element} & {$\sigma_{\symup{K}}^{\symup{lit}}$} & %
    {$\sigma_{\symup{K}}^{\symup{exp}}$} & %
    {$\symup{\Delta}\sigma_{\symup{K}} \mathbin{/} \unit{\percent}$}\\
    \midrule
    Zn & 3.56 & 3.57 & 0.15 \\
    Br & 3.85 & 3.81 & 0.97 \\
    Ga & 3.61 & 3.65 & 1.15 \\
    Sr & 4.00 & 3.99 & 0.09 \\
    Zr & 4.10 & 4.19 & 2.21 \\
    \bottomrule
  \end{tabular}
\end{table}

\subsubsection{Bestimmung der Rydbergkonstanten}
Aus dem Moseleyschen Gesetz \eqref{eq:moseley} !!!!! folgt ein linearer Zusammenhang zwischen $\sqrt{E_{\symup{K}}}$ und $Z$. Der 
Proportionalitätsfaktor ist die Rydbergenergie $R_\infty$. Die Daten werden in dem Diagramm \autoref{fig:moseley} dargestellt und es wird
mit der python-Erweiterung \textit{scipy}\,\cite{scipy} eine lineare Ausgleichsrechnung durchgeführt.

\begin{figure}[H]
  \centering
  \includegraphics[height=8cm]{build/Rydberg.pdf}
  \caption{Plot der Absorptionsenergien und zugehörigen Ordnungszahlen.}
  \label{fig:moseley}
\end{figure}

Die lineare Ausgleichsrechnung liefert die Gerade vom Typ $y = mx +b$ mit den Parametern
\begin{align*}
  m &= \qty{3,57+-0.02}{\sqrt{\unit{\electronvolt}}} \\
  b &= \qty{-8,93+-0.54}{\sqrt{\unit{\electronvolt}}}.
\end{align*}
Für die Rydbergenergie ergibt sich demnach 
\begin{align*}
  R_{\infty} = \qty{12.76+-0.11}{\electronvolt}
\end{align*}
was einer relativen Abweichung zum Theoriewert von rund $\symup{\Delta}R_{\infty} = \qty{6.2}{\percent}$ entspricht.