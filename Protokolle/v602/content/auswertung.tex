\section{Auswertung}
\label{sec:Auswertung}

\subsection{Überprüfung der Bragg-Bedingung}
Zur Überprüfung der Bragg-Bedingung werden die Messwerte graphisch in \autoref{fig:Bragg} dargestellt und es wird das Maximum bei
$\theta_{\symup{exp}}=\qty{13.85}{\degree}$ abgelesen. Da der theoretische Bragg-Winkel bei $\theta_{\symup{theorie}} = \qty{14}{\degree}$
liegt, beläuft sich die relative Abweichung hier auf rund $\symup{\Delta}\theta = \qty{1,1}{\percent}$.

\begin{figure}[H]
  \centering
  \includegraphics[height=8cm]{build/Bragg.pdf}
  \caption{Überprüfung der Bragg-Bedingung liefert experimentellen Bragg-Winkel von $\theta_{\symup{exp}}=\qty{13.85}{\degree}.$}
  \label{fig:Bragg}
\end{figure}

\subsection{Emissionsspektrum einer Cu-Röhre}
Die Daten zu der breiten Messreihe des Emissionsspektrum werden in \autoref{fig:Emissionsspektrum} visualisiert. Der Bremsberg, sowie die 
$K_{\alpha}$- und $K_{\beta}$-Linien sind deutlich zu erkennen und lassen sich zu
\begin{align*}
    K_{\alpha} &= \qty{22,4}{\degree} \\
    K_{\beta} &= \qty{20,2}{\degree}
\end{align*}
bestimmen. Der Grenzwinkel $\theta_{\symup{Grenz}}$ ist nicht besonders deutlich zu sehen, kann aber auf etwa
\begin{equation*}
  \theta_{\symup{Grenz}} = \qty{5,2}{\degree}
\end{equation*}
geschätzt werden. Daraus folgt mit \eqref{eq:Energie} und \eqref{eq:Bragg} eine minimale Wellenlänge von 
\begin{equation*}
  \lambda_{\symup{min}} = \qty{36.51}{\pico\metre}
\end{equation*}
und eine maximale Energie von
\begin{equation*}
  E_{\symup{max}} = \qty{33.96}{\kilo\electronvolt}.
\end{equation*}


\begin{figure}[H]
  \centering
  \includegraphics[height=8cm]{build/Emissionsspektrum.pdf}
  \caption{Emissionsspektrum der Cu-Röhre mit Markierungen des Bremsbergs, sowie der $K_{\alpha}$- und $K_{\beta}$-Linien.}
  \label{fig:Emissionsspektrum}
\end{figure}

In \autoref{fig:Detailspektrum} sind die Daten der Aufzeichung des Detailspektrums dargestellt. Damit das Auflösungsvermögen
der Apperatur bestimmt werden kann, werden die Halbwertsbreiten der $K_{\alpha}$- und $K_{\beta}$-Linien in den Plot eingezeichnet 
und die Schnittpunkte mit der Messwertkurve bestimmt. So ergeben sich für jeden Peak zwei Winkel, denen über \eqref{eq:Energie} und \eqref{eq:Bragg}
Energien zugeordnet werden können. Die Energiedifferenz dieser Energien $\symup{\Delta}E$ wird verwendet, um über
\begin{equation*}
  A = \frac{E}{\symup{\Delta}E}
\end{equation*}
das Auflösungsvermögen zu berechnen. 

\begin{figure}[H]
  \centering
  \includegraphics[height=8cm]{build/Detailspektrum.pdf}
  \caption{Detailspektrum der Cu-Röhre mit eingezeichneten Halbwertsbreiten der $K_{\alpha}$- und $K_{\beta}$-Linien.}
  \label{fig:Detailspektrum}
\end{figure}

\begin{table}
  \centering
  \caption{Darstellung des Auflösungsvermögens.}
  \label{tab:Auflösungsvermögen}
  \begin{tabular}{c S[table-format=1.2] S[table-format=1.2] S[table-format=2.1]}
    \toprule
    {K-Linie} & {$E \mathbin{/} \unit{\kilo\electronvolt}$} & {$\symup{\Delta}E \mathbin{/} \unit{\kilo\electronvolt}$} &%
    {$A$}\\
    \midrule
    $\alpha$  & 8,04 & 0,17 & 52,3 \\
    $\beta$   & 8,87 & 0,21 & 42,4 \\
    \bottomrule
  \end{tabular}
\end{table}

In einem nächsten Schritt werden mit \eqref{eq:Sigma_Kupfer}, \eqref{eq:Sigma_Kupfer2}, und \eqref{eq:Sigma_Kupfer3}, sowie den
Energien der K-Linien aus \autoref{tab:Auflösungsvermögen} die Abschirmkonstanten berechnet. 

\begin{table}
  \centering
  \caption{Darstellung der Abschirmkonstanten. Die Berechnung erfolgt sowohl für die Energien aus \autoref{tab:Auflösungsvermögen}, %
  als auch für Theoriewerte.}
  \label{tab:Abschirmkonstanten}
  \begin{tabular}{c S[table-format=2.2] S[table-format=2.2] S[table-format=2.2]}
    \toprule
    {$\sigma_{\symup{n}}$} & {experimentell} & {theoretisch} & {relative Abweichung}\\
    \midrule
    $\sigma_1$  & {}    & 3.29  & {}        \\
    $\sigma_2$  & 12.30 & 11.95 & 2,91\,\%  \\
    $\sigma_3$  & 20.16 & 23.99 & 15,94\,\% \\
    \bottomrule
  \end{tabular}
\end{table}

\subsection{Absorptionsspektrum verschiedener Elemente}