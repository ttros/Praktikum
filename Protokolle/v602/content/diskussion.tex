\section{Diskussion}
\label{sec:Diskussion}
Der experimentell bestimmte Braggwinkel stimmt gut mit der theoretisch erwarteten Braggbedingung überein.
Der relativ kleine Fehler von $\symup{\Delta}\theta = \qty{1,1}{\percent}$ kann auf eine nicht
perfekte Einstellung der Messapperatur zurückgeführt werden, sollte aber keinen großen Einfluss auf die
weiteren Messergebnisse haben.

Der qualitative Verlauf des Emmisionsspektrums der Cu-Röhre weist den erwarteten Bremsberg
sowie die $K_{\alpha}$- und $K_{\beta}$-Linien auf. Ihre absolouten Werte stimmen gut
mit den Theoriewerten überein, die maximale Energie $E_{\symup{max}} = \qty{33.96}{\kilo\electronvolt}$
liegt leicht unter dem theoretischen Maximum von $\qty{35}{\kilo\electronvolt}$, das die Röhre
erzeugen kann.

Die Abweichungen der $K$-Linien betragen $\symup{\Delta}K_{\alpha}=\qty{0,5}{\percent}$- und
$\symup{\Delta}K_{\beta}=\qty{1,3}{\percent}$ und fallen somit sehr gering aus.
Aus den Energien der K-Linien wurden experimentelle Werte für die Abschirmkonstanten
$\sigma_2$ und $\sigma_3$ bestimmt. Vor allem $\sigma_3$ weißt mit $\qty{15,94}{\percent}$ eine
relativ große Abweichung zur Theorie auf, diese könnte an Ableseungenauigkeiten der Peaks liegen,
da diese relativ breit sind.

Im letzten Versuchsteil wurden die Absorptionsenergien und Abschirmkonstanten von fünf verschiedenen Elementen
berechnet. Nahezu sämtlich Werte besitzen hier sehr geringe Abweichungen von der Theorie, die deutlich kleiner als
$\qty{1}{\percent}$ sind. Die größte Abweichung tritt bei der Abschirmkonstante von Zr auf.
Sie liegt bei legidglich $\qty{2,21}{\percent}$ und ist somit ebenfalls akzeptabel.

Die experimentell berechnete Rydbergenergie weicht um  $\qty{5,74}{\percent}$ von der Theorie ab,
der etwas größere Fehler liegt hier wahrscheinlich daran, dass die für die Regression verwendeten Werte
der Abschirmkonstanten bereits fehlerbehaftet sind.

Zusammenfassend konnten mit diesem Versuch viele Theoriewerte präzise bestätigt werden, die digitale Erfassung
der Daten sowie eine gute Kalibrierung der Messapperatur haben hier positiven Einfluss gehabt.

