\section{Diskussion}
\label{sec:Diskussion}
Mithilfe der ersten Veruschsreihe konnte für die Schallgeschwindigkeit ein plausibler
Wert bestimmt werden, der nah am theoretischen Wert liegt.

Die Bestimmung der Dicke der Kopplungsschicht über die lineare Regression weist einen großen Fehler auf,
weshalb die alternative Berechnung durchgeführt wurde.
Der neue Wert hat anschließend die Ultraschallmessungen gut korrigiert, weshalb die relativen
Fehler der Länge nur bei wenigen Prozent liegen.

Auffällig ist jedoch, dass der Fehler für die Vermessung der Bohrung mit der Nummer 9 von oben mit rund $\qty{48,05}{\percent}$, gerade 
im Vergleich zu den anderen Fehlern, sehr groß ist. Eine mögliche Erklärung hierfür ist das Ablesen einer falschen Reflexion.

Während der \textit{B-Scan} einen guten optischen Eindruck über die Position der Bohrungen ermöglicht,
sind die gewonnen Werte ungenauer. Aufgrund des Rauschen im Bild sind größere relative Fehler entstanden.
Außerdem konnte eine Fehlstelle nicht korrekt identifiziert werden, deshalb liegen nur zehn Messdaten
trotz elf Bohrungen vor.

Ähnliche Schwierigkeiten spielen beim Scan des Brustmodells eine Rolle. Der qualitative Nachweis der Tumore
war mit dem Scanbild möglich, für genauere Aussagen über deren Größe und Art ist allerdings auch hier das
Bild zu verrauscht. Eine bessere Auflösung wäre hier eventuell, wie in \ref{sec:Untersuchung des Auflösungsvermögen}
beobachtet, durch Erhöhung der Frequenz erreichbar.