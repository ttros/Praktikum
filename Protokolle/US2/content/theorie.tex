\section{Zielsetzung}
Ziel des Versuches ist es, zwei unterschiedliche Scanmethoden der Ultraschalltechnik anzuwenden
und damit die Fehlstellen in einem Acrylblock zu bestimmen.
Anschließend soll ein Brustmodell mit derselben Technik auf mögliche Tumore untersucht werden.

\section{Theorie}
\label{sec:Theorie}

\subsection{Ultraschall}
\label{sec:Ultraschall}

Bei Schall handelt es sich allgemein um eine longitudinale Welle, welche sich als Druckschwankung
in einem Medium fortbewegt.
Das menschliche Gehör kann diese in einem Frequenzbereich von ca. $16\,\unit{\hertz}$ bis $20\,\unit{\kilo\hertz}$
wahrnehmen. Schallwellen mit höheren Frequenzen zwischen $20\,\unit{\kilo\hertz}$ und $1\,\unit{\giga\hertz}$ werden als
Ultraschall bezeichnet. \\
Die Ultraschallwellen breiten sich mit einer vom Medium abhängigen Schallgeschwindigkeit $c$ aus.
Weiterhin ist die akustische Impedanz (auch Schallkennwiderstand) $Z=\rho \cdot c$ von Interesse, sie hängt
von der Schallgeschwindigkeit sowie von der Dichte des Mediums ab. \\
Während der Ausbreitung verlieren die Schallwellen aufgrund von Absorption an Energie, ihre Intensität nimmt ab.
Es besteht ein exponentieller Zusammenhang zwischen der Intensität und der zurückgelegten Strecke:
\begin{equation}
    \label{eq:Dämpfung}
    I(x)=I_{0}\symup{e}^{-\alpha x}
\end{equation}
Aufgrund des hohen Absorptionskoeffizienten von Luft wird zwischen Schallgeber und Material ein Kontaktmittel verwendet.

Zur Erzeugung von Ultraschall wird unter anderem der sogenannte reziproke piezo-elektrische Effekt verwendet.
Piezo-elektrische Kristalle schwingen in elektrischen Wechselfeldern und strahlen dabei Ultraschallwellen ab.
Derselbe Effekt funktioniert umgekehrt, somit kann ein solcher Kristall auch als Empfänger verwendet werden,
der abhängig von der Schallintensität ein elektrisches Signal emittiert.

\subsection{Messverfahren}
\label{sec:Messverfahren}
Um Informationen über das zu untersuchende Material mithilfe von Ultraschall zu gewinnen, werden unter anderem
Laufzeitmessungen durchgeführt. Dabei gibt es zwei wesentliche Verfahren.

Beim \textit{Durchschallungs-Verfahren} wird mit einem Sender ein kurzzeitiger Schallimpuls in die Probe gegeben.
Am anderen Ende der Probe wird das Signal dann wieder empfangen, aus der Laufzeit lässt sich die Länge berechnen:
\begin{equation}
    \label{eq:Durchschall}
    s=c\cdot t
\end{equation}

Die andere Möglichkeit nennt sich \textit{Impuls-Echo-Verfahren}. Hier wird der Sender gleichzeitig als Empfänger verwendet.
Der gesendete Impuls wird am Ende der Probe reflektiert und somit wieder empfangen.
Für die Längenberechnung zählt hier deshalb nur die halbe Zeit:
\begin{equation}
    \label{eq:Impuls-Echo}
    s=\frac{1}{2} c\cdot t
\end{equation}

Die gesammelten Daten lassen sich auf zwei verschiedene Arten darstellen,
den sogenannten \textit{A-Scan} und \textit{B-Scan}.
\begin{itemize}
    \item Beim \textit{A-Scan} handelt es sich um ein eindimensionales Verfahren, es werden die Amplituden
    des Echos in Abhängigkeit der Laufzeit oder der Laufstrecke dargestellt.
    \item Beim \textit{B-Scan} werden die Messdaten zweidimensional dargestellt. Die Echoamplituden werden
    in Form von Helligkeitsstufen dargestellt, somit kann ein zweidimensionales Schnittbild erzeugt werden.
\end{itemize}