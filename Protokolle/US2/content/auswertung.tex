\section{Auswertung}
\label{sec:Auswertung}

% Abmessungen Acrylblock
\subsection{Abmessungen des Acrylblocks}
\label{sec:Abmessungen}
Die Höhe des Acrylblocks wird zu
\begin{equation*}
    h = \qty{80,5}{\milli\metre}
\end{equation*}
bestimmt. Aus der Höhe und den Abständen der Bohrungen zu der oberen und der unteren Kante wird außerdem der Durchmesser berechnet.
\begin{table}[H]
    \centering
    \caption{Abmessungen des Acrylblocks.}
    \label{tab:Schieblehre}
    \begin{tabular}{S[table-format=2.0] S[table-format=2.1] S[table-format=2.1] S[table-format=2.1]}
        \toprule
         {Lochnummer} & {$s_{\symup{oben}}\,/\,\unit{\milli\metre}$} & {$s_{\symup{unten}}\,/\,\unit{\milli\metre}$} %
         & {$d\,/\,\unit{\milli\metre}$}\\
        \midrule
         1	& 13.1	& 61.3 &  6.1 \\
         2	& 21.7	& 53.9 &  4.9 \\
         3	& 30.2	& 46.3 &  4.0 \\
         4	& 38.8	& 39.0 &  2.7 \\
         5	& 46.6	& 31.0 &  2.9 \\
         6	& 54.8	& 23.0 &  2.7 \\
         7	& 63.6	& 15.0 &  1.9 \\
         8	& 70.5	&  7.0 &  3.0 \\
         9	& 10.3	& 55.4 & 14.8 \\
         10	& 59.6	& 19.5 &  1.4 \\
         11	& 61.3	& 17.9 &  1.3 \\
        \bottomrule 
    \end{tabular}
  \end{table}

% c_Acryl
\subsection{Bestimmung der Schallgeschwindigkeit in Acryl}
Aus der Laufzeit der Ultraschallwellen und der aus \ref{sec:Abmessungen} bekannten Positionen der Bohrungen wird ein Plot erstellt. Mithilfe einer 
linearen Ausgleichsrechnung mit der \textit{python}-Erweiterung \textit{scipy} \cite{scipy} wird eine Regressiongerade vom Typ $y=mx+b$ mit den
Parametern
\begin{align*}
    m &= \qty{360.3(3.0)e-6}{\second\per\metre} \\
    b &= \qty{0.7(3.0)e-6}{\second}
\end{align*}
bestimmt. Aus \eqref{eq:Impuls-Echo} wird die Schallgeschwindigkeit in Acryl zu
\begin{equation*}
    c_{\symup{exp}} = \frac{1}{m} = \qty{2775+-23}{\metre\per\second}
\end{equation*}
berechnet. Der Parameter $b$ beschreibt die Laufzeit des Schallimpules in der Kopplungsschicht. Mit der Schallgeschwindigkeit in der Kopplungsschicht
von $c_{\symup{Wasser}} = \qty{1483}{\metre\per\second}$ \cite{czichos} kann die Dicke der Kopplungsschicht $\tilde{d_{\symup{K}}}$ bestimmt werden.
\begin{equation*}
    \tilde{d_{\symup{K}}} = c_{\symup{Wasser}} \cdot b = \qty{1+-4}{\milli\metre}
\end{equation*}

\begin{table}[H]
    \centering
    \caption{Daten $c$-Bestimmung mit Impuls-Echo-Verfahren.}
    \label{tab:c-bestimmung}
    \begin{tabular}{S[table-format=1.0] S[table-format=2.1] S[table-format=2.1]}
        \toprule
         {Lochnummer} & {$s_{\symup{oben}}\,/\,\unit{\milli\metre}$} & {$\symup{\Delta}t\,/\,\unit{\micro\second}$} \\
        \midrule
         1	& 13,1	& 10,7 \\
         2	& 21,7	& 16,9 \\
         3	& 30,2	& 23,1 \\
         4	& 38,8	& 29,5 \\
         5	& 46,6	& 35,3 \\
         6	& 54,8	& 41,0 \\
         7	& 63,6	& 46,8 \\ 
        \bottomrule 
    \end{tabular}
  \end{table}

\begin{figure}[H]
    \centering
    \includegraphics{c-bestimmung.pdf}
    \caption{Graphische Darstellung der Messwertpaare aus \autoref{tab:c-bestimmung} mit Ausgleichsgerade.}
    \label{fig:c-bestimmung}
  \end{figure}

% Dicke Kopplungsschicht
\subsection{Berechnung der Dicke der Kopplungsschicht}
Die aus der linearen Regression bestimmte Dicke der Kopplungsschicht weist eine große Messunsicherheit auf, weshalb hier mit einer anderen
Berechnungsmethode versucht wird, diese Unsicherheit zu verringern.

Dafür wird der aus \ref{sec:Abmessungen} bekannte Abstand der Bohrungen zur Oberkante und ein Literaturwert für die Schallgeschwindigkeit
in Acryl von $c_{\symup{Acryl}} = \qty{2730}{\metre\per\second}$ \cite{c_Acryl} verwendet, um so die theoretische Laufzeit des Schallimpules 
zu berechnen. Aus der Differenz der tatsächlichen und der theoretischen Laufzeit aus \autoref{tab:c-bestimmung} kann so wieder mit der 
Schallgeschwindigkeit in der Kopplungsschicht die Dicke dieser zu
\begin{equation*}
    d{\symup{K}} = \qty{0.68+-0.22}{\milli\metre}
\end{equation*}
bestimmt werden.

Im Folgenden wird aufrund der geringeren Messunsicherheit dieser Wert verwendet.

% Vermessung mit A-Scan
\subsection{Vermessung des Acrylzylinder mit einem A-Scan}
Die Abstände der Bohrungen zu der oberen und unteren Seite des Acrylblocks müssen um die Dicke der Anpassungsschicht korrigiert werden. 
Zur Veranschaulichung wird außerdem die relative Abweichung der korrigierten Messwerte zu den Werten aus \ref{sec:Abmessungen} angegeben.

  \begin{table}[H]
    \centering
    \caption{Unkorrigierte Daten der Vermessung des Acrylblocks mit einem A-Scan.}
    \label{tab:a-scan}
    \begin{tabular}{S[table-format=2.0] S[table-format=2.1] S[table-format=2.1] S[table-format=2.1]}
        \toprule
         {Lochnummer} & {$\tilde{s_{\symup{oben}}}\,/\,\unit{\milli\metre}$} & {$\tilde{s_{\symup{unten}}}\,/\,\unit{\milli\metre}$} \\
        \midrule
         1	& 14.7 & 62.5 \\
         2	& 23.1 & 55.1 \\
         3	& 31.6 & 47.6 \\
         4	& 40.0 & 40.2 \\
         5	& 48.1 & 32.2 \\
         6	& 56.0 & 24.1 \\
         7	& 63.9 & 16.3 \\
         8	& 72.3 &  8.4 \\
         9	& 16.6 & 56.7 \\
         10	& 60.7 & 20.7 \\
         11	& 62.4 & 19.2 \\
        \bottomrule 
    \end{tabular}
  \end{table}

  \begin{table}[H]
    \centering
    \caption{Korrigierte Daten der Vermessung des Acrylblocks mit einem A-Scan.}
    \label{tab:a-scan}
    \begin{tabular}{S[table-format=2.0] S[table-format=2.1] S[table-format=2.1] S[table-format=2.1]}
        \toprule
         {Lochnummer} & {$\tilde{s_{\symup{oben}}}\,/\,\unit{\milli\metre}$} & {$\tilde{s_{\symup{unten}}}\,/\,\unit{\milli\metre}$} \\
        \midrule
         1	& 14.7 & 62.5 \\
         2	& 23.1 & 55.1 \\
         3	& 31.6 & 47.6 \\
         4	& 40.0 & 40.2 \\
         5	& 48.1 & 32.2 \\
         6	& 56.0 & 24.1 \\
         7	& 63.9 & 16.3 \\
         8	& 72.3 &  8.4 \\
         9	& 16.6 & 56.7 \\
         10	& 60.7 & 20.7 \\
         11	& 62.4 & 19.2 \\
        \bottomrule 
    \end{tabular}
  \end{table}
% Vermessung mit B-Scan
\subsection{Vermessung des Acrylzylinder mit einem B-Scan}