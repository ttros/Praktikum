\section{Diskussion}
\label{sec:Diskussion}

Es fällt auf, dass die gemessene Grenzspannung der roten Spektrallinie größer ist, als jene der gelben und grünen.
Dies kann physikalisch nicht möglich sein, somit ist dies auf einen Fehler der Messapparatur zurückzuführen.

Im Folgenden wurde dieses Resultat dann bei der Bestimmung von dem Verhältnis $\frac{h}{e_0}$ ausgelassen,
um ein möglichst aussagekräftiges Ergebnis zu erhalten.
Dennoch ist bei den Grenzspannungen kein guter linearer Zusammenhang gegeben, wie sich am Fit in
\autoref{fig:planck} leicht erkennen lässt.
Abschließend soll das Endergebniss noch mit dem theoretischen Wert verglichen werden, hier lauten die genauen Werte
\begin{align*}
    (\frac{h}{e_0})_{\symup{theo}} &= \qty{4.135668e-15}{\volt\second}\\
    (\frac{h}{e_0})_{\symup{exp}} &= \qty{3.0(1.1)e-15}{\volt\second}
\end{align*}
Selbst unter Berücksichtigung der großen Unsicherheit des experimentellen Ergebnisses sind die Werte nicht vereinbar.
Bei der optischen Anordnung der Messapparatur scheint es zu viele Fehlerquellen gegeben zu haben, sodass trotz
des Weglassen der Messergergebnisse für die rote Spekatrallinie zu große Unsicherheiten aufgetreten sind.

Unabhängig davon bestätigt vor allem \autoref{fig:gelb_lang} qualitativ viele Eigenschaften des Photoeffekts,
die hier trotz der Unsicherheiten beobachtet werden konnten.