\section{Auswertung}
\label{sec:Auswertung}

\subsection{Bestimmung der Grenzspannung}

\begin{figure} [H]
  \centering
  \includegraphics{build/plot_rot.pdf}
  \caption{Plot rot.}
  \label{fig:plot_rot}
\end{figure}

\begin{figure} [H]
  \centering
  \includegraphics{build/plot_gelb.pdf}
  \caption{Plot gelb.}
  \label{fig:plot_gelb}
\end{figure}

\begin{figure} [H]
  \centering
  \includegraphics{build/plot_gruen.pdf}
  \caption{Plot gruen.}
  \label{fig:plot_gruen}
\end{figure}

\begin{figure} [H]
  \centering
  \includegraphics{build/plot_violett_1.pdf}
  \caption{Plot violett 1.}
  \label{fig:plot_violett_1}
\end{figure}

\begin{figure} [H]
  \centering
  \includegraphics{build/plot_violett_2.pdf}
  \caption{Plot violett 2.}
  \label{fig:plot_violett_2}
\end{figure}

\subsection{Planksches Wirkungsquantum}

\begin{figure} [H]
  \centering
  \includegraphics{build/plot_planck.pdf}
  \caption{Planck.}
  \label{fig:planck}
\end{figure}

\subsection{Vollständiger Spannungsverlauf}

\begin{figure} [H]
  \centering
  \includegraphics{build/plot_gelb_lang.pdf}
  \caption{Gelb lang.}
  \label{fig:gelb_lang}
\end{figure}