\section{Auswertung}
\label{sec:Auswertung}

\subsection{Bestimmung der Grenzspannung}

Die Messergergebnisse für die rote Spektrallinie finden sich in \autoref{tab:rot}.
Die Daten werden ausgewertet, indem die Bremsspannung $U_{\symup{B}}$ geben die Wurzel des Photostroms $\sqrt{I}$
geplottet wird.
Mithilfe der Python Erweiterung \textit{scipy}\,\cite{scipy} wird eine lineare Ausgleichsrechnung durchgeführt,
die Geradengleichung ist dabei vom Typ
\begin{equation}
  \label{eq:regression}
  \sqrt{I}=m \cdot U_{\symup{B}} + b.
\end{equation}
Der Plot der Messwerte sowie der Regressionsgerade ist in \autoref{fig:plot_rot} zu sehen.

\begin{table}[H]
  \centering
  \caption{Messwerte für die rote Spektrallinie.}
  \label{tab:rot}
  \begin{tabular}{S[table-format=3.1] S[table-format=1.2]}
      \toprule
      $U_{\symup{B}} / \unit{\volt}$ & $I / \unit{\nano\ampere}$ \\
      \midrule
      -0.4 & 	0.7   \\
      -0.3 &	0.56 \\
      -0.2 &	0.41 \\
      -0.1 &	0.29 \\
       0.0 &	0.20 \\
       0.1 &	0.12 \\
       0.2 &	0.06 \\
       0.3 &	0.04 \\
       0.4 &	0.02 \\
       0.5 &	0.01 \\
       0.7 &	0    \\
      \bottomrule
  \end{tabular}
\end{table}

\begin{figure} [H]
  \centering
  \includegraphics{build/plot_rot.pdf}
  \caption{Plot der Bremsspannung $U_{\symup{B}}$ gegen die Wurzel des Photostroms $I$ für die rote Spektrallinie.}
  \label{fig:plot_rot}
\end{figure}

Die gesuchte Grenzspannung $U_{\symup{G}}$ ist dabei die Nullstelle der Regressionsgerade und lässt sich durch
Umstellen von \autoref{eq:regression} bestimmen als
\begin{equation*}
  U_{\symup{G}}=\frac{-b}{m}.
\end{equation*}
Diese Auswertungsschrite werden für die restlichen Spektrallinien analog durchgeführt, die zugehörigen
Wertetabellen und Plots finden sich im Anhang.
Die Ergebnisse für sämtliche Spektrallinien sind in \autoref{tab:U_G} zusammengefasst.

\begin{table}[H]
  \centering
  \caption{Parameter der linearen Regression sowie berechneter Wert der Grenzspannung für alle Spektrallinien..}
  \label{tab:U_G}
  \begin{tabular}{S S S S}
      \toprule
      {$\lambda/\unit{\nano\meter}$} & {$m/(\frac{\unit{\nano\ampere}}{\unit{\volt}})$} & {$b/\unit{\nano\ampere}$} & {$U_{\symup{G}}/\unit{\volt}$} \\
      \midrule
      646 & -2.49$\pm$0.14 & 1.49$\pm$0.14 & 0.598 \\
      578 & -9.9$\pm$0.5   & 4.9$\pm$0.5   & 0.491 \\
      546 & -5.8$\pm$0.5   & 3.3$\pm$0.5   & 0.570 \\
      435 & -4.68$\pm$0.10 & 5.33$\pm$0.10 & 0.749 \\
      405 & -2.70$\pm$0.06 & 3.51$\pm$0.06 & 1.300 \\
      \bottomrule
  \end{tabular}
\end{table}

\subsection{Planksches Wirkungsquantum}

Nun werden die jeweiligen Wellenlängen der Spektrallinien in Frequenzen umgerechnet.
Die Frequenzen werden dann gegen ihre zugehörigen Grenzspannungen geplottet, welche in \autoref{tab:U_G} berechnet wurden.
Erneut wird eine lineare Regression ermittelt, zusammen mit den Daten ist sie in \autoref{fig:planck} zu sehen.

\begin{figure} [H]
  \centering
  \includegraphics{build/plot_planck.pdf}
  \caption{Plot der unterschiedlichen Lichtfrequenzen gegen die zugehörigen Bremsspannungen inklusive linearer Regression.}
  \label{fig:planck}
\end{figure}

Die Geradengleichung der Regression entspricht dem theoretischen Zusammenhang in \autoref{eq:Energie Photoelektronen}.
Stellt man die Gleichung nach $U_{\symup{B}}$ um, so erhält man
\begin{equation*}
  U_{\symup{B}} = \frac{h}{e_0}f - \frac{A_{\symup{k}}}{e_0}.
\end{equation*}
Das Verhältnis $\frac{h}{e_0}$ sowie die Austrittsarbeit $A_{\symup{k}}$ ergeben sich somit direkt aus den
Parametern der linearen Regression, abschließend erhält man
\begin{gather*}
  \frac{h}{e_0}=\qty{3.0(1.1)e-15}{\volt\second} \\
  A_{\symup{k}}=\qty{1.117401729598114+-0.000000000000001}{\electronvolt}
\end{gather*}

\subsection{Kurvenverlauf des Photostroms}

Zuletzt wird der vollständige Verlauf des Photostroms der gelben Spekatrallinie betrachtet,
die dafür gemessenen Daten sind in \autoref{tab:gelb_lang} festgehalten.
Die Daten werden in \autoref{fig:gelb_lang} geplottet, der Verlauf wird anschließend qualitativ analysiert.

\begin{table}[H]
  \centering
  \caption{Messwerte für die gelbe Spektralinie, hier über einen größeren Spannungsbereich.}
  \label{tab:gelb_lang}
  \begin{tabular}{S[table-format=3.1] S[table-format=2.1]}
      \toprule
      $U_{\symup{B}} / \unit{\volt}$ & $I / \unit{\nano\ampere}$ \\
      \midrule
      -19.0 &	101 \\
      -18.0 &	99 \\
      -17.0 &	98 \\
      -16.0 &	96 \\
      -15.0 &	96 \\
      -14.0 &	96 \\
      -13.0 &	95 \\
      -12.0 &	93 \\
      -11.0 &	91 \\
      -10.0 &	88 \\
      -9.0 &	86 \\
      -8.0 &	83 \\
      -7.0 &	80 \\
      -6.0 &	76 \\
      -5.0 &	70 \\
      -4.0 &	64 \\
      -3.0 &	55 \\
      -2.0 &	42 \\
      -1.0 &	28 \\
       0.0 &	4.4\\
       0.1 &	2.4\\
       0.2 &	0.8\\
       0.3 &	0.2\\
       0.4 &	0.0\\
      \bottomrule
  \end{tabular}
\end{table}

\begin{figure} [H]
  \centering
  \includegraphics{build/plot_gelb_lang.pdf}
  \caption{Photostrom in Abhängigkeit der Spannung für die gelbe Spekatrallinie, mit einem
          größeren Messbereich für die Spannung.}
  \label{fig:gelb_lang}
\end{figure}

Für hohe Beschleunigungsspannungen erreicht die Kurve einen Sättigungswert. Hierbei handelt es sich aber nicht um einen
Widerspruch mit dem ohmschen Gesetz, da aufgrund der endlichen Lichtintensität nur eine begrenzte Menge an Elektronen
gelöst werden können, somit ist auch der Photostrom auf ein Maximum beschränkt.
Da niemals alle Elektronen die Elektrode erreichen, nährt sich die Spannnung nur asymptotisch diesem Maximum an.

Nährt man sich der Grenzspannung, so fällt der Strom kontinuierlich auf Null ab, ein sprunghafter Abfall ist
nicht zu beobachten.
Dies liegt an der Energieverteilung der Elektronen im Leiter, die für eine nicht zu vernachlässigende Streuung
der Energie der Elektronen nach dem Lösen aus der Photokathode verantworlich ist.

Zusätzlich zum Photostrom kann außerdem ein diesem entgegengesetzter Strom auftreten. Dies liegt daran,
dass die Photokathode aus einem Material besteht, dass bereits bei Zimmertemperatur verdampft.
Somit können sich losgelöste Elektronen an der Kathode anlagern und anschließend zur Photokathode zurück beschleunigt werden.
Da dieser Effekt im Vergleich zum Photostrom nur eine geringe Menge an Elektronen freisetzt, ist
ein Sättigungswert schnell erreicht.
Das frühe Auftreten dieses negativen Stroms im Experiement zeigt, dass die Photokathode eine recht kleine
Austrittsarbeit aufweist.