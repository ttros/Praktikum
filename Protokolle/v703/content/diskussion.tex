\section{Diskussion}
\label{sec:Diskussion}
Die in diesem Verusuch ermittelte Charakteristik (\autoref{fig:charakteristik}) weist bei einer Plateaubreite
von $L=\qty{290}{\volt}$ eine Steigung von $m_{\text{Plateau}} = \qty{0.78+-0.21}{\percent\per100\volt}$
auf. Ein ideales GMZ hat in diesem Bereich keine Steigung, dennoch kann aufgrund des geringen Wertes von einem ausreichend
genauen GMZ gesprochen werden.

Diese Aussage kann allerdings nur getroffen werden, wenn die in \autoref{fig:charakteristik_roh} zu sehenden, systematisch 
abweichenden Messwertgruppen herausgerechnet werden. Die Abweichung dieser Messwerte lässt sich vermutlich auf einen
technischen Defekt des Verstärkers zurückführen.

Die Totzeit des GMZ wird über zwei verschiedene Arten bestimmt und beträgt
\begin{align*}
    T_{\text{Oszilloskop}} &= \qty{100}{\micro\second} \\
    T_{\text{2-Quellen}} &= \qty{130+-50}{\micro\second}.
\end{align*}
Die beiden Werte lassen sich mit dem großen Fehlerbereich von $T_{\text{2-Quellen}}$ miteinander vereinen und es lässt sich eine Aussage 
über die Größenordnung der Totzeit treffen. Dennoch muss erwähnt werden, dass der mithilfe des Oszilloskops bestimmte Wert
eine nicht weiter betrachtete Ableseungenauigkeit aufweist.

In dem letzten Versuchsteil wird die freigesetzte Ladung des GMZ betrachtet, wobei sich ein annähernd linearer Zusammenhang
zwischen der freigesetzten Ladung $Q$ und dem Strom $I$ ergibt. Dabei fällt auf, dass einzelne Werte stark abweichen, was auch
hier möglicher Weise auf einen defekten Verstärker zurückzuführen ist.