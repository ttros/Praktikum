\section{Diskussion}
\label{sec:Diskussion}
Bei der Bestimmung des Kontaktpotentials ist zu beachten, dass größere Ablesefehler bei der Bestimmung der Steigungsdreicke nicht
auszuschließen sind und daher der Wert des Kontaktpotentials als unsicher zu bezeichnen ist. Auch sind Ungenauigkeiten des analogen 
x-y-Schreibers nicht beachtet worden.

Die Bestimmung der Energieverteilung der Elektronen bei $T=\qty{145}{\celsius}$ weist wie in \autoref{fig:Int Energie 145 Grad} zu 
sehen ist leider nicht die gewünschte Form auf. Bei diesen Temperaturen kommt es vermehrt zu elastischen Stößen zwischen Elektronen
und Hg-Atomen, was bedingt, dass deutlich weniger Elektronen an der Auffangelektrode detektiert werden können, da diese starke
Richtungsänderungen erfahren.

Die experimentell bestimmten Energiedifferenzen der Energieniveaus des Hg-Atoms von
\begin{align*}
    \bar{E_1}&=\qty{5.12+-0.22}{\electronvolt} \\
    \bar{E_2}&=\qty{5.1+-0.4}{\electronvolt}
\end{align*}
lassen sich mit ihren Messunsicherheiten mit dem Literaturwert von
\begin{equation*}
    E_{\symup{lit}} = \qty{4,9}{\electronvolt}
\end{equation*}
in Einklang bringen. Die nicht zu vernachlässigenden Messunsicherheiten lassen sich auch hier mit Ablesefehlern erklären.

Die in \ref{sec:Franck-Hertz-Kurve} berechneten Wellenlängen für die Übergänge der Grundzustände von 
\begin{align*}
    \lambda_1&=\qty{242+-10}{\nano\meter} \\
    \lambda_2&=\qty{241+-18}{\nano\meter}
\end{align*}
sind mit dem Literaturwert von $\lambda_{\symup{lit}} = \qty{253}{\nano\meter}$ knapp vereinbar und liegen wie erwartet im Bereich
der Röntgenstrahlung.