\section{Auswertung}
\label{sec:Auswertung}

\subsection{Diffferentielle Energieverteilung der Elektronen}

Zu Beginn wird die Messung bei $T=20\,\unit{\celsius}$ ausgewertet.
Die mithilfe des analogen x-y-Schreibers erzeugte Grafik ist in \autoref{fig:Int Energie 20 Grad} dargestellt.
Sie stellt die integrale Energieverteilung der Elektronen dar.

\begin{figure}[H]
  \centering
  \includegraphics[height=8cm]{content/pics/originaldaten/1.pdf}
  \caption{Aufgezeichnete Integrale Energieverteilung der Elektronen $T=20\,\unit{\celsius}$.}
  \label{fig:Int Energie 20 Grad}
\end{figure}

Um nun die differentielle Energieverteilung der Elektronen zu erhalten, werden an mehreren Werten
für $U_A$ kleine Steigungsdreiecke verwendet.
Deren Länge wird zu $\symup{\Delta}U_{\symup{A}}=0.5\,\unit{\volt}$ gewählt. Dann berechnet sich die lokale
Änderung des Auffängerstroms über die Gleichung
\begin{equation*}
\symup{\Delta}I_{\symup{A}}(U_{\symup{A}})=I_{\symup{A}}(U_{\symup{A}})-I_{\symup{A}}(U_{\symup{A}}+\symup{\Delta}U_{\symup{A}}).
\end{equation*}
Die so ermittelten Wertepaare sind in \autoref{tab:Diff Energie 20 Grad} festgehalten.

\begin{table}[H]
  \centering
  \caption{Abgelesene Wertepaare für $U_{\symup{A}}$ und $\symup{\Delta}I_{\symup{A}}$ aus zehn Steigungsdreiecken in \autoref{fig:Int Energie 20 Grad}}
  \label{tab:Diff Energie 20 Grad}
  \begin{tabular}{S[table-format=1.1] S[table-format=1.1] S[table-format=2.0]}
      \toprule
       {$U_{\symup{A}}\,/\,\unit{\volt}$} & {$\symup{\Delta}U_{\symup{A}}\,/\,\unit{\volt}$} & {$\symup{\Delta}I_{\symup{A}}(U_{\symup{A}})\,/\,\unit{\ampere}$} \\
      \midrule
      0.0 & 0.5 &	3 \\
      0.5 & 0.5 &	2 \\
      1.0 & 0.5 &	2 \\
      1.5 & 0.5 &	1 \\
      2.0 & 0.5 &	2 \\
      2.5 & 0.5 &	2 \\
      3.0 & 0.5 &	2 \\
      3.5 & 0.5 &	3 \\
      4.0 & 0.5 &	4 \\
      4.5 & 0.5 &	6 \\
      5.0 & 0.5 &	7 \\
      5.5 & 0.5 &	9 \\
      6.0 & 0.5 &	12 \\
      6.5 & 0.5 &	18 \\
      7.0 & 0.5 &	23 \\
      7.5 & 0.5 &	14 \\
      \bottomrule 
  \end{tabular}
\end{table}

In \autoref{fig:Diff Energie 20Grad} sind die Werte geplottet, zusammen beschreiben sie den Verlauf der gesuchten differentiellen
Energieverteilung.
Aufgrund der geringen Temperatur finden hier so gut wie keine Wechselwirkungen von Elektronen und Hg-Atomen statt.
Die freie Weglänge der Elektronen im Quecksilberdampf beträgt bei dieser Temperatur nach \eqref{eq:freie weglänge}
$\bar{w}(T) = 0,810\,\unit{\centi\meter}$. Sie ist damit beinahe so lang wie der Abstand zwischen Kathode und
Beschleunigungselektrode, somit finden nur sehr wenige Stöße statt.

Aus dem Kurvenverlauf lässt sich schließen, dass die Änderung des Auffängerstroms zunächst mit steigendem $U_A$ zunimmt.
Bei $U_{\symup{A}}=7\,\unit{\volt}$ nimmt sie ihr Maximum an, dannach fällt die Änderung wieder schnell ab.
Dies bedeutet, dass die meisten Elektronen eine Energie im Bereich um $U_{\symup{A}}=7\,\unit{\electronvolt}$ aufweisen.
Eine Erhöhung der Bremsspannung hat in diesem Bereich also einen besonders großen Einfluss auf den Auffängerstrom.

Das Kontaktpotential $K$ berechnet sich somit aus der Differenz der tatsächlichen Beschleunigungsspannung und der des Peaks:
\begin{equation*}
  K=U_{\symup{B}}-U_{\symup{Peak}}=(15-7)\,\unit{\volt}=8\,\unit{\volt}
\end{equation*}

\begin{figure}[H]
    \centering
    \includegraphics[height=8cm]{build/Differentielle_Energie_20Grad.pdf}
    \caption{Differentielle Energieverteilng der Elektronen bei $T=20\,\unit{\celsius}$.}
    \label{fig:Diff Energie 20Grad}
\end{figure}

Die Auswertung der Messreihe für $T=145\,\unit{\celsius}$ erfolgt analog zur niedrigen Temperatur.

\begin{figure}[H]
  \centering
  \includegraphics[height=8cm]{content/pics/originaldaten/2.pdf}
  \caption{Aufgezeichnete Integrale Energieverteilung der Elektronen $T=145\,\unit{\celsius}$.}
  \label{fig:Int Energie 145 Grad}
\end{figure}

\begin{table}[H]
  \centering
  \caption{Abgelesene Wertepaare für $U_A$ und $\symup{\Delta}I_A$ aus 4 Steigungsdreiecken in \autoref{fig:Int Energie 145 Grad}}
  \label{tab:Diff Energie 145 Grad}
  \begin{tabular}{S[table-format=1.1] S[table-format=1.1] S[table-format=2.0]}
      \toprule
       {$U_{\symup{A}}\,/\,\unit{\volt}$} & {$\symup{\Delta}U_{\symup{A}}\,/\,\unit{\volt}$} & {$\symup{\Delta}I_{\symup{A}}(U_{\symup{A}})\,/\,\unit{\ampere}$} \\
      \midrule
      0.0 & 0.5 &	15 \\
      0.5 & 0.5 &	15 \\
      1.0 & 0.5 &	16 \\
      1.5 & 0.5 &	17 \\
      2.0 & 0.5 &	14 \\
      \bottomrule 
  \end{tabular}
\end{table}


\begin{figure}[H]
    \centering
    \includegraphics[height=8cm]{build/Differentielle_Energie_145Grad.pdf}
    \caption{Differentielle Energieverteilng der Elektronen bei $T=145\,\unit{\celsius}$.}
    \label{fig:Diff Energie 145Grad}
\end{figure}

Hier fällt der Auffängerstrom schnell auf Null ab, die Steigung ist dabei nahezu konstant.
Aufgrund der deutlich höheren Temperatur können viele elastische Stöße
zwischen den Elektronen und Hg-Atomen stattfinden. Dabei werden die Elektronen gestreut und besitzen nur noch
eine geringe kinetische Energie in z-Richtung.
Dies lässt sich durch die geringe freie Weglänge bestätigen, die legiglich $\bar{w}(T) = 7,303\,\unit{\micro\meter}$ beträgt.

\subsection{Franck-Hertz-Kurve}

Zur Bestimmung der Anregungsenergie der Hg-Atome werden die zwei aufgezeichneten Kurven getrennt ausgewertet.
Entscheidend sind die relativen Abstände der Peaks zueinander, deren Mittelwert entspricht direkt der
gesuchten Anregungsenergie. Die erste Kurve ist in \autoref{fig:FHZ 166Grad} dargestellt.

\begin{figure}[H]
  \centering
  \includegraphics[height=8cm]{content/data/FH_166.pdf}
  \caption{Aufgezeichnete Franck-Hertz Kurve bei $T=166\,\unit{\celsius}$.}
  \label{fig:FHZ 166Grad}
\end{figure}

Die jeweiligen Abstände der Peaks sind beretis auf der x-Achse vermerkt, somit muss nur noch deren
Mittelwert bestimmt werden. Mithilfe der Python Erweiterung \textit{uncertainties}~\cite{uncertainties} 
ergibt sich als Ergebnis der ersten Versuchsreihe:
\begin{gather*}
    \bar{E_1}=\qty{5.12+-0.22}{\electronvolt}.
\end{gather*}
Mithilfe von \eqref{eq:Einstein ist smart} wird berechnet sich die Wellenlänge der beim Übergang in den
Grundzustand emittierten Strahlung zu:
\begin{gather*}
  \lambda_1=\qty{242+-10}{\nano\meter}.
\end{gather*}

Für die zweite Messreihe wird identisch vorgegangen, die zugehörige Frank-Hertz Kurve ist in \autoref{fig:FHZ 195Grad} zu sehen.

\begin{figure}[H]
  \centering
  \includegraphics[height=8cm]{content/data/FH_195.pdf}
  \caption{Aufgezeichnete Franck-Hertz Kurve bei $T=195\,\unit{\celsius}$.}
  \label{fig:FHZ 195Grad}
\end{figure}

Es ergibt sich hier abschließend:
\begin{gather*}
  \bar{E_2}=\qty{5.1+-0.4}{\electronvolt}
\end{gather*}
sowie eine Wellenlänge von
\begin{gather*}
  \lambda_2=\qty{241+-18}{\nano\meter}.
\end{gather*}

Die elastischen Stöße der Elektronen haben keinen Einfluss auf das Ergebnis, da sie lediglich aufgrund des
Energienverlustes die Höhe der Franck-Hertz Kurve beeinflussen. Die relativen Abstände der Peaks bleiben 
davon unverändert.