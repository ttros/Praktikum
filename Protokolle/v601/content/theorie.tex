\section{Zielsetzung}
Ziel des Franck-Hertz-Versuchs ist es, die Quantennatur der Elektronenhülle von Atomen nachzuweisen. Dafür wird gezeigt, dass 
Quecksilberatome in diskreten Energiespektren Energie aufnehmen, indem Stöße zwischen den Quecksilberatomen und Elektronen 
herbeigeführt werden.

\section{Theorie}
\label{sec:Theorie}
\subsection{Anregung eines Quecksilberatoms}
\label{sec:Anregung}
Damit die Anregungsenergie der Hg-Atome bestimmt werden kann, werden Elektronen mit einer bekannten Energie in Quecksilberdampf 
geschossen. Dabei Stoßen die Elektronen mit den Hg-Atomen zusammen und es kommt sowohl zu unelastischen, als auch zu elastischen 
Stößen.
Die elastischen Stöße sorgen vor allem dafür, dass sich die Geschwindigkeiten und Richtung der Elektronen ändern. Bei den unelastischen
Stößen kommt es hingegen zu einer Erhöhung der Energieniveaus der Hg-Atome.

Durch Detektion der Energie der Elektronen vor und nach den Stößen lässt sich aus der Energiedifferenz die auf das Hg-Atom übertragene
Energie berechnen:
\begin{equation}
    E_{\symup{a}} = E_1-E_0 = \frac{m_{\symup{e}}v_{\symup{vor}}}{2}-\frac{m_{\symup{e}}v_{\symup{nach}}}{2}
    \label{eq:Anregungsenergie}
\end{equation}

Die von dem Hg-Atom geht nach einer Relaxationszeit von $t\approx\qty{10e-8}{\second}$ in seinen Grundzustand über,
wobei ein Lichtquant mit der Energie
\begin{equation*}
    E_{\symup{a}} = hf
\end{equation*}
emittiert wird.

\subsection{Idealisierte Franck-Hertz-Kurve}
\label{sec:Frank-Hertz-Kurve}