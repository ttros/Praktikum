\section{Auswertung}
\label{sec:Auswertung}

Im folgenden wird der Brechungsindex $n$ auf zwei Arten aus den experiementellen Daten bestimmt,
die einzelnen Methoden werden getrennt betrachtet.
Vor Versuchsbeginn wird eine Dunkelmessung durchgeführt, um den durch das Umgebungslicht ausgelösten Photostrom
zu bestimmen, dieser ergab sich zu
\begin{equation*}
    I_{\text{D}}=\qty{7}{\nano\ampere}.
\end{equation*}
Außerdem bestimmt sich der Nullstrom, den der unreflektierte Laser in der Photozelle auslöst, zu
\begin{align*}
    I_{\text{0, s}}&=\qty{540}{\micro\ampere} \\
    I_{\text{0, p}}&=\qty{520}{\micro\ampere}.
\end{align*}

\subsection{Bestimmung der Brechungsindizes $n_{\text{s}}$ und $n_{\text{p}}$ über die Fresnelschen Formeln}
\label{sec:fresnel}

Die gemessenen Photoströme sind in \autoref{tab:daten} festgehalten.
Von allen Strömen wird der aus der Dunkelmessung bestimmte Strom abgezogen.
Um die Brechungsindizes aus den Daten berechnen zu können, werden die Fresnelschen Formeln
\eqref{eq:Fresnel s} und \eqref{eq:Fresnel p} nach $n_{\text{s}}$ bzw. $n_{\text{p}}$ umgestellt:

\begin{align}
    n_\text{p}(\alpha, E) &= \left(\frac{E + 1}{E - 1}\right)^2 \frac{1}{2\cos^2(\alpha)} + \sqrt{\frac{1}{4\cos^2(\alpha)}\left(\frac{E + 1}{E - 1}\right)^4 - \left(\frac{E + 1}{E - 1}\right)^2\tan^2(\alpha)} \label{eq:n_s} \\
    n_\text{s}(\alpha, E) &= \sqrt{\frac{1 + E^2 + 2E\cos(2\alpha)}{1 - 2E + E^2}} \label{eq:n_p}
\end{align}

Dabei beschreibt ist $E$ das Verhältnis aus reflektierter und einfallender Amplitude. Da der Photostrom
proportional zur Intensität und diese proportional zum Quadrat der Amplitude ist, lässt sich diese Größe auch schreiben als
\begin{equation*}
    E = \frac{\symbf{E}_{\text{ref}}}{\symbf{E}_{\text{ein}}} = \sqrt{\frac{I(\alpha)}{I_0}}.
\end{equation*}
Dabei ist $I_0$ der bereits bekannte der Nullstrom.
Die Brechungsindizes ergeben sich somit durch Eingesetzen der Messwerte in \eqref{eq:n_s} und \eqref{eq:n_p}.
Hierbei fällt auf, dass durch systematische Fehler teilweise zu kleine Werte für $n_{\text{s}}$ und zu große Werte für $n_{\text{p}}$
resultieren.
Um deren Einfluss auf das gemittelte Endergebnis zu minimieren, werden alle $n_{\text{s}}<3$ und $n_{\text{p}}>4.4$
vernachlässigt.
Anschließend wird der Mttelwert aller Brechungsindizes gebildet, dessen Fehler mithilfe der Python-Erweiterung
\textit{uncertainties} \cite{uncertainties} bestimmt wird. Die Ergebnisse lauten somit
\begin{align*}
    \bar{n}_{\text{s}}&=\qty{3.42+-0.29}{} \\
    \bar{n}_{\text{p}}&=\qty{3.4+-0.4}{}.
\end{align*}


\begin{table} [H]
    \centering
    \caption{Messdaten für die Photoströme bei s- und p-polarisiertem Licht, abhängig vom Einfallswinkel $\alpha$.}
    \label{tab:daten}
    \begin{tabular}{c | c c}
      \toprule
      {$\alpha\,/\,\unit{\degree}$} & {$I_{\text{s}}\,/\,\unit{\micro\ampere}$} & {$I_{\text{p}}\,/\,\unit{\micro\ampere}$}\\
      \midrule
      5  & 46  & 42 \\
      7  & 40  & 39 \\
      9  & 41  & 40 \\
      11 & 42  & 38 \\
      13 & 40  & 36 \\
      15 & 43  & 38 \\
      17 & 46  & 41 \\
      19 & 46  & 39 \\
      21 & 47  & 40 \\
      23 & 49  & 40 \\
      25 & 49  & 40 \\
      27 & 48  & 36 \\
      29 & 51  & 38 \\
      31 & 56  & 39 \\
      33 & 56  & 38 \\
      35 & 57  & 36 \\
      37 & 63  & 38 \\
      39 & 66  & 36 \\
      41 & 62  & 32 \\
      43 & 68  & 33 \\
      45 & 68  & 32 \\
      47 & 76  & 32 \\
      49 & 79  & 30 \\
      51 & 80  & 26 \\
      53 & 76  & 23 \\
      55 & 94  & 24 \\
      57 & 100 & 22 \\
      59 & 100 & 20 \\
      61 & 100 & 17 \\
      63 & 107 & 14 \\
      65 & 113 & 11 \\
      67 & 113 & 9.0\\
      69 & 120 & 6.2\\
      71 & 120 & 3.6\\
      73 & 130 & 2.2\\
      75 & 133 & 1.6\\
      77 & 143 & 3.8\\
      79 & 147 & 6.0\\
      81 & 147 & 14 \\
      83 & 153 & 26 \\
      85 & 153 & 46 \\
      87 & 160 & 88 \\
      \bottomrule
    \end{tabular}
\end{table}

\subsection{Bestimmung der Brechungsindizes $n_{\text{s}}$ und $n_{\text{p}}$ über den Brewsterwinkel}

Der Brewsterwinkel stellt in der Theorie ein vollständiges Verschwinden der Reflektion dar, sodass für den Photostrom
gilt $I{\text{p}}=0$.
Da zu jederzeit ein gewisser Untergrundstrom vorhanden ist, fällt der gemessene Photostrom des p-polarisierten Lichts nie
vollständig auf Null ab. Deshalb wird derjenige Winkel $\alpha$ gewählt, bei dem der Photostrom ein Minimum erreicht.
Die Messwerte aus \autoref{tab:daten} werden in \autoref{fig:plot} geplottet, sodass ein Minimum abgelesen werden kann.
Es ergibt sich ein Brewsterwinkel von $\alpha_{\text{B}}=\qty{75}{\degree}$, dieser wurde ebenfalls in \autoref{fig:plot} eingezeichnet.

\begin{figure}[H]
    \centering
    \includegraphics{build/plot.pdf}
    \caption{Plot der Messwerte sowie Theoriekurven für s- und p-polarisiertes Licht.}
    \label{fig:plot}
\end{figure}

Mithilfe des bestimmten Brewsterwinkels kann aus \eqref{eq:Brewsterwinkel} der Brechungsindex zu
\begin{gather*}
    n=\qty{3.73}{}
\end{gather*}
berechnet werden. Abschließend werden die Theoriekurven sowohl für das s- als auch für das p-polarisierte Licht
in \autoref{fig:plot} eingezeichnet. Dafür werden die in \autoref{sec:fresnel} berechneten Mittelwerte der Brechungsindizes
in die Fresnelschen Formeln \eqref{eq:Fresnel s} und \eqref{eq:Fresnel p} eingesetzt.