\section{Diskussion}
\label{sec:Diskussion}
Die experiementell bestimmten Brechungsindizes werden im Folgenden miteinander verglichen:
\begin{align*}
    \bar{n}_{\text{s}}&=\qty{3.42+-0.29}{} \\
    \bar{n}_{\text{p}}&=\qty{3.4+-0.4}{} \\
    n_{\alpha_\text{B}}&=\qty{3.73}{}
\end{align*}
In der Theorie ist für Licht mit $\lambda=\qty{633}{\nano\meter}$ in Silizium ein Brechungsindex von ca. 3,7 bis 4.0
zu erwarten.
Der über den Brewsterwinkel bestimmte Brechungsindex stimmt also gut mit der Theorie überein.
Allerdings weisen die über die Photoströme berechneten Werte eine größere Unsicherheit auf und sind zu gering.
Die bereits erwähnten systematischen Fehler konnten vermutlich trotz des Aussortierens grob abweichender
Werte nicht ausreichend kompensiert werden.
In \autoref{fig:plot} ist zu erkennen, dass die Messwerte zwar grob der Theoriekurve folgen, aber
stellenweise nicht zu vernachlässigende Abweichungen aufweisen.
Weitere Gründe für die Messfehler können die nicht exakte Winkelmessung an der Messapperatur sowie nicht konstante
Schattenverhältnisse im Raum sein.