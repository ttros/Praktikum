\section{Auswertung}
\label{sec:Auswertung}
In der Folgenden Tabelle sind Materialeigenschaften der Metallstäbe dargestellt.
\begin{table} [H]
  \centering
  \caption{Materialeigenschaften der Metallstäbe.}
  \label{tab:materialeigenschaften}
  \begin{tabular}{l S[table-format=1.0] S[table-format=1.1] S[table-format=1.1] S[table-format=4.0] S[table-format=3.0] S[table-format=3.0]}
    \toprule
    {Metall} & {Länge / $\unit{\centi\metre}$} & {Breite / $\unit{\centi\metre}$} & %
    {Dicke / $\unit{\centi\metre}$} & {$\rho / \frac{\unit{\kilo\gram}}{\unit{\cubic\metre}}$} &%
     {$c / \frac{\unit{\joule}}{\unit{\kilo\gram\kelvin}}$} & {$\kappa / \frac{\unit{\watt}}{\unit{\metre\kelvin}}$ \cite{czichos}}\\
    \midrule
    Messing (schmal)  & 9 & 0,7 & 0,4 & 8520 & 385 & 88  \\
    Messing (breit)   & 9 & 1,2 & 0,4 & 8520 & 385 & 88  \\
    Aluminium         & 9 & 1,2 & 0,4 & 2800 & 830 & 234 \\
    Edelstahl         & 9 & 1,2 & 0,4 & 8000 & 400 & 14  \\
    \bottomrule
  \end{tabular}
\end{table}

\subsection{Statische Methode}
\label{sec:auswertung statische methode}
In \autoref{fig:temperaturverlauf statisch, fern} ist der Temperaturverlauf der verschiedenen Metallstäbe zu sehen.
Dabei werden jeweils die Daten der fernen Thermoelemente verwendet.
Es ist eindeutig zu erkennen, dass die Temperatur des Aluminiumstabes am schnellsten steigt
und außerdem den höchsten Wert aller Stäbe erreicht. Außerdem wird deutlich, dass die Kurven der beiden Messingstäbe zu Beginn
nah beieinander liegen. Im Verlauf der Zeit steigt die Temperatur des breiten Messingstabes jedoch höher als die des schmalen
Messingstabes. Für die Temperatur des Edelstahlstabes ist zunächst kaum eine Erhöhung zu beobachten und auch nach einiger Zeit
ist der Edelstahlstab deutlich weniger erwärmt als die anderen Stäbe.

\begin{figure} [H]
  \centering
  \includegraphics{build/plot_statisch.pdf}
  \caption{Plot der Temperaturverläufe der fernen Thermoelemente.}
  \label{fig:temperaturverlauf statisch, fern}
\end{figure}

Wie auch schon in \autoref{fig:temperaturverlauf statisch, fern} zeigen sich nach $\qty{700}{\second}$ klare Unterschiede in den
Temperaturen der Stäbe.
\begin{table} [H]
  \centering
  \caption{Temperaturen der Stäbe nach $\qty{700}{\second}$.}
  \label{tab:Stabtemperaturen}
  \begin{tabular}{S[table-format=3.2] S[table-format=3.2] S[table-format=3.2] S[table-format=3.2]}
    \toprule
    {Messing (schmal)} & {Messing (breit)} & {Aluminium} & {Edelstahl} \\
    {$T / \unit{\kelvin}$} & {$T / \unit{\kelvin}$} &%
     {$T / \unit{\kelvin}$} & {$T / \unit{\kelvin}$}\\
    \midrule
    319,13 & 316,68 & 321,74 & 308,28 \\
    \bottomrule
  \end{tabular}
\end{table}

Es wird klar, dass Aluminium die beste und Edelstahl die schlechteste Wärmeleitfähigkeit~$\kappa$ hat.

Als nächstes wird der Wärmestrom in den Stäben für 5 verschiedene Messzeiten bestimmt.
In \autoref{tab:Temperaturdifferenzen} sind die Temperaturdifferenzen zwischen den zwei Messpunkten von allen
vier Stäben dargestellt.

\begin{table} [H]
  \centering
  \caption{Temperaturdifferenzen der Stäbe zu ausgewählten Zeitpunkten $t$.}
  \label{tab:Temperaturdifferenzen}
  \begin{tabular}{l S[table-format=1.2] S[table-format=1.2] S[table-format=1.2] S[table-format=1.2]}
    \toprule
    {} & {Messing schmal} & {Messing breit} & {Aluminium} & {Edelstahl} \\
    {$t / \unit{\second}$} & {$\symup{\Delta}T / \unit{\kelvin}$} & {$\symup{\Delta}T / \unit{\kelvin}$} &%
     {$\symup{\Delta}T / \unit{\kelvin}$} & {$\symup{\Delta}T / \unit{\kelvin}$}\\
    \midrule
    100 & 4,89 & 4,01 & 2,61 & 9,05 \\ 
    200 & 3,92 & 3,00 & 1,83 & 9,48 \\
    300 & 3,65 & 2,57 & 1,61 & 9,47 \\
    400 & 3,57 & 2,38 & 1,53 & 9,47 \\
    700 & 3,60 & 2,21 & 1,50 & 9,48 \\
    \bottomrule
  \end{tabular}
\end{table}

Der Wärmestrom lässt sich nun durch Umstellen von \eqref{eq:Wärmemenge} bestimmen.

\begin{equation}
  \frac{\symup{d}Q}{\symup{d}t} = -\kappa A\frac{\symup{\Delta} T}{\symup{\Delta} x} \notag
  \label{eq:Wärmestrom}
\end{equation}

Die Wärmeleitfähigkeit $\kappa$ sowie die Querschnittsfläche $A$ der Metalle werden dafür aus \autoref{tab:materialeigenschaften}
entnommen. Dabei ist der Abstand $\symup{\Delta}x=3\,\symup{cm}$ zwischen den Temperaturmessungen für alle Stäbe gleich.
Die Ergebnisse sind in \autoref{tab:Wärmestrom} dokumentiert.

\begin{table} [H]
  \centering
  \caption{Wärmestrom der Stäbe zu ausgewählten Zeitpunkten $t$.}
  \label{tab:Wärmestrom}
  \begin{tabular}{l S[table-format=-1.3] S[table-format=-1.3] S[table-format=-1.3] S[table-format=-1.3]}
    \toprule
    {} & {Messing schmal} & {Messing breit} & {Aluminium} & {Edelstahl} \\
    {$t / \unit{\second}$} & {$\left(\frac{\symup{d}Q}{\symup{d}t}\right) / \frac{\unit{\watt}}{\unit{\second}}$} & {$\left(\frac{\symup{d}Q}{\symup{d}t}\right) / \frac{\unit{\watt}}{\unit{\second}}$} &%
     {$\left(\frac{\symup{d}Q}{\symup{d}t}\right) / \frac{\unit{\watt}}{\unit{\second}}$} & {$\left(\frac{\symup{d}Q}{\symup{d}t}\right) / \frac{\unit{\watt}}{\unit{\second}}$}\\
    \midrule
    100 & -0.401 & -0.564  & -0.977 & -0.202 \\ 
    200 & -0.321 & -0.422  & -0.685 & -0.212 \\
    300 & -0.299 & -0.361  & -0.602 & -0.212 \\
    400 & -0.294 & -0.335  & -0.572 & -0.212 \\
    700 & -0.295 & -0.311  & -0.561 & -0.212 \\ 
    \bottomrule
  \end{tabular}
\end{table}

Abschließend werden für die statische Methode die Temperaturdifferenzen $T_{7}-T_{8}$ und $T_{2}-T_{1}$,
also die Differenzen in dem breiten Messingstab und Edelstahlstab, berechnet und gegen die Zeit $t$ aufgetragen.
Die resultierenden Verläufe sind in \autoref{fig:temperaturdifferenzen, messing, edelstahl} dargestellt.

Vergleicht man die beiden Temperaturdifferenzen fällt auf, dass diese sich zu Beginn nahezu identisch entwickeln.
Nach $50\,\symup{s}$ erreicht jedoch im Messingstab die Wärme den entfernteren Temperatursensor
und es findet ein Temperaturausgleich statt.

Somit verringert sich ab diesem Zeitpunkt das Temperaturgefälle im Messingstab aufgrund dessen vergleichsweise
hohen Wärmeleitfähigkeit, die, wie in \autoref{tab:materialeigenschaften} zu erkennen ist, fast vier mal größer ist
als die von Edelstahl.
Nach $500\,\symup{s}$ wird daher im Messingstab nahezu ein konstantes Temperaturgefälle erreicht.

Durch die geringe Wärmeleitfähigkeit von Edelstahl stellt sich ein Gleichgewicht erst nach  $150\,\symup{s}$ ein.
Die Temperaturdifferenz ist demtensprechend auch deutlich höher, da die Wärme wesentlich länger braucht
um das Stabende zu erreichen.


\begin{figure} [H]
  \centering
  \includegraphics{build/plot_T_7-8.pdf}
  \caption{Verlauf der Temperaturdifferenzen der fernen und nahen %
  Thermoelemente von Messing, breit und Edelstahl.}
  \label{fig:temperaturdifferenzen, messing, edelstahl}
\end{figure}

\subsection{Dynamische Methode}
Zur Bestimmung der Wärmeleitfähigkeit von Messing und Aluminium werden nun die gemessenen Temperaturen in \autoref{fig:temperaturwellen, messing} 
und \autoref{fig:temperaturwellen, aluminium} dargestellt. Dafür müssen aus den Grafiken die Amplituden und der
Phasendifferenz der Temperaturwellen abgelesen werden. Diese Werte werden dann in \eqref{eq:Kappa} eingesetzt und ergeben die
Wärmeleitfähigkeit~$\kappa$.
\begin{figure} [H]
  \centering
  \includegraphics{build/plot_dynamisch_kurz.pdf}
  \caption{Temperaturverlauf des breiten Messingstabes mit dynamischer Messmethode und %
  Periodendauer $T=\qty{80}{\second}$.}
  \label{fig:temperaturwellen, messing}
\end{figure}

\begin{table} [H]
  \centering
  \caption{Abgelesene Amplituden $A$ und Phasendifferenz $\symup{\Delta}t$ und berechnter Wärmeleitfähigkeit $\kappa$ für den %
  breiten Messingstab.}
  \label{tab:ergebnisse messing, breit}
  \begin{tabular}{S[table-format=1.0] S[table-format=1.0] S[table-format=2.1] S[table-format=2.2]}
    \toprule
    {$A_{\symup{nah}}/\unit{\kelvin}$} & {$A_{\symup{fern}}/\unit{\kelvin}$} & {$\symup{\Delta}t / \unit{\second}$} &%
     {$\kappa / \frac{\unit{\watt}}{\unit{\metre\kelvin}}$} \\
    \midrule
    6 & 2 & 14,3 & 94.05 \\
    \bottomrule
  \end{tabular}
\end{table}

\begin{figure} [H]
  \centering
  \includegraphics{build/plot_dynamisch_kurz_alu.pdf}
  \caption{Temperaturverlauf des Aluminiumstabes mit dynamischer Messmethode und %
  Periodendauer $T=\qty{80}{\second}$.}
  \label{fig:temperaturwellen, aluminium}
\end{figure}

\begin{table} [H]
  \centering
  \caption{Abgelesene Amplituden $A$ und Phasendifferenz $\symup{\Delta}t$ und berechnter Wärmeleitfähigkeit $\kappa$ für den %
  Aluminiumstab.}
  \label{tab:ergebnisse aluminium}
  \begin{tabular}{S[table-format=1.0] S[table-format=1.0] S[table-format=1.1] S[table-format=3.2]}
    \toprule
    {$A_{\symup{nah}}/\unit{\kelvin}$} & {$A_{\symup{fern}}/\unit{\kelvin}$} & {$\symup{\Delta}t / \unit{\second}$} &%
     {$\kappa / \frac{\unit{\watt}}{\unit{\metre\kelvin}}$} \\
    \midrule
    8 & 4 & 9,1 & 165,96 \\
    \bottomrule
  \end{tabular}
\end{table}

Die Vorgehensweise für die Bestimmung der Wärmeleitfähigkeit $\kappa$ von Edelstahl ist identisch zu der vorherigen, nur werden
nun die Daten der langen Periodendauer $T=\qty{200}{\second}$ betrachtet. Dies ist in \autoref{fig:temperaturwellen, edelstahl} 
dargestellt.
\begin{figure} [H]
  \centering
  \includegraphics{build/plot_dynamisch_lang.pdf} 
  \caption{Temperaturverlauf des Edelstahlstabes mit dynamischer Messmethode und %
  Periodendauer $T=\qty{200}{\second}$.}
  \label{fig:temperaturwellen, edelstahl}
\end{figure}

\begin{table} [H]
  \centering
  \caption{Abgelesene Amplituden $A$ und Phasendifferenz $\symup{\Delta}t$ und berechnter Wärmeleitfähigkeit $\kappa$ für den %
  Aluminiumstab.}
  \label{tab:ergebnisse edelstahl}
  \begin{tabular}{S[table-format=1.0] S[table-format=1.0] S[table-format=1.0] S[table-format=2.2]}
    \toprule
    {$A_{\symup{nah}}/\unit{\kelvin}$} & {$A_{\symup{fern}}/\unit{\kelvin}$} & {$\symup{\Delta}t / \unit{\second}$} &%
     {$\kappa / \frac{\unit{\watt}}{\unit{\metre\kelvin}}$} \\
    \midrule
    9 & 1,2 & 57 & 12.51 \\
    \bottomrule
  \end{tabular}
\end{table}