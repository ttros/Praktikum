\section{Auswertung}
\label{sec:Auswertung}
In der Folgenden Tabelle sind Materialeigenschaften der Metallstäbe dargestellt.
\begin{table} [H]
  \centering
  \caption{Materialeigenschaften der Metallstäbe.}
  \label{tab:materialeigenschaften}
  \begin{tabular}{l S[table-format=1.0] S[table-format=1.1] S[table-format=1.1] S[table-format=4.0] S[table-format=3.0] S[table-format=3.0]}
    \toprule
    {Metall} & {Länge / $\unit{\centi\metre}$} & {Breite / $\unit{\centi\metre}$} & %
    {Dicke / $\unit{\centi\metre}$} & {$\rho / \frac{\unit{\kilo\gram}}{\unit{\cubic\metre}}$} &%
     {$c / \frac{\unit{\joule}}{\unit{\kilo\gram\kelvin}}$} & {$\kappa / \frac{\unit{\watt}}{\unit{\metre\kelvin}}$ \cite{czichos}}\\
    \midrule
    Messing (schmal)  & 9 & 0,7 & 0,4 & 8520 & 385 & 88  \\
    Messing (breit)   & 9 & 1,2 & 0,4 & 8520 & 385 & 88  \\
    Aluminium         & 9 & 1,2 & 0,4 & 2800 & 830 & 234 \\
    Edelstahl         & 9 & 1,2 & 0,4 & 8000 & 400 & 25  \\
    \bottomrule
  \end{tabular}
\end{table}

\subsection{Statische Methode}
\label{sec:auswertung statische methode}
Der nachfolgende Plot zeigt den Verlauf der Temperaturen der verschiedenen Metallstäbe. Dabei werden jeweils die Daten der
fernen Thermoelemente verwendet. Es ist eindeutig zu erkennen, dass die Temperatur des Aluminiumstabes am schnellsten steigt
und außerdem den höchsten Wert aller Stäbe erreicht. Außerdem wird deutlich, dass die Kurven der beiden Messingstäbe zu Beginn
nah beieinander liegen. Im Verlauf der Zeit steigt die Temperatur des breiten Messingstabes jedoch höher als die des schmalen
Messingstabes. Für die Temperatur des Edelstahlstabes ist zunächst kaum eine Erhöhung zu beobachten und auch nach einiger Zeit
ist der Edelstahlstab deutlich weniger erwärmt als die anderen Stäbe.

\begin{figure}
  \centering
  \includegraphics{build/plot_statisch.pdf}
  \caption{Plot der Temperaturverläufe der fernen Thermoelemente.}
  \label{fig:temperaturverlauf statisch, fern}
\end{figure}

Wie auch schon in \autoref{fig:temperaturverlauf statisch, fern} zeigen sich nach $\qty{700}{\second}$ klare Unterschiede in den
Temperaturen der Stäbe.
\begin{table}
  \centering
  \caption{Temperaturen der Stäbe nach $\qty{700}{\second}$.}
  \label{tab:materialeigenschaften}
  \begin{tabular}{S[table-format=3.2] S[table-format=3.2] S[table-format=3.2] S[table-format=3.2]}
    \toprule
    {$T_{\symup{Messing (schmal)}} / \unit{\kelvin}$} & {$T_{\symup{Messing (breit)}} / \unit{\kelvin}$} &%
     {$T_{\symup{Aluminium}} / \unit{\kelvin}$} & {$T_{\symup{Edelstahl}} / \unit{\kelvin}$}\\
    \midrule
    319,13 & 316,68 & 321,74 & 308,28 \\
    \bottomrule
  \end{tabular}
\end{table}

Es wird klar, dass Aluminium die beste und Edelstahl die schlechteste Wärmeleitfähigkeit $\kappa$ hat.

\begin{figure}
  \centering
  \includegraphics{build/plot_T_7-8.pdf}
  \caption{Verlauf der Temperaturdifferenzen der fernen und nahen %
  Thermoelemente von Messing, breit und Edelstahl.}
  \label{fig:temperaturdifferenzen, messing, edelstahl}
\end{figure}

\begin{figure}
  \centering
  \includegraphics{build/plot_dynamisch_kurz.pdf}
  \caption{Temperaturverlauf des breiten Messingstabes mit dynamischer Messmethode und %
  Periodendauer $T=\qty{80}{\second}$.}
  \label{fig:temperaturdifferenzen kurz, messing, edelstahl}
\end{figure}

\begin{figure}
  \centering
  \includegraphics{build/plot_dynamisch_lang.pdf} 
  \caption{Temperaturverlauf des Edelstahlstabes mit dynamischer Messmethode und %
  Periodendauer $T=\qty{200}{\second}$.}
  \label{fig:temperaturdifferenzen lang, messing, edelstahl}
\end{figure}