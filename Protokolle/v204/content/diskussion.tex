\section{Diskussion}
\label{sec:Diskussion}
Generell gilt es bei diesem Versuch zu beachten, dass die Isolation der Metallstäbe nicht sonderlich gut ist, wodurch ein Wärmeaustausch mit der 
Umgebungswärme stattfindet, was wiederum die Messwerte beeinflusst.
In \autoref{tab:ergebnisse mit literaturwerten} werden die Ergebnisse für den Wärmeleitfähigkeit $\kappa$ der verschiedenen Materialien 
dargestellt und mit Literaturwerten verglichen. Es fällt auf, dass die Werte für Messing und Edelstahl in guter Übereinstimmung 
zu den Literaturwerten stehen, wohingegen der Wert für Aluminium stark abweicht. Dies könnten eventuelle Ablesefehler aus dem Plot
der Temperaturenwellen erklären. Auch denkbar wäre es, dass es sich bei dem Aluminiumstab um eine Aluminiumlegierung handelt. Werte für
die Wärmeleitfähigkeit von Aluminiumlegierungungen liegen laut \cite{czichos} bei
\begin{equation}
    \kappa_{\symup{Al-Legierung}} = (121\,...\,230)\,\unit{\watt\per\metre\kelvin}.
\end{equation}
\begin{table} [H]
    \centering
    \caption{Ergebnisse der Wärmeleitfähigkeit $\kappa$ im Vergleich zu Literaturwerten.}
    \label{tab:ergebnisse mit literaturwerten}
    \begin{tabular}{l S[table-format=3.2] S[table-format=3.0] S[table-format=2.1]}
      \toprule
      {Material} & {$\kappa_{\symup{exp}} / \frac{\unit{\watt}}{\unit{\metre\kelvin}}$} &%
      {$\kappa_{\symup{lit}} / \frac{\unit{\watt}}{\unit{\metre\kelvin}}$} & {Relativer Fehler} \\
      \midrule
        Messing     & 94.05  & 88  & 6,9\,\% \\ 
        Aluminium   & 165,96 & 234 & 29,1\,\%\\
        Edelstahl   & 12,51  & 14  & 10,1\,\%\\
      \bottomrule
    \end{tabular}
  \end{table}