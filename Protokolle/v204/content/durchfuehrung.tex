\section{Durchführung}
\label{sec:Durchführung}

\subsection{Statische Methode}
\label{sec:Statische Methode}

In der ersten Messreihe wird das Peltierelement mit einer Spannung von $5\,\symup{V}$ und maximal möglichen Strom versorgt.
Die Stäbe werden nun so lange erhitzt, bis $T_{7}$ eine Temperatur von $45\,\symup{°C}$ erreicht hat.
Dabei werden in einem Intervall von $5\,\symup{s}$ jeweils alle acht Temperaturen mit dem \textit{Xplorer GLX}
erfasst und zur Auswertung in einer .txt Datei gespeichert.
Nach der Messung werden die Stäbe auf unter $30\,\symup{°C}$ gekühlt, bevor mit dem nächsten Schritt begonnen wird.

Während allen Messungen wird eine Wärmeisolierung auf den Stäben platziert, um den Wärmeaustausch mit der Umgebung zu minimieren.

\subsection{Dynamische Methode}
\label{sec:Dynamische Methode}

Die Metallstäbe werden nun periodisch geheizt und gekühlt, indem der Schiebeschalter nach der Häflte der angestrebten
Periodendauer von \glqq Heat\grqq{} auf \glqq Cool\grqq{} gestellt wird und umgekehrt.
Die Abtastsrate der Temperaturen beträgt diesmal $2\,\symup{s}$, dabei wird eine Spannung von $8\,\symup{V}$ verwendet.

Im ersten Durchgang wird eine Periodendauer von $80\,\symup{s}$ angesetzt. Die Stäbe werden nun so lange erhitzt,
bis ca. zehn Perioden vergangen sind.
Anschließend werden die Stäbe zur Vorbereitung auf die nächste Messung wieder wie in \autoref{sec:Statische Methode} gekühlt.

Für den zweiten Durchgang beträgt die Periodendauer $200\,\symup{s}$, dabei werden die Stäbe erhitzt, bis eine der Temperaturen
$70\,\symup{°C}$ ereicht hat. Zur Auswertung werden die Daten wie in \autoref{sec:Statische Methode} in einer .txt Datei gespeichert.