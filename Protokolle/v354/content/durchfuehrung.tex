\section{Durchführung}
\label{sec:Durchführung}

Für den Versuch wird ein RCL-Kreis mit den folgenden Größen verwendet:

\begin{align*}
    R_{1} &= \qty{48.1+-0.1}{\ohm} \\ \notag
    R_{2} &= \qty{509.5+-0.5}{\ohm} \\ \notag
    C &= \qty{2.093+-0.003}{\nano\farad} \\ \notag
    L &= \qty{10.11+-0.03}{\milli\henry} \notag
\end{align*}

Dabei kann entweder $R_{1}$ oder $R_{2}$ verwendet werden.
Außerdem existiert ein dritter Widerstand, der von $1\,\unit{\kilo\ohm}$ bis $10\,\unit{\kilo\ohm}$ frei regelbar ist.

\subsection{Gedämpfter RCL-Kreis}
\label{sec:Gedämpfter RCL-Kreis}

Im ersten Versuchsteil wird eine Rechteckspannung verwendet. Deren Periode wird so gewählt, dass die Amplitude der Ausgangsspannung
mindestens um den Faktor 3 abfällt.
Für die Dämpfung wird hier der kleinere Widerstand $R_{1}$ genutzt.

Das Oszilloskopbild wird festgehalten und anschließend verwendet, um Wertepaare der Amplitude und Zeit abzulesen.
Mithilfe dieser Daten kann dann der effektive Dämpfungswiderstand $R_{\symup{eff}}$ bestimmt werden.

\subsection{Aperiodischer Grenzfall}
\label{sec:Aperiodischer Grenzfall Durchführung}

Es soll nun der Widerstand $R_{\symup{ap}}$ bestimmt werden, bei dem der aperiodische Grenzfall eintritt.
Dafür wird der selbe Versuchsaufbau wie in \autoref{sec:Gedämpfter RCL-Kreis} verwendet, wobei der Widerstand $R_{1}$
durch den regelbaren Widerstand ersetzt wird.

Dieser wird zu Beginn auf dem maximalen Wert, also $10\,\unit{\kilo\ohm}$ eingestellt, sodass sich ein überdämpfter Spannungsverlauf ergibt.
Nun verringert man den Widerstand, bis auf dem Oszilloskop eine Überschwingung zu erkennen ist.
Zuletzt wird der Widerstand wieder erhöht, bis die Überschwingung gerade verschwindet.
Dann kann der eingestellte Widerstand $R_{\symup{ap}}$ abgelesen werden.

\subsection{Angeregter RLC-Kreis}
\label{sec:Angeregter RLC-Kreis}

Für den letzten Versuchtsteil wird der RCL-Kreis an einen Frequenzgenerator mit einem Innenwiderstand von $50\,\unit{\ohm}$ angeschlossen.
Dieser liefert hier eine Sinusförmige Wechselspannung mit frei einstellbarer Frequenz.

Untersucht werden soll nun die Frequenzabhängigkeit der Spannungsamplitude und des Phasenversatzes des Ausgangssignals.
Dafür wird die Frequenz ausgehend von $10\,\unit{\kilo\hertz}$ schrittweise auf $50\,\unit{\kilo\hertz}$ erhöht.
Für jede einzelne Frequenz wird dann jeweils die Amplitude der Ausgangsspannung sowie der Phasenversatz zwischen Eingangs-
und Ausgangssignal dem Oszilloskopbild entnommen.

Für die genaue Bestimmung des Phasenversatzes werden am Oszilloskop der Abstand $a$ der Nulldurchgänge sowie die Periodendauer $b$
abgelesen, aus denen dann der Phasenversatz berechnet werden kann.