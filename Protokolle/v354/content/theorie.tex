\section{Einleitung}
\label{sec:Einleitung}
In diesem Versuch werden verschiedene Eigenschaften eines Schwingkreises mit $R$-, $C$-, und $L$-Gliedern untersucht. Ziel 
ist es, die Zeitabhängigkeit der Amplitude im Schwingfall zu ermitteln, den aperiodischen Grenzfall zu erreichen, und einen
Resonanzfall zu betrachten.

\section{Theorie}
\label{sec:Theorie}
Ein Schwingkreis besteht aus einer Spule mit der Induktivität~$L$, einem Kondensator mit der Kapazität~$C$, und einem 
Widerstand~$R$. In einem idealen Schwingkreis ohne Widerstand $R$ gilt die Energieerhaltung, sodass die zur Verfügung stehende Energie zwischen
der Spule mit der Induktivität $L$ und dem Kondensator mit der Kapazität $C$ verlustfrei oszilliert.
\subsection{Gedämpfte Schwingungen}
In einem idealen $LC$-Kreis ohne Widerstand $R$ gilt die Energieerhaltung, sodass die zur Verfügung stehende Energie zwischen
der Spule mit der Induktivität $L$ und dem Kondensator mit der Kapazität $C$ verlustfrei oszilliert. Wird nun ein Widerstand
mit berücksichtigt, ist festzustellen, dass die Energie weiterhin oszilliert, die Amplituden der einzelnen Schwingungen 
jedoch exponenziell abnehmen
