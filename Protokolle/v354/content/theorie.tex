\section{Einleitung}
\label{sec:Einleitung}
In diesem Versuch werden verschiedene Eigenschaften eines Schwingkreises mit $R$-, $C$-, und $L$-Gliedern untersucht. Ziel 
ist es, die Zeitabhängigkeit der Amplitude im Schwingfall zu ermitteln, den aperiodischen Grenzfall zu erreichen, und einen
Resonanzfall zu betrachten.

\section{Theorie}
\label{sec:Theorie}
Ein Schwingkreis besteht aus einer Spule mit der Induktivität~$L$, einem Kondensator mit der Kapazität~$C$, und einem 
Widerstand~$R$. In einem idealen Schwingkreis ohne Widerstand $R$ gilt die Energieerhaltung, sodass die zur Verfügung stehende 
Energie zwischen der Spule mit der Induktivität $L$ und dem Kondensator mit der Kapazität $C$ verlustfrei oszilliert.
\subsection{Gedämpfte Schwingungen}
Wird ein Widerstand mit berücksichtigt, ist festzustellen, dass die Energie weiterhin oszilliert, die Amplituden der einzelnen 
Schwingungen jedoch exponenziell abnehmen. Hierbei wird von einer Dämpfung gesprochen und für den Strom ergibt sich folgende 
Differentialgleichung
\begin{equation*}
    \ddot{I}+\frac{R}{L}\dot{I}+\frac{1}{LC}I=0.
\end{equation*}
Eine allgemeine Lösung dieser Differentialgleichung lautet
\begin{equation}
    I(t) = \symup{e}^{-2\symup{\pi}\mu t} (A_{1}\symup{e}^{\symup{i}2\symup{\pi}\nu t} + A_{2}\symup{e}^{-\symup{i}2\symup{\pi}\nu t}).
    \label{eq:DGL fuer I komplex}
\end{equation}
Hierbei werden die Hilfvariablen $\mu$ und $\nu$ eingführt:
\begin{align*}
    \mu &= \frac{R}{4\symup{\pi}L} & \nu = \frac{1}{2\symup{\pi}}\sqrt{\frac{1}{LC}-\frac{R^{2}}{4L^{2}}}.
\end{align*}
Nun wird zwischen zwei Fällen unterschieden:
\\
\\
\textbf{1. Schwingfall} \\
Damit eine gedämpfte Schwingung auftritt, muss $\nu \in \mathbb{R}$ und der Radikant somit positiv sein. Dies wird erfüllt für
\begin{equation*}
    \frac{1}{LC} > \frac{R^{2}}{4L^{2}}.
\end{equation*}

In diesem Fall vereinfacht sich die Differentialgleichung \eqref{eq:DGL fuer I komplex} zu:
\begin{equation*}
    I(t)=A_{0} \symup{e}^{-2\symup{\pi}\mu t} \cos(2\symup{\pi}\nu t + \varphi).
\end{equation*}
Die Abklingzeit $T_{\symup{ex}}$ ist folglich
\begin{equation}
    T_{\symup{ex}}=\frac{1}{2\symup{\pi}\mu}=\frac{2L}{R}.
    \label{eq:Abklingzeit}
\end{equation}
\\
\\
\textbf{2. Aperiodischer Grenzfall} \\
In dem Fall, dass $\nu$ imaginär, also
\begin{equation*}
    \frac{1}{LC} < \frac{R^{2}}{4L^{2}}
\end{equation*}
gilt, fällt der Strom $I$ proportinal zu einer Exponetialfunktion:
\begin{equation*}
    I(t) /propto /symup{e}^{-2\symup{(\pi}\mu-\symup{i}2\symup{\pi}\nu)t}.
\end{equation*}
Für den Fall, dass der Radikant von $\nu$ genau null ist, also
\begin{equation*}
    \frac{1}{LC} = \frac{R^{2}}{4L^{2}}
\end{equation*}
fällt der Strom am schnellsten. Eben dieser Fall wird als \textit{aperiodischer Grenzfall} bezeichnet. Umgestellt nach $R$ ergibt sich
\begin{equation}
    R_{\symup{ap}} = 
\end{equation}