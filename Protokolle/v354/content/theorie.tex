\section{Einleitung}
\label{sec:Einleitung}
In diesem Versuch werden verschiedene Eigenschaften eines Schwingkreises mit $R$-, $C$-, und $L$-Gliedern untersucht. Ziel 
ist es, die Zeitabhängigkeit der Amplitude im Schwingfall zu ermitteln, den aperiodischen Grenzfall zu erreichen, und einen
Resonanzfall zu betrachten.

\section{Theorie}
\label{sec:Theorie}
Ein Schwingkreis besteht aus einer Spule mit der Induktivität~$L$, einem Kondensator mit der Kapazität~$C$, und einem 
Widerstand~$R$. In einem idealen Schwingkreis ohne Widerstand $R$ gilt die Energieerhaltung, sodass die zur Verfügung stehende 
Energie zwischen der Spule mit der Induktivität $L$ und dem Kondensator mit der Kapazität $C$ verlustfrei oszilliert.
\begin{figure} [H]
    \centering
    \includegraphics[height=5cm]{content/Bilder/Gedämpfter Schwingkreis.pdf}
    \caption{Schaltskizze eines gedämpften Schwingkreises. \cite{v354}}
    \label{fig:Schaltskizze Schwingkreis}
  \end{figure}

\subsection{Gedämpfte Schwingungen}
Wird ein Widerstand mit berücksichtigt, ist festzustellen, dass die Energie weiterhin oszilliert, die Amplituden der einzelnen 
Schwingungen jedoch exponenziell abnehmen. Hierbei wird von einer Dämpfung gesprochen und für den Strom ergibt sich folgende 
Differentialgleichung
\begin{equation*}
    \ddot{I}+\frac{R}{L}\dot{I}+\frac{1}{LC}I=0.
\end{equation*}
Eine allgemeine Lösung dieser Differentialgleichung lautet
\begin{equation}
    I(t) = \symup{e}^{-2\symup{\pi}\mu t} (A_{1}\symup{e}^{\symup{i}2\symup{\pi}\nu t} + A_{2}\symup{e}^{-\symup{i}2\symup{\pi}\nu t}).
    \label{eq:DGL fuer I komplex}
\end{equation}
Hierbei werden die Hilfvariablen $\mu$ und $\nu$ eingführt:
\begin{align*}
    2\symup{\pi}\mu &= \frac{R}{2L} & 2\symup{\pi}\nu = \sqrt{\frac{1}{LC}-\frac{R^{2}}{4L^{2}}}.
\end{align*}
Der effektive Widerstand $R_{\symup{eff}}$ ist demnach
\begin{equation}
    R_{\symup{eff}} = 4\symup{\pi}\mu L.
    \label{eq:effektiver Widerstand}
\end{equation}
Nun wird zwischen zwei Fällen unterschieden:
\\
\\
\textbf{1. Schwingfall} \\
Damit eine gedämpfte Schwingung auftritt, muss $\nu \in \mathbb{R}$ und der Radikant somit positiv sein. Dies wird erfüllt für
\begin{equation*}
    \frac{1}{LC} > \frac{R^{2}}{4L^{2}}.
\end{equation*}

In diesem Fall vereinfacht sich die Differentialgleichung \eqref{eq:DGL fuer I komplex} zu:
\begin{equation*}
    I(t)=A_{0} \symup{e}^{-2\symup{\pi}\mu t} \cos(2\symup{\pi}\nu t + \varphi).
\end{equation*}
Die Abklingzeit $T_{\symup{ex}}$ ist folglich
\begin{equation}
    T_{\symup{ex}}=\frac{1}{2\symup{\pi}\mu}=\frac{2L}{R}.
    \label{eq:Abklingzeit}
\end{equation}
\\
\\
\textbf{2. Aperiodischer Grenzfall} \\
In dem Fall, dass $\nu$ imaginär, also
\begin{equation*}
    \frac{1}{LC} < \frac{R^{2}}{4L^{2}}
\end{equation*}
gilt, fällt der Strom $I$ proportinal zu einer Exponetialfunktion:
\begin{equation*}
    I(t) \propto \symup{e}^{-2\symup{(\pi}\mu-\symup{i}2\symup{\pi}\nu)t}.
\end{equation*}
Für den Fall, dass der Radikant von $\nu$ genau null ist, also
\begin{equation*}
    \frac{1}{LC} = \frac{R^{2}}{4L^{2}}
\end{equation*}
fällt der Strom am schnellsten. Eben dieser Fall wird als \textit{aperiodischer Grenzfall} bezeichnet. Umgestellt nach $R$ ergibt sich
\begin{equation}
    R_{\symup{ap}} = 2\sqrt{\frac{L}{C}}.
    \label{eq:Widerstand aperiodischer Grenzfall}
\end{equation}

\subsection{Angeregte Schwingungen}
Für angeregte Schwingungen beschreibt wieder eine Differentialgleichung den Spannungsverlauf des Kondensators:
\begin{equation*}
    LC\ddot{U}_{\symup{C}}+RC\dot{U}_{\symup{C}}+U_{\symup{C}} = U_0\symup{e}^{\symup{i}\omega t}.
\end{equation*}
Gelöst wird diese Differentialgleichung durch
\begin{equation*}
    U_{\symup{C}}(\omega)=\frac{U_0(1-LC\omega^2-\symup{i}\omega RC)}{(1-LC\omega^2)^{2}+\omega^2R^2\omega^2}.
\end{equation*}
Somit ist die Phasenverschiebung zwischen $U_{\symup{C}}$ und der Erregerspannung~$U_{\symup{err}}=U_0\symup{e}^{\symup{i}\omega t}$
gegeben durch
\begin{equation}
    \varphi(\omega)=\arctan\left(\frac{-\omega RC}{1-LC\omega^2}\right).
    \label{eq:Phasenverschiebung}
\end{equation}
Die charakteristischen Phasenverschiebungungen von $\varphi_1 = \frac{\symup{\pi}}{4}$ und $\varphi_2 = \frac{3\symup{\pi}}{4}$ werden erreicht bei
\begin{align}
    \omega_1&=\frac{R}{2L}+\sqrt{\frac{R^2}{4L^2}+\frac{1}{LC}} \label{eq:omega 1} \\
    \omega_2&=-\frac{R}{2L}+\sqrt{\frac{R^2}{4L^2}+\frac{1}{LC}}. \label{eq:omega 2}
\end{align}

Um die Resonanzfrequenz des Schwingkreises zu erhalten muss zunächst der Betrag der Spannung untersucht werden:
\begin{equation}
    |U(\omega)|=|U_{\symup{C}}(\omega)| = \frac{U_0}{\sqrt{(1-LC\omega^2)^2+\omega^2R^2C^2}}.
    \label{eq:Spannung}
\end{equation}
Für $\omega \to 0$ und für $\omega \to \infty$ gilt
\begin{align*}
    \lim \limits_{\omega \to 0} (U(\omega)) &= U_0 & \lim \limits_{\omega \to \infty} (U(\omega)) &= 0.
\end{align*}
Die Resonanzfrequenz, also die Frequenz $\omega$ für die $U(\omega)$ maximal wird, lautet
\begin{equation}
    \omega_{\symup{res}} = \sqrt{\frac{1}{LC}-\frac{R^2}{2L^2}}.
    \label{eq:Resonanzfrequenz}
\end{equation}
Ebenfalls wichtig ist die so genannte Breite der Resonanzkurve. Diese ist ein Bereich um das Maximum der Resonanzkurve,
in dem der Betrag der Amplitude größer ist als das $\frac{1}{\sqrt{2}}$-Fache des Maximums. Die Bereite der Resonanzkurve
wird für $\frac{R}{L^2} \ll \omega_0^2$ angenähert durch
\begin{equation}
    \symup{\Delta}\omega = \omega_{+}-\omega_{-} \approx \frac{R}{L}.
    \label{eq:Breite Resonanzkurve}
\end{equation}
Zuletzt soll noch die Resonanzüberhöhung bzw. Güte erwähnt sein. Sie beschreibt den Faktor um den die Spannung am Maximum erhöht wird,
was im Fall $\omega=\omega_{\symup{res}}$ vorliegt.
Sie kann bestimmt werden durch die Formel
\begin{equation}
    q=\frac{\omega_0}{\symup{\Delta}\omega}
    \label{eq:Güte}
\end{equation}
wobei $\omega_{0}=\frac{1}{LC}$ die Frequenz des ungedämpften Schwingkreises beschreibt.