\section{Auswertung}
\label{sec:Auswertung}

\subsection{Zeitabhängigkeit der Amplitude}
\label{sec:Zeitabhängigkeit der Amplitude}

\begin{figure} [H]
  \centering
  \includegraphics[height=10cm]{content/Bilder/Aufgabe_a.pdf}
  \caption{Bild des Oszilloskopbildschirms aus dem die Höhe der Amplituden der einzelnen Peaks abgelesen wird.}
  \label{fig:aufgabe a}
\end{figure}

Sagen, dass Abstand zwischen Peaks immer \qty{27,27}{\micro\second}
\begin{table}
  \centering
  \caption{Amplitude der einzelnen Peaks aus \autoref{fig:aufgabe a}.}
  \label{tab:Aufgabe a}
  \begin{tabular}{S[table-format=1.1]}
    \toprule
    {$U\,/\,\unit{\volt}$} \\
    \midrule
    6.0 \\
    5.0 \\
    4.1 \\
    3.3 \\
    2.7 \\
    2.2 \\
    1.8 \\
    1.4 \\
    1.1 \\
    0.8 \\
    0.5 \\
    0.4 \\
    0.2 \\
    0.1 \\
    \bottomrule
  \end{tabular}
\end{table}

\begin{figure} [H]
  \centering
  \includegraphics{build/plot_a.pdf}
  \caption{Zeitabhängigkeit der Spannung}
  \label{fig:plot_a}
\end{figure}

\subsection{Aperiodischer Grenzfall}
\label{sec:Aperiodischer Grenzfall}

Nach der Vorgehensweise aus \autoref{sec:Aperiodischer Grenzfall Durchführung} ergibt sich für den gesuchten Widerstand \\
\begin{gather*}
  R_{\symup{ap}} = \qty{3,40+-0.01}{\kilo\ohm}. \\ \notag
\end{gather*}

Als Referenz wird außerdem der theoretische Wert berechnet, der sich aus \eqref{eq:Widerstand aperiodischer Grenzfall} ergibt.
Dessen Fehler bestimmt sich mit der gaußschen Fehlerfortpflanzung zu:

\begin{align*}
  \label{eqn:Fehler Rap}
  \symup{\Delta} R_{\text{ap}} &= \sqrt{\Bigl(\frac{\symup{d}R_{\text{ap}}}{\symup{d}L}\symup{\Delta} L \Bigr)^2+\Bigl(\frac{\symup{d}R_{\text{ap}}}{\symup{d}C}\symup{\Delta} C\Bigr)^2} \\
  &= \sqrt{\Bigl(\frac{1}{\sqrt{LC}}\symup{\Delta} L\Bigr)^2+\Bigl(\frac{\sqrt{LC}}{C^2}\symup{\Delta} C\Bigr)^2} \\
  &\approx 7\,\unit{\ohm}
\end{align*}

Insgesamt erhält man also aus der Theorie:
\begin{gather*}
  R_{\symup{ap, theoretisch}} = \qty{4,396+-0.007}{\kilo\ohm}. \\ \notag
\end{gather*}

\subsection{Frequenzabhängigkeit der Spannung}
\label{sec:Frequenzabhängigkeit der Spannung}

\begin{table} [H]
  \centering
  \caption{Messwertpaare der zur Frequenz $f$ zugehörigen Amplituden $U_{0}$, $U$ sowie der Relativamplitude $\frac{U}{U_{0}}$.}
  \label{tab:Aufgabe c}
  \begin{tabular}{S[table-format=2.0] S[table-format=1.1] S[table-format=2.1] S[table-format=1.1]}
    \toprule
    {$f\,/\,\unit{\hertz}$} & {$U_{0}\,/\,\unit{\volt}$} & {$U\,/\,\unit{\volt}$} & {$\frac{U}{U_{0}}$} \\
    \midrule
    10 & 3.6 & 4.0  & 1.1 \\
    15 & 3.6 & 4.2  &	1.2 \\
    20 & 3.6 & 5.2  & 1.4 \\
    22 & 3.6 & 6.0  & 1.7 \\
    24 & 3.6 & 6.6  & 1.8 \\
    26 & 3.6 & 7.6  & 2.1 \\
    28 & 3.6 & 9.0  & 2.5 \\
    29 & 3.6 & 10.0 & 2.8 \\
    30 & 3.6 & 11.0 & 3.1 \\
    31 & 3.6 & 12.5 & 3.5 \\
    32 & 3.6 & 13.0 & 3.6 \\
    33 & 3.6 & 14.0 & 3.9 \\
    34 & 3.6 & 13.8 & 3.8 \\
    35 & 3.6 & 13.0 & 3.6 \\
    36 & 3.6 & 12.0 & 3.4 \\
    37 & 3.6 & 11.0 & 3.1 \\
    38 & 3.6 & 9.5  & 2.6 \\
    40 & 3.6 & 7.6  & 2.1 \\
    42 & 3.6 & 6.0  & 1.6 \\
    45 & 3.6 & 4.4  & 1.2 \\
    47 & 3.6 & 3.8  & 1.1 \\
    \bottomrule
  \end{tabular}
\end{table}

\begin{figure} [H]
  \centering
  \includegraphics{build/plot_c.pdf}
  \caption{Frequenzabhängigkeit der Spannung}
  \label{fig:plot_c}
\end{figure}

\begin{figure} [H]
  \centering
  \includegraphics{build/plot_c_2.pdf}
  \caption{Linearer Plot der Messwerte als Kurve im Resonanzbereich}
  \label{fig:plot_c_2}
\end{figure}

\subsection{Frequenzabhängigkeit der Phasenverschiebung}
\label{sec:Frequenzabhängigkeit der Phasenverschiebung}

\begin{table} [H]
  \centering
  \caption{Messwertepaare der von der Frequenz $f$ abhängigen
  Phasenverschiebung $\varphi$, die aus dem Phasenversatz $a$ und der Periodenlänge $b$ berechnet wird.}
  \label{tab:aufgabe d}
  \begin{tabular}{S[table-format=2.0] S[table-format=1.1] S[table-format=2.0] S[table-format=1.3]}
    \toprule
    {$f\,/\,\unit{\hertz}$} & {$a\,/\,\unit{\micro\metre}$} & {$b\,/\,\unit{\micro\metre}$} & {$\varphi$} \\
    \midrule
    10 & 1.0 & 98 & 0.064 \\ 
    15 & 1.0 & 75 & 0.083 \\
    20 & 1.2 & 49 & 0.153 \\
    22 & 1.6 & 44 & 0.228 \\
    24 & 1.8 & 41 & 0.275 \\
    26 & 2.2 & 38 & 0.363 \\
    28 & 2.8 & 35 & 0.502 \\
    29 & 3.4 & 34 & 0.628 \\
    30 & 3.8 & 33 & 0.723 \\
    31 & 4.4 & 32 & 0.863 \\
    32 & 5.0 & 31 & 1.013 \\
    33 & 6.0 & 30 & 1.256 \\
    34 & 7.0 & 29 & 1.516 \\
    35 & 7.8 & 29 & 1.689 \\
    36 & 8.4 & 28 & 1.884 \\
    37 & 9.0 & 27 & 2.094 \\
    38 & 9.2 & 26 & 2.223 \\ 
    40 & 9.6 & 25 & 2.412 \\
    42 & 9.7 & 24 & 2.539 \\ 
    45 & 9.6 & 22 & 2.741 \\
    47 & 9.6 & 21 & 2.872 \\
    \bottomrule
  \end{tabular}
\end{table}

\begin{figure} [H]
  \centering
  \includegraphics{build/plot_d.pdf}
  \caption{Frequenzabhängigkeit der Phasenverschiebung}
  \label{fig:plot_d}
\end{figure}

\begin{figure} [H]
  \centering
  \includegraphics{build/plot_d_2.pdf}
  \caption{Linearer Plot der Messwerte als Kurve im Bereich um $\frac{\pi}{2}$}
  \label{fig:plot_d_2}
\end{figure}