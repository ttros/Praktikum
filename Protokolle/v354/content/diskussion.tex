\section{Diskussion}
\label{sec:Diskussion}
Der theoretische und der experiementelle Wert des Effektivwiderstands $R_{\symup{eff}}$ weichen um etwas mehr als den Faktor~2
voneinander ab. Eine mögliche Erklärung liegt darin, dass zwar der Innenwiderstand des Frequenzgenerators berücksichtigt wird, allerdings
die Leitungswiderstände vollständig unbetrachtet bleiben. Weiterhin können auch kleine Ablesefehler der Spannung der einzelnen Peaks auf
dem Oszilloskopbildschirm in \autoref{fig:aufgabe a} im Folgenden zu großen Unsicherheiten führen.
Diese Unsicherheiten werden besonders im Plot \autoref{fig:plot_a} deutlich, da hier der y-Achsenabschnitt der 
Ausgleichsgerade~\eqref{eq:Ausgleichsgerade} signifikant von null abweicht und die Messwerte teilweise stark vom Fit abweichen.
Die Abklingzeiten ergeben sich zu
\begin{align*}
    T_{\symup{ex, theoretisch}} &= \qty{206,12+-0,21}{\micro\second} \\
    T_{\symup{ex, experimentell}} &= \qty{95+-6}{\micro\second}. 
\end{align*}
Durch die eventuellen Ablesefehler ließen sich auch die um etwa Faktor~2 verschiedenen Abklingzeiten $T_{\symup{ex}}$ erklären.

Bei dem Widerstand, bei dem der aperiodische Grenzfall eintritt, weicht der experiementelle Wert um rund 29\,\% von dem theoretisch zu 
erwartenden Wert ab. Die Werte lauten hier
\begin{align*}
    R_{\symup{ap,theoretisch}} &= \qty{4,395+-0,007}{\kilo\ohm} \\
    R_{\symup{ap,experimentell}} &= \qty{3,40+-0,01}{\kilo\ohm}.
\end{align*} 
Dies lässt sich nicht mit den zu erwartenden Messunsicherheiten erklären, möglich sind auch hier Ablesefehler
an dem Oszilloskopbildschirm. Außerdem wird diese Messung nur einmal durchgeführt. Daher wäre eine Verbesserung bei mehrmaliger Messung
gegebenenfalls denkbar.

Die experimentellen Werte für die Breite der Resonanzkurve und die Resonanzüberhöhung liegen jeweils im Fehlerbereich von dem theoretischen
Wert:
\begin{align*}
    \symup{\Delta}f_{\symup{exp}} &= 8,8\,\unit{\kilo\hertz} \\
    \symup{\Delta}f_{\symup{theorie}} &= \qty{8.808+-0.008}{\kilo\hertz}
\end{align*}
\begin{align*}
    q_{\symup{theorie}} &= \num{3.928+-0.003} \\
    q_{\symup{exp}} &= \num{3,9}
\end{align*}

Die Messwerte für die Frequenzabhängigkeit der Phasenverschiebung weichen alle um weniger als 1\,\% von den Theoriewerten ab, liegen 
allerdings nicht in dem Fehlerbereich.
\begin{align*}
    f_{\symup{res, exp}} &= 34,3\,\unit{\kilo\hertz} \\
    f_{\symup{1,exp}} &= 30,4\,\unit{\kilo\hertz} \\
    f_{\symup{2,exp}} &= 39,4\,\unit{\kilo\hertz} \\ \\
    f_{\symup{res, theorie}} &= \qty{34.033+-0.001}{\kilo\hertz} \\
    f_{\symup{1,theorie}} &= \qty{30.473+-0.003}{\kilo\hertz} \\
    f_{\symup{2,theorie}} &= \qty{39.282+-0.004}{\kilo\hertz}.
\end{align*}
Aufgrund der geringen Abweichung ist auch hier von kleinen Ablesefehlern auszugehen.
