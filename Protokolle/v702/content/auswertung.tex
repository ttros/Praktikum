\section{Auswertung}
\label{sec:Auswertung}

Um die Messwerte präzise Auswerten zu können, muss zuerst die Nullrate berechnet werden.
Anschließend wird die Nullrate bei allen Auswertungsschritten von den tatsächlichen Werten abgezogen.

\subsection{Nullrate}

In einem Messintervall von $\symup{\Delta}t=\qty{600}{\second}$ wurden $184$ Impulse gemessen, daher
ergibt sich je Sekunde eine Nullrate von
\begin{gather*}
    N_0=\qty{0.3067}{\per\second} \\
\end{gather*}

\subsection{Vanadium}

\begin{table}
    \centering
    \begin{tabular}{c c}
      \toprule
      Messzeit $t/\unit{\second}$ & Zählrate $N$\\
      \midrule
       30 & 204 $\pm$ 14 \\
       60 & 155 $\pm$ 12 \\
       90 & 146 $\pm$ 12 \\
      120 & 162 $\pm$ 12 \\
      150 & 125 $\pm$ 11 \\
      180 & 145 $\pm$ 12 \\
      210 & 131 $\pm$ 11 \\
      240 & 101 $\pm$ 10 \\
      270 &  92 $\pm$  9 \\
      300 &  75 $\pm$  8 \\
      330 &  73 $\pm$  8 \\
      360 &  73 $\pm$  8 \\
      390 &  75 $\pm$  8 \\
      420 &  73 $\pm$  8 \\
      450 &  60 $\pm$  7 \\
      \bottomrule
    \end{tabular}
    \begin{tabular}{c c}
      \toprule
      Messzeit $t/\unit{\second}$ & Zählrate $N$\\
      \midrule
      480 & 53 $\pm$ 7 \\
      510 & 47 $\pm$ 6 \\
      540 & 56 $\pm$ 7 \\
      570 & 51 $\pm$ 6 \\
      600 & 46 $\pm$ 6 \\
      630 & 28 $\pm$ 4 \\
      660 & 32 $\pm$ 5 \\
      690 & 40 $\pm$ 6 \\
      720 & 37 $\pm$ 5 \\
      750 & 33 $\pm$ 5 \\
      780 & 27 $\pm$ 4 \\
      810 & 17 $\pm$ 3 \\
      840 & 19 $\pm$ 3 \\
      870 & 19 $\pm$ 3 \\
      900 & 22 $\pm$ 4 \\
      \bottomrule
    \end{tabular}
    \caption{Messwerte der Zählrate für Vanadium inklusive der jeweiligen Fehler.}
    \label{tab:Vn}
\end{table}

\begin{figure}[H]
    \centering
    \includegraphics[height=8cm]{build/Vanadium.pdf}
    \caption{Halblogarithmischer Plot der Messwerte für Vanadium inklusive Fehlerbalken und berechneter Ausgleichsgerade.}
    \label{fig:Vn}
\end{figure}

\subsection{Rhodium}

\begin{table}
    \centering
    \begin{tabular}{c c}
      \toprule
      Messzeit $t/\unit{\second}$ & Zählrate $N$\\
      \midrule
       30 & 204 $\pm$ 14 \\
       60 & 155 $\pm$ 12 \\
       90 & 146 $\pm$ 12 \\
      120 & 162 $\pm$ 12 \\
      150 & 125 $\pm$ 11 \\
      180 & 145 $\pm$ 12 \\
      210 & 131 $\pm$ 11 \\
      240 & 101 $\pm$ 10 \\
      270 &  92 $\pm$  9 \\
      300 &  75 $\pm$  8 \\
      330 &  73 $\pm$  8 \\
      360 &  73 $\pm$  8 \\
      390 &  75 $\pm$  8 \\
      420 &  73 $\pm$  8 \\
      450 &  60 $\pm$  7 \\
      \bottomrule
    \end{tabular}
    \begin{tabular}{c c}
      \toprule
      Messzeit $t/\unit{\second}$ & Zählrate $N$\\
      \midrule
      480 & 53 $\pm$ 7 \\
      510 & 47 $\pm$ 6 \\
      540 & 56 $\pm$ 7 \\
      570 & 51 $\pm$ 6 \\
      600 & 46 $\pm$ 6 \\
      630 & 28 $\pm$ 4 \\
      660 & 32 $\pm$ 5 \\
      690 & 40 $\pm$ 6 \\
      720 & 37 $\pm$ 5 \\
      750 & 33 $\pm$ 5 \\
      780 & 27 $\pm$ 4 \\
      810 & 17 $\pm$ 3 \\
      840 & 19 $\pm$ 3 \\
      870 & 19 $\pm$ 3 \\
      900 & 22 $\pm$ 4 \\
      \bottomrule
    \end{tabular}
    \caption{Messwerte der Zählrate für Vanadium inklusive der jeweiligen Fehler.}
    \label{tab:Rh}
\end{table}

\begin{figure}[H]
    \centering
    \includegraphics[height=8cm]{build/Rhodium.pdf}
    \caption{Halblogarithmischer Plot der Messwerte für Rhodium inklusive Fehlerbalken und berechneter Ausgleichsgeraden
    für den kürzeren und längeren Zerfall.}
    \label{fig:Rh}
\end{figure}