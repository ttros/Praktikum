\section{Auswertung}
\label{sec:Auswertung}

Um die Messwerte präzise Auswerten zu können, muss zuerst die Nullrate berechnet werden.
Anschließend wird die Nullrate bei allen Auswertungsschritten von den tatsächlichen Werten abgezogen.

\subsection{Nullrate}

In einem Messintervall von $\symup{\Delta}t=\qty{600}{\second}$ werden $184$ Impulse gemessen, daher
ergibt sich je zehn Sekunden eine Nullrate von
\begin{gather*}
    N_0=3,1\pm1,8 \frac{1}{10\,\unit{\second}} \\
\end{gather*}

\subsection{Vanadium}

In \autoref{tab:Vn} sind die nach jedem Messzeitintervall abgelesenen Zählraten festgehalten.
Die Fehler dieser Werte werden wegen der zugrunde liegenden Poissonverteilung mit $\sqrt{N}$ berechnet und sind
direkt mit angegeben.

Zur Bestimmung der Halbwertszeit von Vanadium wird zuerst die Zerfallskonstante $\lambda_\text{Vn}$ mithilfe einer linearen Regression
ermittelt.
Dafür werden die Daten in \autoref{fig:Vn} halblogarithmisch geplottet.
Die zugehörige Geradengleichung erhält man durch logarithmieren von \eqref{eq:Zerfall}, sie lautet dann
\begin{equation*}
  \ln(N(t)) = -\lambda t + \ln{N_{\symup{0}}}.
\end{equation*}
Die lineare Regression wird mithilfe der Python-Erweiterung \textit{scipy}\,\cite{scipy} durchgeführt, dabei
ergibt sich für die gesuchten Parameter
\begin{align*}
  \lambda_{\symup{V}} &= (3,28 \pm 0,16) \cdot 10^{-3} \unit{\per\second} \\
  N_{\symup{0, V}} &= 212 \pm 18.
\end{align*}
Abschließend kann die Halbwertszeit von Vanadium über \eqref{eq:Halbwertszeit} berechnet werden zu
\begin{gather*}
  T_{\symup{V}} = \qty{211+-10}{\second}.
\end{gather*}
Der Fehler der Halbwertszeit wird hierbei mithilfe der gaußschen Fehlerfortpflanzung bestimmt, die
benötigte Formel zur Berechnung von $\symup{\Delta}T$ ergibt sich zu
\begin{equation}
  \label{delta_t}
  \symup{\Delta}T = \frac{\ln(2)}{\lambda^2} \cdot \symup{\Delta} \lambda. 
\end{equation}

\begin{table} [H]
    \caption{Messwerte der Zählrate für Vanadium inklusive der jeweiligen Fehler.}
    \label{tab:Vn}
    \centering
    \begin{tabular}{c c}
      \toprule
      Messzeit $t/\unit{\second}$ & Zählrate $N$\\
      \midrule
       30 & 204 $\pm$ 14 \\
       60 & 155 $\pm$ 12 \\
       90 & 146 $\pm$ 12 \\
      120 & 162 $\pm$ 12 \\
      150 & 125 $\pm$ 11 \\
      180 & 145 $\pm$ 12 \\
      210 & 131 $\pm$ 11 \\
      240 & 101 $\pm$ 10 \\
      270 &  92 $\pm$  9 \\
      300 &  75 $\pm$  8 \\
      330 &  73 $\pm$  8 \\
      360 &  73 $\pm$  8 \\
      390 &  75 $\pm$  8 \\
      420 &  73 $\pm$  8 \\
      450 &  60 $\pm$  7 \\
      \bottomrule
    \end{tabular}
    \begin{tabular}{c c}
      \toprule
      Messzeit $t/\unit{\second}$ & Zählrate $N$\\
      \midrule
      480 & 53 $\pm$ 7 \\
      510 & 47 $\pm$ 6 \\
      540 & 56 $\pm$ 7 \\
      570 & 51 $\pm$ 6 \\
      600 & 46 $\pm$ 6 \\
      630 & 28 $\pm$ 4 \\
      660 & 32 $\pm$ 5 \\
      690 & 40 $\pm$ 6 \\
      720 & 37 $\pm$ 5 \\
      750 & 33 $\pm$ 5 \\
      780 & 27 $\pm$ 4 \\
      810 & 17 $\pm$ 3 \\
      840 & 19 $\pm$ 3 \\
      870 & 19 $\pm$ 3 \\
      900 & 22 $\pm$ 4 \\
      \bottomrule
    \end{tabular}
\end{table}

\begin{figure}[H]
    \centering
    \includegraphics[height=8cm]{build/Vanadium.pdf}
    \caption{Halblogarithmischer Plot der Messwerte für Vanadium inklusive Fehlerbalken und berechneter Ausgleichsgerade.}
    \label{fig:Vn}
\end{figure}

\subsection{Rhodium}

Analog zur Auswertung von Vanadium sind die Messwerte in \autoref{tab:Rh} samt Fehler festgehalten.

\begin{table} [H]
    \centering
    \caption{Messwerte der Zählrate für Rhodium inklusive der jeweiligen Fehler.}
    \label{tab:Rh}
    \begin{tabular}{c c}
      \toprule
      Messzeit $t/\unit{\second}$ & Zählrate $N$\\
      \midrule
      15  & 748 $\pm$ 27 \\
      30  & 591 $\pm$ 24 \\
      45  & 456 $\pm$ 21 \\
      60  & 342 $\pm$ 18 \\
      75  & 307 $\pm$ 17 \\
      90  & 245 $\pm$ 16 \\
      105 & 222 $\pm$ 15 \\
      120 & 166 $\pm$ 13 \\
      135 & 141 $\pm$ 12 \\
      150 & 124 $\pm$ 11 \\
      165 &  97 $\pm$ 10 \\
      180 & 101 $\pm$ 10 \\
      195 &  80 $\pm$ 9 \\
      210 &  76 $\pm$ 8 \\
      225 &  74 $\pm$ 8 \\
      240 &  65 $\pm$ 8 \\
      255 &  56 $\pm$ 7 \\
      270 &  47 $\pm$ 7 \\
      285 &  46 $\pm$ 6 \\
      300 &  42 $\pm$ 6 \\
      315 &  29 $\pm$ 5 \\
      330 &  54 $\pm$ 7 \\
      345 &  50 $\pm$ 7 \\
      360 &  34 $\pm$ 5 \\
      \bottomrule
    \end{tabular}
    \begin{tabular}{c c}
      \toprule
      Messzeit $t/\unit{\second}$ & Zählrate $N$\\
      \midrule
      375 &  27 $\pm$ 5 \\
      390 & 26 $\pm$ 5 \\
      405 & 26 $\pm$ 5 \\
      420 & 28 $\pm$ 5 \\
      435 & 25 $\pm$ 5 \\
      450 & 32 $\pm$ 5 \\
      465 & 20 $\pm$ 4 \\
      480 & 24 $\pm$ 4 \\
      495 & 27 $\pm$ 5 \\
      510 & 23 $\pm$ 4 \\
      525 & 22 $\pm$ 4 \\
      540 & 20 $\pm$ 4 \\
      555 & 25 $\pm$ 5 \\
      570 & 23 $\pm$ 4 \\
      585 & 14 $\pm$ 3 \\
      600 & 20 $\pm$ 4 \\
      615 & 24 $\pm$ 4 \\
      630 & 17 $\pm$ 4 \\
      645 & 24 $\pm$ 4 \\
      660 & 20 $\pm$ 4 \\
      675 & 13 $\pm$ 3 \\
      690 & 17 $\pm$ 4 \\
      705 & 15 $\pm$ 3 \\
      720 & 20 $\pm$ 4 \\
      \bottomrule
    \end{tabular}
\end{table}

Im Gegensatz zu Vanadium laufen hier zwei Zerfälle gleichzeitig ab, zum einen der schnellere Zerfall von
$\ce{^{104\text{i}} Rh}$ und zum anderen der langsamere Zerfall von $\ce{^104 Rh}$.
Die gemessenen Daten sind die Überlagerung der beiden Zerfälle, zur korrekten Auswertung müssen beide
Prozesse einzeln betrachtet werden.

\subsubsection{Zerfall von Rhodium-104}

Zuerst wird der langsamere Zerfall betrachtet. Ein halblogarithmischer Plot der Daten ist in \autoref{fig:Rh} zu sehen.
Da ab einem gewissen Zeitpunkt $t^*$ der Messung bereits ein
Großteil der kurzlebigeren $\ce{^{104\text{i}} Rh}$ Kerne zerfallen ist, ergibt sich im Plot ab diesem Zeitpunkt
dann nahezu eine Gerade, die nur noch den Zerfall von $\ce{^104 Rh}$ beschreibt.
Konkret weisen die Messdaten in \autoref{fig:Rh} ab $t^*=\qty{315}{\second}$ einen nahezu linearen Verlauf auf,
die gewählte Grenze ist auch im Plot eingezeichnet.

Für den Zerfall von $\ce{^104 Rh}$ kann also für alle $t>t^*$ analog wie bei Vanadium eine lineare Regression
berechnet werden, die Regressionsgerade ist im Plot eingezeichnet, für die Parameter liefert die Rechnung
\begin{align*}
  \lambda_{\ce{^104 Rh}} &= (2,5 \pm 0,4) \cdot 10^{-3} \unit{\per\second} \\
  N_{\symup{0, \ce{^104 Rh}}} &= 70 \pm 15.
\end{align*}
Mit \eqref{eq:Halbwertszeit} ergibt sich die Halbwertszeit hier zu
\begin{gather*}
  T_{\ce{^104 Rh}} = \qty{270+-40}{\second}.
\end{gather*}
Dabei wurde der Fehler $\symup{\Delta} T_{\ce{^104 Rh}}$ mit der bereits berechneten Formel
\eqref{delta_t} bestimmt.

\subsubsection{Zerfall von Rhodium-104i}

Um den Zerfall des kurzlebigeren $\ce{^{104\text{i}} Rh}$ zu analysieren, wird das Ergebnis für $\ce{^104 Rh}$
zur Hilfe gezogen.
Die Anzahl an $\ce{^104 Rh}$ Kernen kann mithilfe der dafür bestimmten Regressionsgeraden bis zum Anfang
der Messung extrapoliert werden. Betrachtet werden also nun Zeitpunkte $t_{\text{j}}$ ab dem Anfang der
Messung bis zu einem $t_{\text{max}}$, wobei $t_{\text{max}}<t^*$ gilt.
Hier wird $t_{\text{max}}=\qty{225}{\second}$ gewählt.
Um die Anzahl an $\ce{^{104\text{i}} Rh}$ Kernen in diesem Intervall zu erhalten, wird die Differenz
aus der gemessenen Gesamzahl und der extrapolierten Zahl der $\ce{^104 Rh}$ Kerne gebildet.
Der Übersichtilchkeit halber gelte hier $m=\lambda_{\ce{^104 Rh}}$ und $b=\ln(N_{\symup{0, \ce{^104 Rh}}})$.
\begin{align*}
  N_{\ce{^{104\text{i}} Rh}}(t_{\text{j}}) &= N_{\text{Ges}}(t_{\text{j}}) - N_{\ce{^104 Rh}}(t_{\text{j}}) \\
  &= N_{\text{Ges}}(t_{\text{j}}) - \symup{e}^{m \cdot t + b}
\end{align*}
Da die Anzahl der $\ce{^104 Rh}$ Kerne bereits fehlerbehaftet ist, muss für die Angabe des Fehlers der
$\ce{^{104\text{i}} Rh}$ Kerne die gaußsche Fehlerfortpflanzung verwendet werden.
Einsetzen in die Formel ergibt
\begin{align*}
  \symup{\Delta} N_{\symup{kurz}}(t) &= \sqrt{
    \left(\frac{\partial N_{\symup{kurz}}}{\partial N_{\symup{Ges}}}\right)^2 (\symup{\Delta} N_{\symup{Ges}})^2 +
    \left(\frac{\partial N_{\symup{kurz}}}{\partial m}\right)^2 (\symup{\Delta} m)^2 + 
    \left(\frac{\partial N_{\symup{kurz}}}{\partial b}\right)^2 (\symup{\Delta} b)^2} \\
  &= \sqrt{
    (\symup{\Delta} N_{\symup{Ges}})^2 +
    \left(te^{-mt+b}\right)^2 (\symup{\Delta} m)^2 +
    \left(e^{-mt+b}\right)^2 (\symup{\Delta} b)^2
    }\,. \\
\end{align*}

Mithilfe dieser neuen Daten kann eine lineare Regression für $\ce{^{104\text{i}} Rh}$ berechnet werden,
aus der sich folgende Parameter ergeben
\begin{align*}
  \lambda_{\ce{^{104\text{i}} Rh}} &= (15,6 \pm 0,5) \cdot 10^{-3} \unit{\per\second} \\
  N_{\symup{0, \ce{^{104\text{i}} Rh}}} &= 770 \pm 50.
\end{align*}
Die Halbwertszeit ergibt sich über Formel \eqref{eq:Halbwertszeit} zu
\begin{gather*}
  T_{\ce{^{104\text{i}} Rh}} = \qty{44.5+-1.4}{\second}.
\end{gather*}
Erneut wurde deren Fehler mit \eqref{delta_t} berechnet.

Abschließend wird in \autoref{fig:Rh} noch die Summenkurve eingezeichnet, diese ergibt sich aus der
Summe der beiden unterschiedlichen Kerne im gesamten Messzeitintervall:
\begin{equation*}
N_{\text{Ges}}(t) = N_{\ce{^104 Rh}}(t) +  N_{\ce{^{104\text{i}} Rh}}(t)
\end{equation*}
Dabei lassen sich die Anzahlen $N_{\ce{^104 Rh}}(t)$ und $N_{\ce{^{104\text{i}} Rh}}(t)$ mithilfe der
berechneten Regressionsgeraden über das gesamte Messzeitintervall extrapolieren.

\begin{figure}[H]
    \centering
    \includegraphics[height=8cm]{build/Rhodium.pdf}
    \caption{Halblogarithmischer Plot der Messwerte für Rhodium inklusive Fehlerbalken und berechneter Ausgleichsgeraden
    für den kürzeren und längeren Zerfall.}
    \label{fig:Rh}
\end{figure}