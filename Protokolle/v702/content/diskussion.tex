\section{Diskussion}
\label{sec:Diskussion}

Die berechneten Halbwertszeiten lassen sich mit Theoriewerten vergleichen, um die Genauigkeit der
Messung beurteilen zu können.
Zuerst wird das Ergebnis für Vanadium betrachtet.
\begin{align*}
    T_{\ce{^52 V}} &= \qty{211+-10}{\second} \\
    T_{\ce{^52 V}, \text{Theorie}} &= \qty{224}{\second}\,\text{\cite{Zerfall}}.
\end{align*}
Auch unter Berücksichtigung der Unsicherheit kann das experimentelle Ergebnis nicht ganz mit
dem Theoriewert vereinbart werden, die Abweichung liegt bei $\qty{5.33}{\percent}$.

Bei Rhodium wird ebenfalls ein Vergleich mit Literaturwerten durchgeführt.
\begin{align*}
    T_{\ce{^104 Rh}} &= \qty{270+-40}{\second} \\
    T_{\ce{^104 Rh}, \text{Theorie}} &= \qty{260}{\second}\,\text{\cite{Zerfall}} \\
    T_{\ce{^{104\text{i}} Rh}} &= \qty{44.5+-1.4}{\second} \\
    T_{\ce{^{104\text{i}} Rh}, \text{Theorie}} &= \qty{42.3}{\second}\,\text{\cite{104iRh}}.
\end{align*}
Das Ergebnis für $\ce{^104 Rh}$ ist mit dem Theoriewert vereinbar, allerdings liegt hier eine
relativ große Unsicherheit vor. Die Abweichung beträgt $\qty{3.85}{\percent}$.
Bei $\ce{^{104\text{i}} Rh}$ liegt das Ergebnis knapp über dem Literaturwert bei einer relativen
Abweichung von $\qty{4.94}{\percent}$.

Insgesamt liegen alle ermittelten Halbwertszeiten nah bei den Theoriewerten, allesamt mit Fehlern kleiner
als $\qty{10}{\percent}$. Die guten Ergebnisse lassen sich auch darauf zurückführen, dass es nur wenig
Störfaktoren während der Messung gibt, die Nullrate wird berücksichtigt.
Die größte Fehlerquelle ruht bei der Durchführung darin, dass der Zerfall nicht direkt zu Beginn gemessen
werden kann. Die aktivierte Probe muss erst zum Geiger-Müller-Zählrohr gebracht werden, was einige Sekunden
in Anspruch nimmt.