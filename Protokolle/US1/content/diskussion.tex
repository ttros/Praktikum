\section{Diskussion}
\label{sec:Diskussion}
Die theoretische Schallgeschwindigkeit in Acryl von rund $c=\qty{2730}{\metre\per\second}$ \cite{c_Acryl} wird mit gewissen Abweichungen
auch in diesem Versuch bestätigt.
\begin{table}[H]
    \centering
    \caption{Vergleich theoretische und experiementelle Schallgeschwindigkeit.}
    \begin{tabular}{l S[table-format=3.0] S[table-format=2.1]}
        \toprule
         {Messmethode} & {$c\,/\,\frac{\unit{\metre}}{\unit{\second}}$} & {Relative Abweichung} \\
        \midrule
           {Impuls-Echo-Verfahren}	&  {$\num{3190(220)}$} & {$\qty{16,85}{\percent}$}\\
           {Durchschallungs-Verfahren}	&  {$\num{2770(40)}$} & {$\qty{1,47}{\percent}$}\\
        \bottomrule 
    \end{tabular}
  \end{table}
Der experiementelle Wert des Impuls-Echo-Verfahrens lässt sich trotz der großen Messunsicherheit nicht mit dem theoretischen Wert bestätigen. 
Ein möglicher Grund hierfür ist die mangelhafte Kopplung zweier Zylinder oder nicht betrachtete Unsicherheiten bei dem Ablesen der Messungen.

Im Gegensatz dazu liegt der theoretische Wert  des Durchschallungsverfahrens im Fehlerbereich des experiementell ermittelten Wertes. Bei diesem Verfahren
werden die Zylinder nur einzeln gemessen und nicht untereinander nochmals gekoppelt.

Für den Dämpfungskoeffizienten wird der Wert $\alpha = \qty{13.8(1.3)}{\per\metre}$ bestimmt. Bei der Bestimmung dieses Wertes werden die
Acrylzylinder ebenfalls gekoppelt. Folglich ist dieser Wert vermutlich fehlerbehaftet, ein theoretischer Wert zum Vergleich ist nicht gegeben.

Die Werte der Augenvermessung lassen sich nicht exakt überprüfen, scheinen aber im Sachzusammenhang realistisch.
