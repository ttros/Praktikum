\section{Zielsetzung}
Ziel des Versuches ist, die Grundlagen der Ultraschalltechnik zu verstehen und anzuwenden.
Dafür wird die Schallgeschwindigkeit in Acryl sowie die Schalldämpfung untersucht, ausßerdem
wird ein Augenmodell mithilfe der Ultraschalltechnik vermessen.

\section{Theorie}
\label{sec:Theorie}

\subsection{Ultraschall}
\label{sec:Ultraschall}

Bei Schall handelt es sich allgemein um eine longitudinale Welle, welche sich aufgrund von Druckschwankungen
in einem Medium fortbewegt.
Das menschliche Gehöhr kann diese in einem Frequenzbereich von ca. $16\,\unit{\hertz}$ bis $20\,\unit{\kilo\hertz}$
wahrnehmen. Schallwellen mit höhren Frequenzen zwischen $20\,\unit{\kilo\hertz}$ und $1\,\unit{\giga\hertz}$ werden als
Ultraschall bezeichnet. \\
Die Ultraschallwellen breiten sich mit einer vom Medium abhängigen Schallgeschwindigkeit $c$ aus.
Weiterhin ist die akustische Impedanz (auch Schallkennwiderstand) $Z=\rho \cdot c$ von Interesse, sie hängt
von der Schallgeschwindigkeit sowie von der Dichte des Mediums ab. \\
Während der Ausbreitung verlieren die Schallwellen aufgrund von Absorption an Energie, ihre Intensität nimmt ab.
Es besteht ein exponentieller Zusammenhang zwischen der Intensität und der zurückgelegten Strecke:
\begin{equation}
    \label{eq:Dämpfung}
    I(x)=I_{0}\symup{e}^{-\alpha x}
\end{equation}
Aufgrund der hohen Absorption von Luft wird zwischen Schallgeber und Material ein Kontaktmittel verwendet.

Zur Erzeugung von Ultraschall wird unter anderem der sogenannte reziproke piezo-elektrischen Effekt verwendet.
Piezo-elektrische Kristalle schwingen in elektrischen Wechselfeldern und strahlen dabei Ultraschallwellen ab.
Der selbe Effekt funktioniert umgekehrt, somit kann ein solcher Kristall auch als Empfänger verwendet werden,
der abhängig von der Schallintensität ein elektrisches Signal abgibt.

\subsection{Messverfahren}
\label{sec:Messverfahren}
Um Informationen über das zu untersuchende Material mithilfe von Ultraschall zu gewinnen, werden unter anderem
Laufzeitmessungen durchgeführt. Dabei gibt es zwei wesentliche Verfahren.

Beim \textit{Durchschallungs-Verfahren} wird mit einem Sender ein kurzzeitiger Schallimpuls in die Probe gegeben.
Am anderen Ende der Probe wird das Signal dann wieder empfangen, aus der Laufzeit lässt sich die Länge berechnen:
\begin{equation}
    \label{eq:Durchschall}
    s=c\cdot t
\end{equation}

Die andere Möglichkeit nennt sich \textit{Impuls-Echo-Verfahren}. Hier wird der Sender gleichzeitig als Empfänger verwendet.
Der gesendete Impuls refelektiert am Ende der Probe und wird so wieder empfangen.
Für die Längenberechnung zählt hier daher nur die halbe Zeit:
\begin{equation}
    \label{eq:Impuls-Echo}
    s=\frac{1}{2} c\cdot t
\end{equation}