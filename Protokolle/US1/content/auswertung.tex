\section{Auswertung}
\label{sec:Auswertung}

\subsection{Verifizierung des Messverfahrens}
\label{sec:Verifizierung des Messverfahrens}
Damit die Funktionalität des Versuchaufbaus überprüft werden kann, wird die $\qty{2}{\mega\hertz}$ Ultraschallsonde mit bidestilliertem Wasser an 
eine Acrylplatte mit bereits bekannter Dicke $d=\qty{10}{mm}$ gekoppelt. Aus \autoref{fig:Geräteeinstellung} wird die doppelte Laufzeit der 
Ultraschallwelle zu $\symup{\Delta}\tilde{t}=\qty{8}{\micro\second}$ bestimmt. Mit \eqref{eq:Impuls-Echo} wird somit die Schallgeschwindigkeit in Acryl
zu $c=\qty{2500}{\metre\per\second}$ bestimmt.
\begin{figure}[H]
  \centering
  \includegraphics[height=6.5cm]{content/Abbildungen/Geräteeinstellung.pdf}
  \caption{Ausgabe der Vermessung der $\qty{10}{mm}$ Acrylplatte.}
  \label{fig:Geräteeinstellung}
\end{figure}

% $c$-Bestimmung mit Impuls-Echo-Verfahren
\subsection{Messung der Schallgeschwindigkeit mit Impuls-Echo-Verfahren}
\label{sec:c_IE}
Für diesen Teil des Versuches werden Zylinder übereinandergestellt und mit bidestilliertem Wasser gekoppelt, um so mehrere verschiedene Längen 
zu erreichen. Die Laufzeiten der Schallwellen werden abhängig von der Länge der Zylinder in \autoref{tab:c_IE} aufgeführt und in \autoref{fig:c_IE}
graphisch dargestellt.

Nach \eqref{eq:Durchschall} sollte ein linearer Zusammenhang vorliegen. Mithilfe der \textit{python}-Erweiterung \textit{scipy}\cite{scipy} wird eine
lineare Ausgleichsrechnung durchgeführt\footnote{Anm: Der grau hinterlegte Messwert %
wurde nicht bei der linearen Regression betrachtet, da hier sehr wahrscheinlich ein falscher Peak abgelesen wurde.}, welche für die Geradengleichung

\begin{equation*}
  l(\symup{\Delta}t) = m \cdot \symup{\Delta}t + b
\end{equation*}

folgende Parameter liefert:
\begin{align*}
  m_{\symup{IE}} &= \qty{313(21)e-6}{\second\per\metre}\\
  b_{\symup{IE}} &= \qty{0.4(2.1)e-5}{\metre} \\
\end{align*}
Daraus lässt sich schließen, dass die Schallgeschwindigkeit in Acryl hier zu
\begin{equation*}
  c_{\symup{IE}} = \frac{1}{m} = \qty{3.19(0.22)e3}{\metre\per\second}
\end{equation*}
bestimmt wird.
Der Parameter $b$ spiegelt hier die Dicke der Anpassungschicht wider.

\begin{table}[H]
  \centering
  \caption{Daten $c$-Bestimmung mit Impuls-Echo-Verfahren.}
  \label{tab:c_IE}
  \begin{tabular}{S[table-format=3.1] S[table-format=3.0] S[table-format=2.1]}
      \toprule
       {$l\,/\,\unit{\milli\metre}$} & {$\symup{\Delta}\tilde{t}\,/\,\unit{\micro\second}$} & {$\symup{\Delta}t\,/\,\unit{\micro\second}$} \\
      \midrule
         40,4	&  30 & 15,0\\
         61,5	&  46 & 23,0\\
         80,5	&  60 & 30,0\\
        120,5	&  88 & 44,0\\
        101,9	&  75 & 37,5\\
        160,9	& 104 & 52,0\\
        142,0	&  75 & 37,5\\ 
      \bottomrule 
  \end{tabular}
\end{table}

\begin{figure}[H]
  \centering
  \includegraphics{c-Bestimmung_IE.pdf}
  \caption{Graphische Darstellung der Messwertpaare aus \autoref{tab:c_IE} mit Ausgleichsgerade.}
  \label{fig:c_IE}
\end{figure}

% $c$-Bestimmung mit Durchschallungs-Verfahren
\subsection{Messung der Schallgeschwindigkeit mit Durchschallungs-Verfahren}
Die Auswertung der Messwerte erfolt analog zu \ref{sec:c_IE}. Hier liefert die lineare Ausgleichsrechnung die Werte
\begin{align*}
  m_{\symup{D}} &= \qty{361(5)e-6}{\second\per\metre} \\
  b_{\symup{D}} &= \qty{1(5)e-6}{\metre}. \\
\end{align*}
Daraus folgt für die Schallgeschwindigkeit
\begin{equation*}
  c_{\symup{D}} = \qty{2.77(0.04)e+03}{\metre\per\second}.
\end{equation*}

\begin{table}[H]
  \centering
  \caption{Daten $c$-Bestimmung mit Durchschallungs-Verfahren.}
  \label{tab:c_D}
  \begin{tabular}{S[table-format=3.1] S[table-format=2.0]}
      \toprule
       {$l\,/\,\unit{\milli\metre}$} & {$\symup{\Delta}t\,/\,\unit{\micro\second}$} \\
      \midrule
         40,4	& 15\\
         61,5	& 23\\
         80,5	& 30\\
        120,5	& 44\\
      \bottomrule 
  \end{tabular}
\end{table}

\begin{figure}[H]
  \centering
  \includegraphics{c-Bestimmung_D.pdf}
  \caption{Graphische Darstellung der Messwertpaare aus \autoref{tab:c_D} mit Ausgleichsgerade.}
  \label{fig:c_D}
\end{figure}

% Messung der Dämpfung mit Impuls-Echo-Verfahren
\subsection{Messung der Dämpfung mit Impuls-Echo-Verfahren}
Für die Bestimmung des Dämpfungskoeffizienten $\alpha$ wird das Verhältnis der Amplituden der ausgesendeten und der einfallenden Ultraschallwelle
gebildet. Dieses Verhältnis wird logarithmisch gegenüber der doppelten Länge der Acrylzylinder aufgetragen und es wird eine lineare Ausgleichsrechnung 
durchgeführt.\footnote{Anm: Die grau hinterlegten Messwerte wurden für die Ausgleichsrechnung nicht berücksichtigt, da hier davon auszugehen ist, dass 
Kopplung zweier Zylinder nicht ausreichend gut war. Dies hatte vermutlich zur Folge, dass der Peak der eigentlichen Reflexion am Ende des Zylinders
im Hintergrundrauschen untergegangen ist.}
\begin{align*}
  m_{\symup{Dämpfung}} &= \qty{-13.8(1.3)}{\per\metre} \\
  b_{\symup{Dämpfung}} &= \qty{0.9(1.3)}{\metre}
\end{align*}
Nach \eqref{eq:Dämpfung} liegt ein exponenzieller Zusammenhang zwischen Signalstärke und zurückgelegter Strecke vor. Folglich liefert die 
Steigung der linearen Regressiongeraden den Dämpfungskoeffizienten $\alpha$.
\begin{equation}
    \alpha = \qty{13.8(1.3)}{\per\metre}
\end{equation}

\begin{table}[H]
  \centering
  \caption{Daten Dämpfungsbestimmung mit Impuls-Echo-Verfahren.}
  \label{tab:Dämpfung}
  \begin{tabular}{S[table-format=3.1] S[table-format=1.2] S[table-format=1.2]}
      \toprule
       {$l\,/\,\unit{\milli\metre}$} & {$A_{\symup{out}}\,/\,\unit{\volt}$} & {$A_{\symup{in}}\,/\,\unit{\volt}$} \\
      \midrule
         40,4	& 0,84 & 0,63\\
         61,5	& 1,00 & 0,50\\
         80,5	& 1,16 & 0,41\\
        120,5	& 1,22 & 0,11\\
        101,9	& 1,23 & 0,17\\
        160,9	& 1,23 & 0,32\\
        142,0	& 1,23 & 0,30\\ 
      \bottomrule 
  \end{tabular}
\end{table}

\begin{figure}[H]
  \centering
  \includegraphics{Daempfungsbestimmung.pdf}
  \caption{Logarithmische Darstellung des Amplitudenverhältnisses aus \autoref{tab:Dämpfung} mit Ausgleichsgerade.}
  \label{fig:Dämpfung}
\end{figure}

\subsection{Vermessung des Augenmodells}