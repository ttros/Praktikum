\section{Diskussion}
\label{sec:Diskussion}
Bei der Bestimmung der maximalen Strahlungsreichweite des $\beta^{-}$-Strahlers werden $\d_{\text{max}}$ 
und die dazugehörige maximale Energie $E_{\text{max}}$ zu
\begin{align*}
    d_{\text{max}} &= \qty{252+-34}{\micro\metre} & E_{\text{max}} &= \qty{0.075+-0.005}{\mega\electronvolt}
\end{align*}
bestimmt. Ein Literaturwert für die maximale Energie des $\beta^{-}$-Strahlers $\ce{^99 Tc}$ lautet 
$E_{\text{max, lit}} = \qty{0,294}{\mega\electronvolt}$. Somit ergibt sich eine relative Abweichung von
rund $\qty{74,5}{\percent}$.
Aufgrund der sehr großen relativen Abweichung ist von einem systematischen Fehler bei der Datenaufnahme auszugehen.
Möglich ist, dass durch eine zu kurz angelegte Nullmessung die Hintergrundstrahlung nicht ausreichend genau 
bestimmt wird.

Die experimentellen Werte für die Bestimmung der Absorptionskoeffizienten von Zink und Blei sind in 
\autoref{tab:diskussion} den theoretisch ermittelten Compton Absorptionskoeffizienten gegenübergestellt.

\begin{table} [H]
    \centering
    \caption{Experimentelle und theoretische Werte zu den (Compton-)Absorptionskoeffizienten von Zn und Pb.}
    \label{tab:diskussion}
    \begin{tabular}{S S S S}
      \toprule
      {} & {$\mu_{\text{exp}} \mathbin{/} \unit{\per\metre}$} & {$\mu_{\text{com}} \mathbin{/} \unit{\per\metre}$} & {rel. Abweichung\,/\,\%}  \\
      \midrule
      {Zn} & $\num{42.6+-1.4}$ & $\num{69,09}$ & $\num{38,3}$ \\ 
      {Pb} & $\num{92.9+-1.9}$ & $\num{50,61}$ & $\num{83,4}$ \\
      \bottomrule
    \end{tabular}
\end{table}

Die experimentell ermittelten Absorptionskoeffizienten weichen stark von den berechneten Compton Absorptionskoeffizienten
ab. Dies liegt darin begründet, dass nicht nur der Comptoneffekt für die Absorption von $\gamma$-Strahlung verantwortlich
ist, sondern dass in dem Energiebereich von $\ce{^137 Cs}$ auch der Photoeffekt relevant ist.

Für Zink ist der theoretische Wert des Compton Absorptionskoeffizienten größer als der experimentell bestimmte, 
sodass hier davon ausgegangen werden kann, dass der Comptoneffekt hier eine übergeordnete Rolle spielt.
Dabei gilt es jedoch zu beachten, dass der theoretische Wert in dem Experiment nicht erreicht wird. Mögliche 
Gründe hierfür sind Fehler in der Bestimmung der Dicke des Absorbermaterials oder nicht vollständig aus einem 
Element bestehendes Absorbermaterial.

