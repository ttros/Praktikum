\section{Auswertung}
\label{sec:Auswertung}
Die in dieser Auswertung erstellten Plots werden mithilfe der \textit{Python}-Erweiterung 
\textit{matplotlib}~\cite{matplotlib} erstellt. Die Fortplanzung der Messunsicherheiten werden mithilfe von
\textit{uncertainties}~\cite{uncertainties} bestimmt und genügen der Gaußschen Fehlerfortplanzung
\begin{equation*}
    \symup{\Delta} F = \sqrt{\sum_{i}\left(\frac{\symup{d}F}{\symup{d}y_{i}}\symup{\Delta} y_{i} \right)^2}.
\end{equation*}
Die Messunsicherheiten der Zählraten $N$ sind poissonverteilt und werden daher mit $\symup{\Delta}N = \sqrt{N}$ berechnet.


\subsection{\texorpdfstring{Absorptionskurve des $\beta$-Strahlers}{Absorptionskurve des Beta-Strahlers}}
\label{sec:beta}
Die Hintergrundmessung für den $\beta^{-}$-Strahler ergibt $\num{623}$ Impulse in $\qty{900}{\second}$, was 
einer Nullrate von $\qty{0.692+-0.028}{\per\second}$ entpricht.

Die Messwerte zur Bestimmung der Reichweite des $\beta$-Strahlers sind in \autoref{tab:beta} dargestellt.

\begin{table} [H]
    \centering
    \caption{Messwerte zur Bestimmung der Reichweite des $\beta$-Strahlers.}
    \label{tab:beta}
    \begin{tabular}{S S S}
      \toprule
      {$d \mathbin{/} \unit{\micro\metre}$} & {$\symup{\Delta} t \mathbin{/} \unit{\second}$} & {Zählrate} \\
      \midrule
      $\num{100(0  )}$ &  200 & 7842 \\
      $\num{125(0  )}$ &  220 & 2151 \\
      $\num{135(0.5)}$ &  240 & 2205 \\
      $\num{160(1  )}$ &  260 & 1517 \\
      $\num{200(1  )}$ &  280 & 564  \\
      $\num{253(1  )}$ &  300 & 249  \\
      $\num{302(1  )}$ &  320 & 228  \\
      $\num{338(5  )}$ &  340 & 217  \\
      $\num{400(1  )}$ &  360 & 286  \\
      $\num{444(2  )}$ &  380 & 271  \\
      $\num{482(1  )}$ &  400 & 265  \\  
      \bottomrule
    \end{tabular}
\end{table}

In \autoref{fig:beta} wird der Logarithmus der zeitlich normierte Zählrate gegen die Dicke des Absorbermaterials 
aufgetragen. Außerdem wird das zuvor gemessene Hintergrundsignal eingezeichnet.

Mithilfe der \textit{Python}-Erweiterung \textit{scipy} \cite{scipy} wird eine Regressionsgerade der Messwerte gebildet, die sich
vom Hintergrundsignal hervorheben.
Die Dicke, nach der die $\beta$-Strahlung vollständig abgeschirmt ist, ergibt sich als Schnittstelle dieser Geraden mit der
Geraden des Hintergrundsignals.

\begin{figure}[H]
    \centering
    \includegraphics{build/beta.pdf}
    \caption{Absorptionskurve des $\beta$-Strahlers mit eingezeichneter Regressionsgeraden und Hintergrundsignal.}
    \label{fig:beta}
\end{figure}

Die lineare Ausgleichsrechnung liefert für eine Gerade des Typs $y = mx + b$ die Werte
\begin{align*}
    m &= \qty{-0.0234+-0.0026}{\per\micro\metre} & b &= \num{5.5+-0.4}.
\end{align*}

Die Schnittstelle und somit die maximale Reichweite ergibt sich zu
\begin{equation*}
    d_{\text{max}} = \qty{252+-34}{\micro\metre}.
\end{equation*}

Mit der Dichte von Aluminium $\rho_{\text{Al}} = \qty{2.7}{\gram\per\cubic\centi\metre}$ \cite{czichos} lässt sich die Massenbelegung
zu
\begin{equation*}
    R_{\text{max}} = \rho_{\text{Al}} \cdot d_{\text{max}} = \qty{0.0068+-0.0009}{\gram\per\metre\squared}
\end{equation*}
berechnen.
Über den Zusammenhang \eqref{e_max} wird die maximale Energie der $\beta$-Strahlung berechnet. Diese beträgt
\begin{equation*}
    E_{\text{max}} = \qty{0.075+-0.005}{\mega\electronvolt}.
\end{equation*}

\subsection{\texorpdfstring{Bestimmung der Absorptionskoeffizienten verschiedener Stoffe bei $\gamma$-Strahlung}
{Bestimmung der Absorptionskoeffizienten verschiedener Stoffe bei Gamma-Strahlung}}
Genau wie für die Absorptionskurve des $\beta$-Strahlers wird auch in diesem Teil des Versuches zunächst eine Messung
der Hintergrundstrahlung durchgeführt. Hierbei werden 1000 Impulse in $\qty{900}{\second}$ gemessen, woraus sich eine 
Nullrate von $\qty{1.111+-0.035}{\per\second}$ berechnen lässt.

Die in der Durchführung \ref{sec:Durchführung} beschriebene Methode zur Bestimmung der Absorptionskoeffizienten wird
für die Materialien Blei und Zink angewendet. Die Messwerte sind in \autoref{tab:gamma} dargestellt.

\begin{table} [H]
    \centering
    \caption{Messwerte zur Bestimmung der Absorptionskoeffizienten von Zink und Blei unter Strahlung des
    $\gamma$-Strahlers $\ce{^137 Cs}$.}
    \label{tab:gamma}
    \begin{tabular}{S S S | S S S}
      \toprule
      \multicolumn{3}{c}{Zink} & \multicolumn{3}{c}{Blei} \\
      \midrule
      {$d \mathbin{/} \unit{\milli\metre}$} & {$\symup{\Delta} t \mathbin{/} \unit{\second}$} & {Zählrate} &%
      {$d \mathbin{/} \unit{\milli\metre}$} & {$\symup{\Delta} t \mathbin{/} \unit{\second}$} & {Zählrate} \\
      \midrule
      2  & 100 & 10281 & 0     &  30 & 3471  \\
      4  & 110 & 10708 & 1.18  & 100 & 10022 \\
      6  & 120 & 10494 & 2.40  & 105 & 9330  \\
      8  & 130 & 10345 & 5.00  & 120 & 8824  \\
      10 & 140 & 10215 & 10.30 & 135 & 5687  \\
      12 & 150 & 10486 & 13.75 & 155 & 5128  \\
      14 & 160 & 10073 & 15.30 & 165 & 4535  \\
      16 & 170 & 10192 & 20.00 & 180 & 2806  \\
      18 & 180 & 9733  & 30.50 & 190 & 1579  \\
      20 & 190 & 8905  & 40.60 & 200 & 733   \\
      \bottomrule
    \end{tabular}
\end{table}

Analog zu dem Vorgehen in \ref{sec:beta} werden auch hier die logarithmierten und normierten Zählraten in 
Abhängigkeit zu der Dicke des Absorbermaterials in einem Plot aufgetragen und es wird eine lineare Ausgleichsrechnung
durchgeführt. In \autoref{fig:gamma} sind die Plots für die Absorber Zink und Blei zu sehen.

\begin{figure}[H]
    \begin{subfigure}{0.95\textwidth}
      \centering
      \includegraphics[height=8cm]{build/Zn.pdf}
      \caption{Zink.}\vspace{4mm}%
    \end{subfigure}
    \begin{subfigure}{0.95\textwidth}
      \centering
      \includegraphics[height=8cm]{build/Pb.pdf}
      \caption{Blei.}
    \end{subfigure}
    \caption{Darstellung der Messwerte aus \autoref{tab:gamma} für die verschiedenen Absorber mit Ausgleichgeraden.}
    \label{fig:gamma}
  \end{figure}

  Für die linearen Regressionen ergeben sich die Werte
  \begin{align*}
      m_{\text{Zn}} & = \qty{-0.0426+-0.0014}{\per\milli\metre} & b_{\text{Zn}} &= \num{4.719+-0.017} \\
      m_{\text{Pb}} & = \qty{-0.0929+-0.0019}{\per\milli\metre} & b_{\text{Pb}} &= \num{4.70+-0.04}. \\
  \end{align*}

  Aus diesen Werten und unter zuhilfenahme der Gleichungen \eqref{Absorptionsgesetz} und \eqref{mu} lassen sich die
  Absorptionskoeffizienten $\mu$, sowie die Werte für $N_0$ bestimmen. Diese ergeben sich zu
  \begin{align*}
    \mu_{\text{Zn}} & = \qty{42.6+-1.4}{\per\metre} & N_{0,\text{Zn}} &= \num{112+-2} \\
    \mu_{\text{Pb}} & = \qty{92.9+-1.9}{\per\metre} & N_{0,\text{Pb}} &= \num{110+-4}. \\
  \end{align*}

  In einem weiteren Schritt soll die Compton Absorptionskoeffizienten der beiden Absorbermaterialien berechnet werden.
  Dafür werden der Compton Wirkunksquerschnitt $\sigma_{\text{com}} = \qty{2.57e-29}{\metre\squared}$ sowie weitere 
  Konstanten aus \autoref{tab:compton} verwedent. 
  
\begin{table} [H]
    \centering
    \caption{Literaturwerte verschiedener Materialkonstanten von Zink und Blei \cite{Gestis}.}
    \label{tab:compton}
    \begin{tabular}{S S S}
      \toprule
      {Materialkonstante} & {Zink} & {Blei}  \\
      \midrule
      {Z}     & 82 & 30 \\
      $\rho \mathbin{/} \unit{\gram\per\cubic\centi\metre}$ & 11,3 & 7,14 \\
      $M \mathbin{/} \unit{\gram\per\mol}$ & 207,2 & 65,39 \\
      \bottomrule
    \end{tabular}
\end{table}

Mit den Formeln \eqref{mu} und \eqref{n} ergeben sich die Compton Absorptionskoeffizienten zu
\begin{align*}
    \mu_{\text{com, Zn}} &= \qty{69,09}{\per\metre} & \mu_{\text{com, Pb}} &= \qty{50,61}{\per\metre}.
\end{align*}