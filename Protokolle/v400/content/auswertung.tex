\section{Auswertung}
\label{sec:Auswertung}

\subsection{Reflexionsgesetz}

Im folgenden werden in \autoref{tab:reflex} die gemessenen Winkel zur Überprüfung des Reflexionsgesetzes festgehalten.

\begin{table} [H]
    \centering
    \caption{Wertepaare der Einfalls- und Ausfallswinkel zur Überprüfung des Reflexionsgesetzes.}
    \label{tab:reflex}
    \begin{tabular}{S S S}
      \toprule
      {Einfallswinkel $\alpha_1$ / $\unit{\degree}$} & {Ausfallswinkel $\alpha_2$ / $\unit{\degree}$} & {Differenz $\symup{\Delta}\alpha$ / $\unit{\degree}$}\\
      \midrule
      20 & 20   & 0\\
      25 & 25   & 0\\
      30 & 30   & 0\\
      40 & 40   & 0\\
      45 & 45.5 & 0.5\\
      60 & 59.5 & 0.5\\
      70 & 69.5 & 0.5\\
      \bottomrule
    \end{tabular}
\end{table}

Der Ausfallswinkel entpricht hier bis auf geringe Fehler bei größeren $\alpha_1$ genau dem Ausfallswinkel, somit
kann das Reflexionsgesetz verifiziert werden.
Die Messungenauigkeit lag bei ca. $\qty{0.25}{\degree}$ , da der Laserstrahl eine gewisse Breite aufweist, die ein genaueres
Ablesen verhindert.

\subsection{Brechungsgesetz}

Die Messergebnisse für diesen Versuchsteil sind in \autoref{tab:brechung} zu finden.

\begin{table} [H]
    \centering
    \caption{Einfallwinkel $\alpha$ und zugehörige Brechungswinkel $\beta$.}
    \label{tab:brechung}
    \begin{tabular}{S S}
      \toprule
      {Einfallswinkel $\alpha$ / $\unit{\degree}$} & {Brechungswinkel $\beta$ / $\unit{\degree}$} \\
      \midrule
      10 & 6.5  \\
      20 & 13   \\
      30 & 19.5 \\
      40 & 25.5 \\
      50 & 31.0 \\
      60 & 35.5 \\
      70 & 39.0 \\
      \bottomrule
    \end{tabular}
\end{table}

Mithilfe der Formel \eqref{eq:Brechung} wird für alle Messwertepaare der Brechungsindex $\eta$ berechnet,
es ergibt sich als Mittelwert der sieben Messungen:
\begin{gather*}
    \eta_{\text{exp}}=\qty{1.502+-0.016}{}.
\end{gather*}
Der theoretische Wert für Plexiglas wurde im Rahmen der Vorbereitung bestimmt und liegt bei
\begin{gather*}
    \eta_{\text{theo}}=\qty{1.49}{}.
\end{gather*}
Unter Berücksichtigung der Unsicherheit des Mittelwertes lässt sich das Messergebnis also mit der Theorie vereinen.

Die Lichtgeschwindigkeit $v$ in Plexiglas lässt sich somit nun folgendermaßen berechnen:
\begin{equation*}
    v=\frac{c_{\text{Vakuum}}}{\eta}=(1.995\pm0.022)10^6\unit{\kilo\meter\per\second}
\end{equation*}