\section{Diskussion}
\label{sec:Diskussion}

In den ersten beiden Versuchsteilen konnten alle theoretisch erwarteten Werte bestätigt werden.
Die Messergebnisse besitzen also nur geringe Unsicherheiten, da größere
Fehlerquellen in dem Versuch außzuschließen sind.

Bei der Berechnung des Strahlenversatzes konnte zwischen den beiden Methoden kein wesentlicher Unterschied
festgestellt werden, auch hier werden die geringen Unsicherheiten bestätigt.

Bei der Brechung im Prisma konnte erfolgreich gezeigt werden, dass die Ablenkung tatsächlich
Abhängig von der Wellenlänge ist, was theoretisch auch zu erwarten ist.

Im letzten Versuchsteil ließen sich ebenfalls realistische Werte für die Wellenlängen berechnen.
In den ersten beiden Zeilen in \autoref{tab:100mm} fällt jedoch auf, dass sich die
selben Wellenlängen für rot und grün ergeben.
Dies ist höchstwahrscheinlich auf einen Ablesefehler der Intensitätsmaxima zurückzuführen, der im Nachhinein
jedoch nur schwierig nachzuvollziehen ist.
Insgesamt ergeben sich im Mittel trotzdem realistische Werte, sodass dieser Fehler keine allzu große Auswirkung
zu haben scheint.
