\section{Diskussion}
\label{sec:Diskussion}
Die Zeitkonstante $RC$ ergibt sich mit den unterschiedlichen Messverfahren zu den Werten
\begin{gather}
    RC_{1} = (39,92\pm0,28)\, \symup{\mu s} \notag \\
    RC_{2} = (68,750206\pm0,000009)\,\symup{\mu s} \notag \\
    RC_{3} = (69,01295\pm0,00001)\,\symup{\mu s} \notag .
\end{gather}
Die ermittelte Größe $RC_{1}$ weicht deutlich von den Größen
$RC_{2}$ und $RC_{3}$ ab. Dabei ist zu beachten, für die Ermittlung der Größen $RC_{2}$ und $RC_{3}$
der Versuchsaufbau nicht verändert wurde und die Werte zur Bestimmung dieser beiden Größen zusammen gemessen wurden,
entsprechend ergeben sich ähnliche Werte. Ein möglicher Grund für die große Abweichung von rund $42\%$ des ersten Wertes 
auf die anderen Werte ist ein fehlerhafter
Frequenzgenerator, der keine Konstante Amplitude erzeugt hat. Alternativ ist es auch denkbar, dass Einstellungen am Oszilloskop
fehlerhaft vorgenommen wurden.
Über die Größenordnung der Zeitkonstante $RC$ lässt sich dennoch eine Aussage treffen, da alle drei Messungen in derselben
Größenordnung von $10\,\unit{\micro\second}$ liegen.

Qualitativ konnte die Theorie zur Funktionsweise es Tiefpass verifiziert werden, da die Plots
in \autoref{fig:plot_b} und \autoref{fig:plot_c} den Verlauf von Spannung und Phasenversatz wie in der Theorie 
beschrieben darstellen.
Auch die Oszilloskopbilder aus \autoref{fig:aufgabe d - rechteckspannung}, \autoref{fig:aufgabe d - dreieckspannung}
und \autoref{fig:aufgabe d - sinusspannung} bestätigen qualitativ die Funktionsweise eines Tiefpass als Integrationsglied.