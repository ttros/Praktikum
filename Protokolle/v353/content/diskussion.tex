\section{Diskussion}
\label{sec:Diskussion}
Die Zeitkonstante $RC$ ergibt sich mit den unterschiedlichen Messverfahren zu den Werten
\begin{gather}
    RC_{1} = (39,92\pm0,28) \symup{\mu s} \notag \\
    RC_{2} = (68,7502063\pm0,0000092)\symup{\mu s} \notag \\
    RC_{3} = (69,0129590\pm0,0000199)\symup{\mu s} \notag .
\end{gather}
Wie unschwer zu erkennen ist, weicht die ermittelte Größe $RC_{1}$ deutlich von den Größen
$RC_{2}$ und $RC_{3}$ ab. Dabei ist zu beachten, für die Ermittlung der Größen $RC_{2}$ und $RC_{3}$
der Versuchsaufbau nicht verändert wurde und die Werte zur Bestimmung dieser beiden Größen zusammen gemessen wurden,
entsprechend ergeben sich ähnliche Werte.

Eine abschließende Aussage über die Zeitkonstante des untersuchten RC-Kreises ist somit nicht möglich,
da das Messergebniss von $RC_{1}$ zu stark von der anderen Messreihe abweicht. Ein möglicher Grund hierfür ist
mangegelnde Genauigkeit des Frequenzgenerators und dessen nicht berücksichtigter Innenwiderstand.

Qualitativ konnte jedoch die Theorie zur Funktionsweise es Tiefpass verifiziert werden, da die Plots
in \autoref{fig:plot_b} und \autoref{fig:plot_c} den Verlauf von Spannung und Phasenversatz wie in der Theorie 
beschrieben darstellen.
Auch die Oszilloskopbilder aus \autoref{fig:aufgabe d - rechteckspannung}, \autoref{fig:aufgabe d - dreieckspannung}
und \autoref{fig:aufgabe d - sinusspannung} bestätigen qualitativ die Funktionsweise eines Tiefpass als Integrationsglied.