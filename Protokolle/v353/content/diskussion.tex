\section{Diskussion}
\label{sec:Diskussion}
Die Zeitkonstante $RC$ ergibt sich mit den unterschiedlichen Messverfahren zu den Werten
\begin{gather}
    RC_{1} = 39.92\pm0.28\symup{\mu s} \notag \\
    RC_{2} = 431.97025\pm0.00037\symup{\mu s} \notag \\
    RC_{3} = 433.62156\pm0.00079\symup{\mu s} \notag .
\end{gather}
Wie unschwer zu erkennen ist, weicht die ermittelte Größe $RC_{1}$ deutlich von den Größen
$RC_{2}$ und $RC_{3}$ ab. Dabei ist zu beachten, für die Ermittlung der Größen $RC_{2}$ und $RC_{3}$
der Versuchsaufbau nicht verändert wurde und die Werte zur Bestimmung dieser beiden Größen zusammen gemessen wurden.
---Ergänzen: Fehler... 
