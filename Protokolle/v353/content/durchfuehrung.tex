\section{Durchführung}
\label{sec:Durchführung}

\subsection{Messung der Zeitkonstanten $RC$ über die Entladekurve}

Für die erste Aufgabe wird die Schaltung in \autoref{fig:Schaltbild A} verwendet.
Dabei wird wie in der Theorie bereits erläutert eine Rechteckspannung angelegt.

\begin{figure}
    \centering
    \includegraphics{content/Durchführung - a.pdf}
    \caption{Schaltbild für Messung der Zeitkonstanten $RC$ über die Entladekurve. \cite{v353}}
    \label{fig:Schaltbild A}
\end{figure}


Es wird der Verlauf der Kondensatorspannung $U_{\symup{C}}(t)$ in Abhängigkeit der Zeit $t$ beobachtet.
Die Frequenz der angelegten Rechteckspannung wird hierbei so eingestellt,
dass sich eine Änderung um den Faktor 5 bis 10 von $U_{\symup{C}}(t)$ auf dem Oszilloskop ablesen lässt.
Anschließend wird das Bild des Oszilloskops fotografiert und in der Auswertung analysiert.

\subsection{Messung der Frequenabhängigkeit der Amplitude und der Phasenverschiebung}

Nun wird das Schaltbild entsprechend  \autoref{fig:Schaltbild C+D} verändert und eine Wechselspannung angelegt.

\begin{figure} [H]
    \centering
    \includegraphics[height=6cm]{content/Durchführung - c+d.pdf}
    \caption{Schaltbild für Messung der Frequenabhängigkeit der Amplitude und der Phasenverschiebung. \cite{v353}}
    \label{fig:Schaltbild C+D}
\end{figure}

Dabei ist die Frequenz variabel einstellbar, wobei die Amplitude der Wechselspannung identisch bleibt.
Es wird nun die Amplitude $A$ der Kondensatorspannung in Abhängigkeit der Frequenz $f$ gemessen,
dabei wird $f$ von $50\,\unit{\hertz}$ bis $100\,\unit{\kilo\hertz}$, also über mehrere Zehnerpotenzen hinweg, variert.

Außerdem werden die Periodendauer $b$ und der Abstand $a$ der Nulldurchgänge gemessen, aus denen in der Auswertung
der Phasenversatz zwischen $U_{\symup{C}}$ und $U_{0}$ ermittelt wird. In \autoref{fig:Phase messen} ist die 
geometrische Bedeutung von $b$ und $a$ skizziert.
Die Messwerte werden in einer Tabelle erfasst und anschließend ausgewertet.
\begin{figure}% [H]
    \centering
    \includegraphics{content/Durchführung - Phase_messen.pdf}
    \caption{Skizze zur Ermittlung des Phasenversatzes. \cite{v353}}
    \label{fig:Phase messen}
\end{figure}

\subsection{Nutzung des $RC$-Kreises zur Integration von Spannungen hoher Frequenz}

Zuletzt wird die Wirkung des $RC$-Kreises als Integrationsglied überprüft, dafür kann die Schaltung aus
Aufgabe C beibehalten werden. Es wird wie in der Theorie erklärt eine Wechselspannung mit
geigneter Frequenz $\omega >> 1/RC$ eingestellt.
Dabei werden 3 verschiedene Formen der Wechselspannung gewählt: Eine Rechteck-, Sinus und Dreieickspannung.
Auf dem 2-Kanal Oszilloskop ist jeweils die Form der Kondensatorspannung $U_{\symup{C}}(t)$ und die Form der gewählten
Wechselspannung $U_{\symup{G}}(t)$ zu sehen.
Die 3 Bilder werden fotografiert und in der Auswertung analysiert.