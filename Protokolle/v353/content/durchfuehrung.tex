\section{Durchführung}
\label{sec:Durchführung}

\subsection{Aufgabe A}

Für die erste Aufgabe wird die Schaltung in \autoref{fig:Schaltbild A} verwendet.
Dabei wird wie in der Theorie bereits erläutert eine Rechteckspannung angelegt.

\begin{figure}
    \centering
    \includegraphics{content/Durchführung - a.pdf}
    \caption{Schaltbild für Messung A. \cite{v353}}
    \label{fig:Schaltbild A}
\end{figure}


Es wird der Verlauf der Kondensatorspannung $U_{C}(t)$ in Abhängigkeit der Zeit $t$ beobachtet.
Die Frequenz der angelegten Rechteckspannung und die Ablenkgeschwindigkeit des Kathodenstrahls werden hierbei so eingestellt,
dass sich eine Änderung um den Faktor 5 bis 10 von $U_{C}(t)$ auf dem Oszilloskop ablesen lässt.
Anschließend wird das Bild des Oszilloskops fotografiert und in der Auswertung analysiert.

\subsection{Aufgabe B}

Nun wird das Schaltbild entsprechend  \autoref{fig:Schaltbild B} verändert und eine Wechselspannung angelegt.

\begin{figure}% [H]
    \centering
    \includegraphics{content/Durchführung - b.pdf}
    \caption{Schaltbild für Messung B. \cite{v353}}
    \label{fig:Schaltbild B}
\end{figure}

Dabei ist die Frequenz variabel einstellbar, wobei die Amplitude der Wechselspannung identisch bleibt.
Es wird nun die Amplitude $A$ der Kondensatorspannung in Abhängigkeit der Frequenz $f$ gemessen,
dabei wird $f$ von 50Hz bis 100.000Hz, also über mehrere Zehnerpotenzen hinweg, variert.
Die Messwerte werden in einer Tabelle erfasst und anschließend ausgewertet.

\subsection{Aufgabe C}

Für diese Aufgabe wird die Schaltung aus Aufgabe B leicht verändert, sodass der Verlauf der Wechselspannung  $U_{G}(t)$
und der Kondensatorspannung $U_{C}(t)$ mit einem 2-Kanal Oszilloskop betrachtet werden können.
Das neue Schaltbild ist in \autoref{fig:Schaltbild C+D} zu sehen.

\begin{figure}% [H]
    \centering
    \includegraphics[height=6cm]{content/Durchführung - c+d.pdf}
    \caption{Schaltbild für Messung C und D. \cite{v353}}
    \label{fig:Schaltbild C+D}
\end{figure}

Gemessen werden nun die Periodendauer $b$ und der Abstand $a$ der Nulldurchgänge, aus denen in der Auswertung
der Phasenversatz ermittelt wird. In \autoref{fig:Phase messen} ist die geometrische Bedeutung von $b$ und $a$ skizziert.
Die Ergebnisse werden wie in Aufgabe B tabellarisch erfasst und ausgewertet.

\begin{figure}% [H]
    \centering
    \includegraphics{content/Durchführung - Phase_messen.pdf}
    \caption{Skizze zur Ermittlung des Phasenversatzes. \cite{v353}}
    \label{fig:Phase messen}
\end{figure}

\subsection{Aufgabe D}

Zuletzt wird die Wirkung des $RC$-Kreises als Integrationsglied überprüft, dafür kann die Schaltung aus
Aufgabe C beibehalten werden. Es wird wie in der Theorie erklärt eine Wechselspannung mit
geigneter Frequenz $\omega >> 1/RC$ eingestellt.
Dabei werden 3 verschiedene Formen der Wechselspannung gewählt: Eine Rechteck-, Sinus und Dreieickspannung.
Auf dem 2-Kanal Oszilloskop ist jeweils die Form der Kondensatorspannung $U_{C}(t)$ und die Form der gewählten
Wechselspannung $U_{G}(t)$ zu sehen.
Die 3 Bilder werden fotografiert und in der Auswertung analysiert.