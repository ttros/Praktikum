\section{Durchführung}
\label{sec:Durchfuehrung}
Für den gesamten Versuch wird ein Lock-In-Verstärker mit eingbautem Vorverstärker, Tief- und Bandpass, Phasenverschieber,
sowie Funktions- und Rauschgenerator verwendet. Außerdem wird ein Oszilloskop verwendet, um die verschiedenen Signale darzustellen.

\begin{figure} [H]
    \centering
    \includegraphics[height=8cm]{content/Bilder/Aufbau_Bild.png}
    \caption{Lock-In-Verstärker ohne Verkabelung.\cite{v303}}
    \label{fig:Aufbau Lock-In-Verstaerker}
\end{figure}

% Ohne Rauschen
\subsection{Messung ohne Rauschen}
\label{sec:Messung ohne Rauschen}
Das in \autoref{fig:Aufbau Lock-In-Verstaerker} zu sehende Gerät wird wie in \autoref{fig:Schematischer Aufbau} verkabelt, 
wobei in einer ersten Messreihe der Rauschgenerator überbrückt wird. Der Funktionsgenerator wird auf eine Sinusspannung mit 
$U_{\symup{sig}}\approx\qty{2}{\volt}$ und eine Frequenz von rund $\qty{1}{\kilo\hertz}$ eingestellt. Dieses Signal wird 
verstärkt und durch den Bandpassfilter geleitet. Das Referenzsignal $U_{\symup{ref}}$ wird ebenfalls im Funktionsgenerator 
erzeugt und hat dieselbe Frequenz wie $U_{\symup{sig}}$, wir allerdings im Phasenverschieber um $\symup{\Delta}\phi$ 
phasenverschoben. Dieses Signal $U_{\symup{ref}}$ wird im Detektor mit dem Signal $U_{\symup{sig}}$ multipliziert und in einem
Tiefpass nochmals gefiltert. Das Ausgangssignal $U_{\symup{out}}$ wird auf einem Oszilloskop angezeigt.

Im Verlauf der Messung wird die Verschiebung der Phase zwischen $U_{\symup{sig}}$ und $U_{\symup{ref}}$ mit dem Phasenverschieber
verändert und es wird mithilfe des Oszilloskop die Spannung des Ausgangssignal $U_{\symup{out}}$ gemessen und tabellarisch
protokolliert, sowie graphisch dargestellt. Für die einzelnen Messpunkte wird ein nicht-linearer Fit bestimmt.

\begin{figure} [H]
    \centering
    \includegraphics[height=5cm]{content/Bilder/Aufbau_Schema.png}
    \caption{Schematischer Aufbau des Lock-In-Verstärkers.\cite{v303}}
    \label{fig:Schematischer Aufbau}
\end{figure}

\subsection{Messung mit Rauschen}
\label{sec:Messung mit Rauschen}
Die Messung aus \ref{sec:Messung ohne Rauschen} wird nun mit einem eingeschalteten Rauschgenerator wiederholt. Die Amplitude
des Rauschens liegt hierbei in derselben Größenordnung wie die des Signals $U_{\symup{sig}}$.
Auch hier erfolgt eine Tabellierung und graphische Darstellung der Messwerte mit anschließendem nicht-linearen Fit.

\subsection{Messung mit einer LED}
\label{sec:Messung mit einer LED}
Es wird eine Photodetektorschaltung gemäß \autoref{fig:Photodetektorschaltung} aufgebaut. Der Unterschied zu der vorherigen 
Schaltung (\autoref{fig:Schematischer Aufbau}) liegt darin, dass nun nicht mehr ein Rauschgenerator verwendet wird, sondern 
das Ausgabesignals eines Photodetektors, welcher Lichtwellen einer blinkenden LED registriert, analysiert wird. Die LED wird also
an den 

\begin{figure} [H]
    \centering
    \includegraphics[height=5cm]{content/Bilder/Aufbau_led.png}
    \caption{Schematischer Aufbau der Photodetektorschaltung.\cite{v303}}
    \label{fig:Schematischer Aufbau LED}
\end{figure}

Hierbei wird nun nicht mehr die Phase zwischen $U_{\symup{sig}}$ und $U_{\symup{ref}}$ verändert, sondern die blinkende LED 
wird schrittweise weiter von dem Photodetektor entfernt und es wird die Signalstärke $U_{\symup{out}}$ abhängig von dem 
Abstand $x$ zwischen LED und Photodetektor mithilfe des Oszilloskops bestimmt. Auch diese Daten werden erneut tabelliert und
graphisch dargestellt. 

Linearisieren?!?!?!?!?!?!? Wenn ja, in Theorie formel dazu und sagen, was ziel.