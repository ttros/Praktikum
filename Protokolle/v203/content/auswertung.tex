\section{Auswertung}
\label{sec:Auswertung}

\subsection{Verdampfungswärme von Wasser für $p\,≤\,\qty{100}{\kilo\pascal}$}
Der Umgebungsdruck $p_0$ wird vor Beginn der Messreihe zu $p_0 = \qty{101,7}{\kilo\pascal}$ bestimmt. Die Messwerte für die Gastemperatur $T$
und des Drucks $p$ werden tabellarisch erfasst und anschließend graphisch in \autoref{fig:Verdampfungswärme} dargestellt. 

\begin{longtable}{S[table-format=3.0] S[table-format=3.0] | S[table-format=3.0] S[table-format=2.1]}
  \caption{Messwertepaare Temperatur $T$ und Druck $p$ mit $p\,≤\,\qty{100}{\kilo\pascal}$.}\label{tab:Teil 1} \\
    \toprule
    {$T\,/\,\unit{\celsius}$} & {$p\,/\,\unit{\milli\bar}$} & {$T\,/\,\unit{\kelvin}$} & {$p\,/\,\unit{\kilo\pascal}$} \\
    \midrule
    \endfirsthead
    \caption[]{Messwertepaare Temperatur $T$ und Druck $p$ mit $p\,≤\,\qty{100}{\kilo\pascal}$. (Fortsetzung)}\\
    \hline
    \endhead
    \hline
    \endfoot
    26  & 107 & 299 & 10.7 \\
    28  & 110 & 301 & 11.0 \\
    30  & 116 & 303 & 11.6 \\
    32  & 120 & 305 & 12.0 \\
    34  & 125 & 307 & 12.5 \\
    36  & 129 & 309 & 12.9 \\
    38  & 134 & 311 & 13.4 \\
    40  & 139 & 313 & 13.9 \\
    42  & 143 & 315 & 14.3 \\
    44  & 148 & 317 & 14.8 \\
    46  & 153 & 319 & 15.3 \\
    48  & 159 & 321 & 15.9 \\
    50  & 165 & 323 & 16.5 \\
    52  & 171 & 325 & 17.1 \\
    54  & 177 & 327 & 17.7 \\
    56  & 185 & 329 & 18.5 \\
    58  & 200 & 331 & 20.0 \\
    60  & 218 & 333 & 21.8 \\
    62  & 238 & 335 & 23.8 \\
    64  & 257 & 337 & 25.7 \\
    66  & 277 & 339 & 27.7 \\
    68  & 302 & 341 & 30.2 \\
    70  & 327 & 343 & 32.7 \\
    72  & 353 & 345 & 35.3 \\
    74  & 382 & 347 & 38.2 \\
    76  & 414 & 349 & 41.4 \\
    78  & 448 & 351 & 44.8 \\
    80  & 482 & 353 & 48.2 \\
    82  & 525 & 355 & 52.5 \\
    84  & 562 & 357 & 56.2 \\
    86  & 601 & 359 & 60.1 \\
    88  & 637 & 361 & 63.7 \\
    90  & 676 & 363 & 67.6 \\
    92  & 696 & 365 & 69.6 \\
    94  & 726 & 367 & 72.6 \\
    96  & 756 & 369 & 75.6 \\
    98  & 776 & 371 & 77.6 \\
    100 & 785 & 373 & 78.5 \\
    102 & 789 & 375 & 78.9 \\
    104 & 794 & 377 & 79.4 \\
    106 & 803 & 379 & 80.3 \\
    108 & 820 & 381 & 82.0 \\
    110 & 836 & 383 & 83.6 \\
    112 & 850 & 385 & 85.0 \\
    114 & 864 & 387 & 86.4 \\ 
\end{longtable}

\begin{figure}
  \centering
  \includegraphics{plot_1.pdf}
  \caption{Graphische Darstellung der Messwertpaare aus \autoref{tab:Teil 1} mit Ausgleichsgerade.}
  \label{fig:Verdampfungswärme}
\end{figure}

Mithilfe der 
\textit{Python}-Erweiterung \textit{scipy}\cite{scipy} wird eine lineare Ausgleichsrechnung durchgeführt und es ergibt sich eine
Ausgleichsgerade vom Typ
\begin{equation*}
  \ln\left(\frac{p}{p_0}\right)= \qty{3220+-80}{\kelvin} \, \frac{1}{T} + (\num{1+-8}).
\end{equation*}
Dies enstpricht der Form von \eqref{eq:Druck} und somit ergibt sich die Verdampfungswärme $L$ zu
\begin{align*}
  \frac{L}{R} &= \qty{3220+-80}{\kelvin} \\
  \Leftrightarrow L &= \qty{26,8+-0,7}{\kilo\joule\per\mol}.
\end{align*}
Für die Gaskonstante $R$ wird hier der Wert $R = \qty{8,314}{\joule\per\mol\per\kelvin}$ \cite{czichos} verwendet.

Um die innere Verdampfungswärme $L_{\symup{i}}$ zu bestimmen wird der Zusammenhang~ REFERENZZZ!!!!!!\eqref{eq:}, sowie die allgemeine 
Gasgleichung~ REFERENZZZ!!!!!!!\eqref{eq:} verwendet. Zunächst wird die äußere Verdampfungswärme $L_{\symup{a}}$ für eine 
Temperatur~$T = \qty{373}{\kelvin}$ abgeschätzt:
\begin{align*}
  L_{\symup{a}} &= pV = RT \\
                &= \qty{3,101}{\kilo\joule\per\mol}
\end{align*}
Folglich ergibt sich für die innere Verdampfungswärme $L_{\symup{i}}$ der Wert
\begin{align*}
  L_{\symup{i}} &= L - L_{\symup{a}} \\
                &= \qty{26,8+-0,7}{\kilo\joule\per\mol} - \qty{3,101}{\kilo\joule\per\mol} \\
                &= \qty{23,7+-0,7}{\kilo\joule\per\mol}.
\end{align*}
Bezieht sich die Verdampfungswärme nun nicht mehr auf eine mol-Masse, sondern auf einzelne Moleküle, muss $L_{\symup{i}}$ durch die
Avogadro-Konstante $N_{\symup{A}}=\qty{6,02e23}{\per\mol}$ \cite{czichos} geteilt werden. Zur Übersichtlichkeit wird $L_{\symup{i, M}}$ in eV angegeben:
\begin{align*}
  L_{\symup{i, M}} = \qty{0.278+-0.007}{\electronvolt}
\end{align*}

%%%%%%%%%%% Kapitel 2 %%%%%%%%%%%
\subsection{Temperaturabhängigkeit der Verdampfungswärme für $\qty{100}{\kilo\pascal}\,≤\,p\,≤\,\qty{1500}{\kilo\pascal}$}
Die Messwerte für den Druck $p$ und die Temperatur $T$ werden auch hier zunächst tabellarisch erfasst und anschließend graphisch dargestellt.
\begin{table}
  \centering
  \caption{Messwertepaare Temperatur $T$ und Druck $p$ mit $\qty{100}{\kilo\pascal}\,≤\,p\,≤\,\qty{1500}{\kilo\pascal}$}
  \label{tab:Teil 2}
  \begin{tabular}{S[table-format=2.0] S[table-format=3.0] | S[table-format=4.0] S[table-format=3.0]}
      \toprule
       {$p\,/\,\unit{\milli\bar}$} & {$T\,/\,\unit{\celsius}$} & {$p\,/\,\unit{\kilo\pascal}$} & {$T\,/\,\unit{\kelvin}$} \\
      \midrule
          1	  & 117 & 100  & 390 \\
          2	  & 134 & 200  & 407 \\
          3	  & 141 & 300  & 414 \\
          4	  & 148 & 400  & 421 \\
          5	  & 154 & 500  & 427 \\
          6	  & 161 & 600  & 434 \\
          7	  & 167 & 700  & 440 \\
          8	  & 172 & 800  & 445 \\
          9	  & 176 & 900  & 449 \\
          10	& 180 & 1000 & 453 \\
          11	& 184 & 1100 & 457 \\
          12	& 187 & 1200 & 460 \\
          13	& 191 & 1300 & 464 \\
          14	& 194 & 1400 & 467 \\
          15	& 197 & 1500 & 470 \\  
          \bottomrule 
  \end{tabular}
\end{table}
Auch hier wird eine Ausgleichsrechnung mit \textit{scipy} durchgeführt und es wird ein Fit eingezeichnet. Die Ausgleichsfunktion wird hierbei durch ein 
allgemeines Polynom 3.~Grades bestimmt.
Das Polynom lässt sich bestimmen zu
\begin{align*}
  p(T) &= a \cdot T^3 + b\cdot T^2 + c \cdot T + d
\end{align*}
mit den Parametern
\begin{align*}
  a&=\qty{0.00065+-0.00029}{\kilo\pascal\per\cubic\kelvin}
\end{align*}

A: 0.00065+/-0.00029
B: -0.7+/-0.4
C: (2.4+/-1.6)e+02
D: (-2.8+/-2.3)e+04
