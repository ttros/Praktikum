\section{Diskussion}
\label{sec:Diskussion}

Im Folgenden wird die in \ref{sec:Auswertung 1} berechnete Verdampfungswärme mit einem Literaturwert~\cite{czichos} verglichen.
\begin{align}
    L_{\symup{exp}}&=\qty{26,8+-0,7}{\kilo\joule\per\mol} \notag \\ 
    L_{\symup{lit}}&=\qty{40,65}{\kilo\joule\per\mol} \notag
\end{align}
Die Abweichung vom Literaturwert liegt somit bei ca. $34\%$ und ist damit so groß, dass die Messung den
Literaturwert nicht bestätigen kann.
Aufgrund von einer Undichtigkeit der Apperatur in \autoref{fig:Versuchsaufbau 1bar} an einem Ventil
der Woulffschen Flasche wurde die Messung des Dampfdruckes verfälscht.

Dieser Fehler spiegelt sich auch in \autoref{fig:Verdampfungswärme} wider, die Messwerte weichen deutlich
von der erwarteten exponentiellen Natur ab und streuen daher stark um die Ausgleichsgerade.

Für die Zeitabhängigkeit von $L$ haben sich zwei mögliche mathematische Funktionen ergeben.
Dabei stellt jedoch nur die Funktion $L_{+}$ in \autoref{fig:L_plus} eine physikalisch sinnvolle
Modellierung dar, da die Verdampfungswärme mit steigender Temperatur logischerweise abnehmen muss.
