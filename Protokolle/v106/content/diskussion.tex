\section{Diskussion}
\label{sec:Diskussion}
\subsection{Kurzes Pendel}
Die Messwerte der Periodendauern der einzelnen Pendel $T_1$ und $T_2$, sowie der Messwert der gleichsinnigen Schwingung, ergeben sich zu 
\begin{align*}
    \bar{T}_{1,\symup{exp}} &= \qty{1.446+-0.003}{\second} \\
    \bar{T}_{2,\symup{exp}} &= \qty{1.448+-0.006}{\second} \\
    \bar{T}_{+,\symup{exp}} &= \qty{1.454+-0.009}{\second}
    \end{align*}
und liegen damit innerhalb der Messunsicherheiten voneinander. Daher kann an dieser Stelle davon ausgegangen werden,
dass die Pendel eine nahezu identische Länge haben.
Werden diese Werte allerdings mit dem Theoriewert für ein Pendel der Länge $l=\qty{50}{\centi\metre}$ verglichen, fällt auf, dass für alle drei
Messwerte der Theoriewert
\begin{equation*}
    \bar{T}_{\symup{theorie}} = \qty{1.418}{\second}
\end{equation*}
nicht im Bereich der Messunsicherheiten liegt. Dennoch ist die relative Abweichung von rund $\qty{2}{\percent}$ so gering, dass eventuelle Messfehler der
Länge der Pendel dies erklären könnten.

Die Werte aus Theorie und Experiement für die gegenphasige Schwingung lauten
\begin{align*}
    \bar{T}_{-,\symup{exp}} &= \qty{1.418+-0.008}{\second} \\
    \bar{T}_{-,\symup{theorie}} &= \qty{1.414+-0.001}{\second}
\end{align*}
und sind somit auch innerhalb der Messunsicherheiten kongruent.

Für die Schwebungsdauer werden die Werte
\begin{align*}
    \bar{T}_{\symup{Schwebung},\symup{exp}} &= \qty{36.07+-0.15}{\second} \\
    \bar{T}_{\symup{Schwebung},\symup{theorie}} &= \qty{57+-19}{\second}
\end{align*}
ermittelt. Die große Messunsicherheit des Theoriewertes liegt darin begründet, dass hier zur Berechnung fehlerbehaftete, experiementelle
Größen verwendet werden müssen. Auch die große relative Abweichung von etwa $\qty{37}{\percent}$ lässt sich hiermit begründen.

\subsection{Langes Pendel}
Für die Messwerte der einzeln schwingenden Pendel und der gleichphasigen Schwingung ergibt sich auch hier eine Übereinstimmung. 
\begin{align*}
    \bar{T}_{1,\symup{exp}} &= \qty{1.973+-0.005}{\second} \\
    \bar{T}_{2,\symup{exp}} &= \qty{1.974+-0.006}{\second} \\
    \bar{T}_{+,\symup{exp}} &= \qty{1.976+-0.004}{\second} \\
    \bar{T}_{+,\symup{theorie}} &= \qty{2.006}{\second} 
\end{align*}
Ebenso wie für die kurzen Pendel ergibt sich eine relative Abweichung zum Theoriewert von jeweils etwa $\qty{2}{\percent}$.

Der relative Fehler für die gegenphasige Schwingung liegt bei rund $\qty{12,5}{\percent}$ und ist damit vergleichsweise groß.
\begin{align*}
    \bar{T}_{-,\symup{exp}} &= \qty{1.732+-0.007}{\second} \\
    \bar{T}_{-,\symup{theorie}} &= \qty{1.9799+-0.0008}{\second}
\end{align*}
Die Abweichung lässt sich damit erklären, dass hier die experimentellen Fehler der Kopplungskonstanten $\kappa$ erneut starken Einfluss auf 
den Theoriewert haben.

Für die gekoppelte Schwingung zeigt es sich, dass die experiementelle Schwebungsdauer mit der theoretischen vereinbar ist.
\begin{align*}
    \bar{T}_{\symup{Schwebung},\symup{exp}} &= \qty{14.43+-0.13}{\second} \\
    \bar{T}_{\symup{Schwebung},\symup{theorie}} &= \qty{14.1+-0.5}{\second}
\end{align*}

