\section{Zielsetzung}
\label{sec:Zielsetzung}
In diesem Versuch werden zwei gekoppelte Pendel untersucht. Ziel ist es, die Schwingungs- und Schwebungsdauern~$T$ und~$T_{\symup{S}}$, sowie die
Kopplungskonstante $\kappa$ zu bestimmen. Dafür werden gleichsinnige, gegensinnige und gekoppelte Schwingungen betrachtet.

\section{Theorie}
\label{sec:Theorie}
Wird ein Fadelpendel der Länge $l$ in dem Gravitationsfeld der Erde um einen kleinen Winkel $\varphi$ ausgelenkt, so ist die rücktreibende Kraft
proportional zur Auslenkung $x$ aus der Ruhelage. Hierbei handelt es sich somit um eine harmonische Schwingung die mit der Diffenentialgleichung für
den harmonischen Oszillator beschrieben werden kann. Die Kreisfrequenz $\omega$ und die Schwingungsdauer $T$ lauten dann
\begin{align}
    \omega &= \sqrt{\frac{g}{l}} \label{eq:Kreisfrequenz} \\
    T &= 2\symup{\pi}\sqrt{\frac{l}{g}}. \label{eq:Schwingungsdauer}
\end{align}
