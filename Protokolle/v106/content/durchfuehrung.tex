\section{Durchführung}
\label{sec:Durchführung}
Für den gesamten Versuch wird ein Aufbau verwendet, bei dem zwei parallel zueinander hängende Pendel verwendet werden. Diese Pendel können über
eine einhängbare Feder gekoppelt werden. Außerdem ist die Länge der Pendel einstellbar.
Die Durchführung lässt sich grob in vier Teile gliedern, in denen jeweils immer die \textit{5-fache Periodendauer} einer bestimmten Pendelkonstellation
gemessen.

Der vollständige Ablauf wird für \textit{2 verschiedene Pendellängen} durchgeführt.


\subsection{Messung der Periodendauer der einzelnen Pendel}
Zu Beginn werden die Pendel auf dieselbe Länge $l$ eingestellt, die Kopplungsfeder wird gegebenenfalls entfernt und es wird \textit{10-mal} die 
\textit{5-fache Periodendauer} bestimmt, indem das Pendel um einen kleinen Winkel $\alpha$ ausgelenkt wird und die Zeit zwischen den Ruhelage 
gemessen wird. Dies wird für beide Pendel durchgeführt.

\subsection{Messung der Periodendauer der gleichsinnigen Schwingung}
Die Pendel werden nun gekoppelt, indem die Feder in die dafür vorgesehenen Löcher an den Pendelstangen eingehängt wird. Nun werden die Pendel beide 
in dieselbe Richtung um denselben Winkel $\alpha$ ausgelenkt und es wird erneut die \textit{5-fache Periodendauer} gemessen. Auch dies wird 
insgesamt \textit{10-mal} erledigt.

\subsection{Messung der Periodendauer der gegensinnigen Schwingung}
Die immer noch gekoppelten Pendel werden in entgegengesetzte Richtungen um denselben Winkel $\alpha$ ausgelenkt und die \textit{5-fache Periodendauer}
wird \textit{10-mal} gemessen.

\subsection{Messung der Schwbungsdauer}
Eins der Pendel wird um einen kleinen Winkel $\alpha$ ausgelenkt, während das andere in seiner Ruhelage verweilt. Die Schwbungsdauer $T_{\symup{S}}$ wird
ermittelt, indem die Zeit gemessen wird, die zwischen dem völligen Stillstand eines Pendels bis zum nächsten völligen Stillstand desselben Pendels vergeht.
Darüber hinaus wird auch die \textit{5-fache Periodendauer} der einzelnen Schwingungen gemessen. Die gesamte Prozedur wird \textit{10-mal} durchgeführt.