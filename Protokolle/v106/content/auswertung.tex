\section{Auswertung}
\label{sec:Auswertung}

Die Versuchsergebnisse der beiden Pendellängen werden getrennt ausgewertet.

\subsection{Kurzes Pendel}
\label{sec:Kurzes Pendel}

Zuerst werden die Schwingungsdauern $T_1$ und $T_2$ der einzelnen Pendel ausgewertet.
Dabei wurde die Zeit jeweils für fünf Perioden gemessen, die Messwerte dazu stehen in \autoref{tab:kurze Pendel einzeln}.

\begin{table}[H]
    \centering
    \caption{Messwerte für die Schwingungsdauern der einzelnen Pendel.}
    \label{tab:kurze Pendel einzeln}
    \begin{tabular}{S[table-format=1.2] S[table-format=1.2]}
        \toprule
        {$5\cdot T_1$ / s} & {$5\cdot T_2$ / s} \\
        \midrule
            7.27 & 7.06 \\
            7.21 & 7.27 \\
            7.20 & 7.28 \\
            7.14 & 7.33 \\
            7.26 & 7.07 \\
            7.26 & 7.33 \\
            7.25 & 7.31 \\
            7.21 & 7.26 \\
            7.33 & 7.25 \\
            7.20 & 7.26 \\
        \bottomrule
    \end{tabular}
\end{table}

Aus diesen Werten wird der Mittelwert mitsamt seinem Fehler bestimmt.
Die dafür verwendeten Formeln lauten:

\begin{equation}  
    \label{eqn:Mittelwert}
    \bar{T} = \frac{1}{N} \sum_{k=1}^{N} T_{\symup{k}}
\end{equation}

\begin{equation}  
    \label{eqn:Standartabweichung}
    \sigma = \sqrt{\frac{1}{N}\sum_{k=1}^{N}(T_{\symup{k}}-\bar{T})^2 }
\end{equation}

\begin{equation}
    \label{eqn:Mittelwertfehler}
    \increment \bar{T} = \frac{\sigma}{\sqrt{N}} 
\end{equation}

Es ergeben sich daraus die Mittelwerte:

\begin{align*}
\bar{T_1} &= \qty{1.446+-0.003}{\second} \\
\bar{T_2} &= \qty{1.448+-0.006}{\second}
\end{align*}

Mit den Fehlertoleranzen lassen sich die beiden Werte vereinen.
Daher können die Pendel als identisch angenommen werden,
und somit nun deren Verhalten bei Kopplung untersucht werden.

Die Messwerte für die gleichphasige und gegenphasige Schwingung der Pendel sind in
\autoref{tab:kurze Pendel gleichsinnig und gegensinnig} festgehalten.

\begin{table}[H]
    \centering
    \caption{Messwerte für die Schwingungsdauern der gleichsinnigen und gegensinnigen Schwingung bei %
    kurzer Pendellänge $l=\qty{50}{\centi\metre}$.}
    \label{tab:kurze Pendel gleichsinnig und gegensinnig}
    \begin{tabular}{S[table-format=1.2] S[table-format=1.2]}
        \toprule
        {$5\cdot T_{+}$ / s} & {$5\cdot T_{-}$ / s} \\
        \midrule
            7.02 & 7.09 \\
            7.41 & 7.08 \\
            7.29 & 7.09 \\
            7.22 & 7.15 \\
            7.28 & 6.83 \\
            7.48 & 7.08 \\
            7.34 & 7.28 \\
            7.15 & 6.98 \\
            7.35 & 7.22 \\
            7.15 & 7.09 \\
        \bottomrule
    \end{tabular}
\end{table}

Mithilfe von \eqref{eqn:Mittelwert}, \eqref{eqn:Standartabweichung} und \eqref{eqn:Mittelwertfehler}
wird zuerst der Mittelwerte der gleichphasigen Schwingung berechnet zu:
\begin{gather*}
    \bar{T}_{+,\symup{exp}} = \qty{1.454+-0.009}{\second}
\end{gather*}
Dieses Messergebnis wird im folgenden mit dem Theoriewert verglichen. Dieser entspricht genau der Schwingungsdauer
eines einzelnen Pendels und lässt sich daher mit Hilfe von \eqref{eq:Schwingungsdauer} berechnen.
\begin{gather*}
    \bar{T}_{+,\symup{theorie}} = \qty{1.418}{\second}
\end{gather*}

Als nächstes wird analog zur gleichphasigen Schwingung der Mittelwert der gegenphasigen Schwingung berechnet.
\begin{gather*}
    \bar{T}_{-,\symup{exp}} = \qty{1.418+-0.008}{\second}
\end{gather*}

Um hier den theoretischen Wert bestimmen zu können, muss die Kopplungskonstante $\kappa$ bekannt sein.
Da hierfür kein theortischer Wert existiert, wird die Konstante durch die experiementellen Ergebnisse
mit \eqref{eq:Kopplungskonstante} ermittelt.
\begin{gather*}
    \kappa = \qty{0.025+-0.008}{}
\end{gather*}

Damit kann nun der theoretische Wert für die Schwingungsdauer der gegenphasigen Schwingung mithilfe von \eqref{eq:T-}
bestimmt werden.
\begin{gather*}
    \bar{T}_{-,\symup{theorie}} = \qty{1.414+-0.001}{\second}
\end{gather*}

Im letzten Teil der Auswertung wird die Schwebungsdauer betrachtet.
Die Messwerte hierfür finden sich in \autoref{tab:kurze Pendel gekoppelt}.

\begin{table}[H]
    \centering
    \caption{Messwerte für die Schwingungsdauer der gekoppelten Schwingung und der Schwebungsdauer bei
    kurzer Pendellänge $l=\qty{50}{\centi\metre}$.}
    \label{tab:kurze Pendel gekoppelt}
    \begin{tabular}{S[table-format=1.2] S[table-format=2.2]}
        \toprule
        {$5\cdot T$ / s} & {$T_{\symup{S}}$ / s} \\
        \midrule
            8.02 & 35.97 \\
            8.12 & 35.84 \\
            8.38 & 35.78 \\
            8.00 & 36.62 \\
            8.14 & 35.25 \\
            7.95 & 36.65 \\
            8.07 & 36.53 \\
            8.13 & 35.88 \\
            8.13 & 36.12 \\
        \bottomrule
    \end{tabular}
\end{table}

Wieder werden die Mittelwerte der gemessenen Zeiten berechnet zu:
\begin{gather*}
    \bar{T}_{\symup{Schwingung},\symup{exp}} = \qty{1.621+-0.008}{\second} \\
    \bar{T}_{\symup{Schwebung},\symup{exp}} = \qty{36.07+-0.15}{\second}
\end{gather*}
Abschließend wird für die Schwebungsdauer mit \eqref{eq:T_S} ein theoretischer Wert berechnet.
\begin{gather*}
    \bar{T}_{\symup{Schwebung},\symup{theorie}} = \qty{57+-19}{\second}
\end{gather*}

\subsection{Langes Pendel}
\label{sec:Langes Pendel}

Die Auswertung für das lange Pendel ist vollständig analog zum kurzen Pendel.
Da alle Formeln identisch verwendet werden, folgen hier nur die Ergebnisse.

Die individuellen Schwingungsdauern der Pendel finden sich in \autoref{tab:lange Pendel einzeln}.

\begin{table}[H]
    \centering
    \caption{Messwerte für die Schwingungsdauern der einzelnen Pendel bei langer Pendellänge $l=\qty{100}{\centi\metre}$.}
    \label{tab:lange Pendel einzeln}
    \begin{tabular}{S[table-format=2.2] S[table-format=2.2]}
        \toprule
        {$5\cdot T_1$ / s} & {$5\cdot T_2$ / s} \\
        \midrule
        9.76  &	9.95  \\
        9.86  &	9.76  \\
        9.96  &	9.82  \\
        9.83  &	9.92  \\
        9.90  &	9.89  \\
        9.89  &	9.86  \\
        10.02 &	9.87  \\
        9.80  &	9.82  \\
        9.79  &	10.08 \\
        9.82  &	9.75  \\
        \bottomrule
    \end{tabular}
\end{table}

Gemittelt ergeben sich folgende Werte:
\begin{align*}
    \bar{T_1} &= \qty{1.973+-0.005}{\second} \\
    \bar{T_2} &= \qty{1.974+-0.006}{\second}.
\end{align*}
Wie in \autoref{sec:Kurzes Pendel} liegen die Mittelwerte innerhalb ihrer gegenseitigen
Fehlertoleranzen und die Pendel können als identisch angenommen werden.

Für die gleichphasige und gegenphasige Schwingung sind die Messwerte in \autoref{tab:lange Pendel gleichsinnig und gegensinnig}
festgehalten.

\begin{table}[H]
    \centering
    \caption{Messwerte für die Schwingungsdauern der gleichsinnigen und gegensinnigen Schwingung bei%
    langer Pendellänge $l=\qty{100}{\centi\metre}$.}
    \label{tab:lange Pendel gleichsinnig und gegensinnig}
    \begin{tabular}{S[table-format=2.2] S[table-format=1.2]}
        \toprule
        {$5\cdot T_{+}$ / s} & {$5\cdot T_{-}$ / s} \\
        \midrule
        9.80  & 8.58 \\
        9.92  & 8.64 \\
        9.89  & 8.51 \\
        9.82  & 8.78 \\
        9.89  & 8.64 \\
        10.01 & 8.65 \\
        9.82  & 8.70 \\
        9.94  & 8.64 \\
        9.88  & 8.58 \\
        9.82  & 8.90 \\
        \bottomrule
    \end{tabular}
\end{table}

Die Ergebnisse für deren Mittelwerte, die neue Kopplungskonstante $\kappa$ und die Theoriewerte lauten:
\begin{align*}
    \bar{T}_{+,\symup{exp}} &= \qty{1.976+-0.004}{\second} \\
    \bar{T}_{+,\symup{theorie}} &= \qty{2.006}{\second} \\
    \bar{T}_{-,\symup{exp}} &= \qty{1.732+-0.007}{\second} \\
    \kappa &= \qty{0.131+-0.004}{} \\
    \bar{T}_{-,\symup{theorie}} &= \qty{1.9799+-0.0008}{\second}
\end{align*}

Zuletzt wird wieder die Schwebungsdauer untersucht, die Messwerte finden sich in \autoref{tab:lange Pendel gekoppelt}.

\begin{table}[H]
    \centering
    \caption{Messwerte für die Schwingungsdauer der gekoppelten Schwingung und der Schwebungsdauer bei
    langer Pendellänge $l=\qty{100}{\centi\metre}$.}
    \label{tab:lange Pendel gekoppelt}
    \begin{tabular}{S[table-format=1.2] S[table-format=2.2]}
        \toprule
        {$5\cdot T$ / s} & {$T_{\symup{S}}$ / s} \\
        \midrule
        8.20 & 14.88 \\
        8.24 & 13.91 \\
        8.40 & 14.12 \\
        8.33 & 13.82 \\
        8.39 & 14.42 \\
        8.47 & 14.93 \\
        8.72 & 14.24 \\
        8.53 & 14.90 \\
        8.52 & 14.64 \\
        8.92 & 14.42 \\
        \bottomrule
    \end{tabular}
\end{table}

Hier ergeben sich die Mittelwerte:
\begin{gather*}
    \bar{T}_{\symup{Schwingung},\symup{exp}} = \qty{1.694+-0.014}{\second} \\
    \bar{T}_{\symup{Schwebung},\symup{exp}} = \qty{14.43+-0.13}{\second}.
\end{gather*}
Der Theoriewert bestimmt sich für das lange Pendel zu:
\begin{gather*}
    \bar{T}_{\symup{Schwebung},\symup{theorie}} = \qty{14.1+-0.5}{\second}.
\end{gather*}