\section{Auswertung}
\label{sec:Auswertung}

Die Versuchsergebnisse der beiden Pendellängen werden getrennt ausgewertet.

\subsection{Kurzes Pendel}
\label{Kurzes Pendel}

Zuerst werden die Schwingungsdauern $T_1$ und $T_2$ der einzelnen Pendel ausgewertet.
Dabei wurde die Zeit jeweils für fünf Perioden gemessen, die Messwerte dazu stehen in \autoref{tab:kurze Pendel einzeln}

\begin{table}[H]
    \centering
    \caption{Messwerte für die Schwingungsdauern der einzelnen Pendel.}
    \label{tab:kurze Pendel einzeln}
    \begin{tabular}{S[table-format=1.2] S[table-format=1.2]}
        \toprule
        {$5\cdot T_1$ / s} & {$5\cdot T_2$ / s} \\
        \midrule
            7.27 & 7.06 \\
            7.21 & 7.27 \\
            7.20 & 7.28 \\
            7.14 & 7.33 \\
            7.26 & 7.07 \\
            7.26 & 7.33 \\
            7.25 & 7.31 \\
            7.21 & 7.26 \\
            7.33 & 7.25 \\
            7.20 & 7.26 \\
        \bottomrule
    \end{tabular}
\end{table}

Aus diesen Werten wird der Mittelwert mitsamt seinem Fehler bestimmt.
Die dafür verwendeten Formeln lauten:

\begin{equation}  
    \label{eqn:Mittelwert}
    \bar{T} = \frac{1}{N} \sum_{k=1}^{N} T_{\text{k}}
\end{equation}

\begin{equation}  
    \label{eqn:Standartabweichung}
    \sigma = \sqrt{\frac{1}{N}\sum_{k=1}^{N}(T_{\symup{k}}-\bar{T})^2 }
\end{equation}

\begin{equation}
    \label{eqn:Mittelwertfehler}
    \increment \bar{T} = \frac{\sigma}{\sqrt{N}} 
\end{equation}

Es ergeben sich daraus die Mittelwerte:

\begin{align*}
\bar{T_1} &= \qty{1.446+-0.003}{\second} \\
\bar{T_2} &= \qty{1.448+-0.006}{\second}
\end{align*}

Mit den Fehlertoleranzen lassen sich die beiden Werte vereinen.
Daher können die Pendel als identisch angenommen werden,
und somit nun deren Verhalten bei Kopplung untersucht werden.

Mittelwert gleichphasig: 1.454+/-0.009 s
gleichphasig theoretisch: 1.4185033534428875 s


Mittelwert gegenphasig: 1.418+/-0.008 s
Kappa: 0.025+/-0.008
gegenphasig theoretisch: 1.4149+/-0.0012 s


Mittelwert gekoppelt: 1.621+/-0.008 s
Mittelwert Schwebung: 36.07+/-0.15 s
Schwebung theoretisch: 57+/-19 s

\subsection{Langes Pendel}
\label{Langes Pendel}

Mittelwert 1. Pendel: 1.973+/-0.005 s
Mittelwert 2. Pendel: 1.974+/-0.006 s


Mittelwert gleichphasig: 1.976+/-0.004 s
gleichphasig theoretisch: 2.0060666807106475 s


Mittelwert gegenphasig: 1.732+/-0.007 s
Kappa: 0.131+/-0.004
gegenphasig theoretisch: 1.9799+/-0.0008 s


Mittelwert gekoppelt: 1.694+/-0.014 s
Mittelwert Schwebung: 14.43+/-0.13 s
Schwebung theoretisch: 14.1+/-0.5 s