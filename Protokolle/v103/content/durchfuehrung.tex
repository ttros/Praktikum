\section{Durchführung}
\label{sec:Durchführung}

Es werden insgesamt vier Messreihen durchgeführt, für die 2 verschiende Stäbe aus Kupfer verwendet werden.
Die Abmessungen der Stäbe sowie deren Massen werden vor Ort gemessen.
Der erste Kupferstab besitzt einen runden Querschnitt mit Durchmesser $d$, die Daten lauten:

\begin{gather}
    d=10\pm0,1\,\symup{mm} \notag \\
    l_{rund}=590\pm1\,\symup{mm} \notag \\
    m_{rund}=412\pm0,1\,\symup{g} \notag.
\end{gather}

Der andere Kupferstab weist einen quadratischen Querschnitt mit Kantenlänge $a$ auf:

\begin{gather}
    a=10\pm0,1\,\symup{mm} \notag \\
    l_{eckig}=600\pm1\,\symup{mm} \notag \\
    m_{eckig}=535,6\pm0,1\,\symup{g} \notag.
\end{gather}

Zuerst wird der Kupferstab mit rundem Querschnitt einseitg eingespannt und dessen Durchbiegung ohne Gewicht gemessen.
Dabei werden Messwerte in Abständen von $\Delta 3\,\symup{cm}$ aufgenommen.
Darauf platziert man ein Gewicht von $550\,\symup{g}$ am Stabende umd misst wie zuvor die Durchbiegung des Stabs.

Als nächstes wird die Apparatur so umgebaut, dass der runde Kupferstab auf beiden Seiten aufliegt.
Erneut werden zuerst Messwerte ohne ein Gewicht genommen, danach werden $1750\,\symup{g}$
in der Mitte des Stabes platziert und wieder wird die Durchbiegung ermittelt.

Die selbe Vorgehensweise wird jetzt mit dem eckigen Kupferstab durchgeführt.
Bei einseitiger Einspannung wird dieser mit $750\,\symup{g}$ und bei beidseitiger Auflage mit $1750\,\symup{g}$ belastet.

Die Gewichte werden dabei jeweils so gewählt, dass eine maximale Auslenkung von mindestens $3\,\symup{cm}$ erzeugt wird.