\section{Auswertung}
\label{sec:Auswertung}

\subsection{Elastizitätsmodul des runden Stabs}  %%%%%%%%%%%%%%%%%%%%%%%%%%%%%%%%% RUND
\label{sec:Elastizitätsmodul rund}

Die Messwerte für die Durchbiegung des runden Stabes bei einseitiger Einspannung ohne Gewicht $D_{0}(x)$
und mit Gewicht $D_{\symup{G}}(x)$ sind in \autoref{tab:rund einseitig} aufgeführt.
Deren Messfehler wird dabei mit $\Delta D(x) =  0,01\,\symup{mm}$ angesetzt, während bei der horizontalen Position
$x$ mit einem Fehler von $\Delta x = 0,1\,\symup{cm}$ gerechnet wird.
Die tatsächliche Durchbiegung $D(x)$ wird aus der Differenz der Werte berechnet:

\begin{equation}
  D(x)= D_{0}(x)-D_{\symup{G}}(x) \notag
  \label{eq:Delta D}
\end{equation}


Um eine lineare Regressionsrechnung durchzuführen, wird eine Hilfsvariable $\eta(x)$ eingeführt:

\begin{equation}
  \eta(x)=Lx^{2}-\frac{1}{3}x^{3} \notag
  \label{eq:eta einseitig rund}
\end{equation}
$L$ wird wie in \autoref{fig:stab einseitig eingespannt} zu sehen bestimmt und ergibt sich zu $L=\qty{0,48}{\metre}$.
Setzt man $\eta(x)$ in \autoref{eq:Durchbiegung einseitig} ein, resultiert ein linearer Zusammenhang zwischen $D(\eta)$ und $\eta(x)$:

\begin{equation}
  D(\eta)=\frac{F}{2EI}\eta
  \label{eq:Geradengleichung}
\end{equation}

Nun wird $\eta(x)$ in Abhängigkeit von $D(x)$ in \autoref{fig:rund_einseitig} geplottet.
Die Regression liefert dann einen Wert für die Steigung $m$, aus dem durch Umstellen das Elastizitätsmodul bestimmt werden kann:

\begin{equation}
  \frac{F}{2EI}=m \Leftrightarrow E=\frac{F}{2Im}
  \label{eq:Elastizitätsmodul aus Steigung}
\end{equation}

Das Flächenträgheitsmoment $I$ ist aus \autoref{eq:Flächenträgheitsmoment rund} bekannt, die Gewichtskraft
bestimmt sich aus $F=Mg$.

\begin{table} [H]
  \centering
  \caption{Durchbiegung runder Stab einseitig eingespannt}
  \label{tab:rund einseitig}
  \begin{tabular}{S[table-format=2.0] S[table-format=2.0] S[table-format=2.0]}
    \toprule
    {$x$ [cm]} & {$D_{0}(x)$ [mm]} & {$D_{\symup{G}}(x)$ [mm] ($550\,\symup{g}$)} \\
    \midrule
     3 & 8.00 & 7.95 \\
     6 & 7.93 & 7.82 \\
     9 & 7.81 & 7.59 \\
    12 & 7.63 & 7.29 \\
    15 & 7.42 & 6.90 \\ 
    18 & 7.15 & 6.43 \\
    21 & 6.92 & 5.98 \\
    24 & 6.63 & 5.45 \\
    27 & 6.34 & 4.88 \\
    30 & 6.03 & 4.31 \\
    33 & 5.74 & 3.69 \\
    36 & 5.43 & 3.05 \\
    39 & 4.96 & 2.35 \\
    42 & 4.68 & 1.60 \\
    45 & 4.30 & 0.98 \\
    48 & 3.90 & 0.24 \\ 
    \bottomrule
  \end{tabular}
\end{table}

\begin{figure} [H]
  \centering
  \includegraphics[height=8cm]{build/plot_rund_einseitig.pdf}
  \caption{Durchbiegung runder Stab einseitig eingespannt.}
  \label{fig:rund_einseitig}
\end{figure}

Als nächstes wird die beidseitige Auflage mit Gewicht in der Mitte betrachet,
die Messwerte finden sich in \autoref{tab:rund beidseitig}.
Da für diesen Fall die zwei theoretische Formeln (\autoref{eq:Durchbiegung beidseitig 1} und \autoref{eq:Durchbiegung beidseitig 2}) existieren,
werden nun für jede Hälfte die Werte unabhängig geplottet.

Dementsprechend werden auch zwei Hilfsvariablen verwendet:

\begin{align}
  \eta_{1}(x)&=3L^{2}x-4x^{3}  \notag \\
  \eta_{2}(x)&=4x^{3}-12Lx^{2}+9L^{2}x-L^{3} \notag
\end{align}
$L$ wird in diesem Fall mit $L=\qty{0,54}{\metre}$ gemessen.
Dabei ist $\eta_{1}(x)$ für $0\leq x\leq \frac{L}{2}$ gültig und $\eta_{2}(x)$ für $\frac{L}{2}\leq x\leq L$.
Eingesetzt in \autoref{eq:Durchbiegung beidseitig 1} und \autoref{eq:Durchbiegung beidseitig 2}
resultieren erneut die linearen Zusammenhänge:

\begin{align}
  D_{1}(\eta_{1})=\frac{F}{48EI}\eta_{1} \notag \\
  D_{2}(\eta_{2})=\frac{F}{48EI}\eta_{2} \notag
\end{align}

Umstellen der Gleichungen analog zu \autoref{eq:Geradengleichung} und Bestimmung von m durch lineare Regressionen
liefert abschließend zwei weitere Werte für das Elastizitätsmodul.

\begin{table} [H]
  \centering
  \caption{Durchbiegung runder Stab beidseitig aufliegend}
  \label{tab:rund beidseitig}
  \begin{tabular}{S[table-format=2.0] S[table-format=2.0] S[table-format=2.0]}
    \toprule
    {$x$ [cm]} & {$D_{0}(x)$ [mm]} & {$D_{G}(x)$ [mm] ($1750\,\symup{g}$)} \\
    \midrule
     3 & 7.95 & 7.79 \\
     6 & 7.89 & 7.59 \\
     9 & 7.80 & 7.37 \\
    12 & 7.73 & 7.17 \\
    15 & 7.65 & 6.99 \\
    18 & 7.58 & 6.82 \\
    21 & 7.61 & 6.75 \\
    24 & 7.60 & 6.70 \\
    27 & 7.61 & 6.74 \\
    30 & 7.19 & 6.28 \\
    33 & 7.25 & 6.38 \\
    36 & 7.35 & 6.53 \\
    39 & 7.42 & 6.70 \\
    42 & 7.51 & 6.90 \\
    45 & 7.63 & 7.14 \\
    48 & 7.71 & 7.44 \\
    51 & 7.85 & 7.65 \\
    54 & 8.02 & 7.95 \\
    \bottomrule
  \end{tabular}
\end{table}

\begin{figure} [H]
  \centering
  \includegraphics[height=8cm]{build/plot_rund_beidseitig.pdf}
  \caption{Durchbiegung runder Stab beidseitig aufliegend.}
  \label{fig:rund_beidseitig}
\end{figure}

Die linearen Regressionen werden mithilfe der Python Erweiterungen \textit{numpy}\cite{numpy} und \textit{scipy}\cite{scipy} 
durchgeführt und liefern die Ausgleichsgeraden vom Typ $D(x)=m\cdot\eta(x) + b$ jeweils mit den Parametern
\begin{align}
  m_{\symup{einseitig}} &= 0,04947\pm0,00032 & b_{\symup{einseitig}} &= 0,0000\pm0,0003 \notag \\
  m_{\symup{beidseitig,1}} &= 0,00565\pm0,00012 & b_{\symup{beidseitig,1}} &= 0,0000\pm0,0001 \notag \\
  m_{\symup{beidseitig,2}} &= 0,00565\pm0,00019 & b_{\symup{beidseitig,2}} &= 0,0000\pm0,0001. \notag
\end{align}


Mit \autoref{eq:Elastizitätsmodul aus Steigung} ergibt sich das Elastizitätsmodul des runden Stabs dann zu:

\begin{align}
  E_{\symup{einseitig}}&=\qty{111+-9}{\giga\pascal} \notag \\
  E_{\symup{beidseitig,1}}&=\qty{129+-11}{\giga\pascal} \notag \\
  E_{\symup{beidseitig,2}}&=\qty{129+-11}{\giga\pascal}. \notag
\end{align}

Der Mittelwert, als bester Schätzwert für das tatsächliche Elastizitätsmodul $E$ von Kupfer, lässt sich zu
\begin{equation}
  \bar{E}_{\symup{rund}} = \qty{123+-10}{\giga\pascal} \notag
\end{equation}
bestimmen.

\subsection{Elastizitätsmodul des eckigen Stabs}  %%%%%%%%%%%%%%%%%%%%%%%%%%%%%%%%% ECKIG
\label{sec:Elastizitätsmodul eckig}

Das Vorgehen beim eckigen Stab ist identisch wie in \autoref{sec:Elastizitätsmodul rund}.
Das Flächenträgheitsmoment ist nun ein anderes und wird mit \autoref{eq:Flächenträgheitsmoment eckig} bestimmt.
Auch die Messfehler werden unverändert wie in \autoref{sec:Elastizitätsmodul rund} behandelt.

\begin{table} [H]
  \centering
  \caption{Durchbiegung eckiger Stab einseitg eingespannt}
  \label{tab:eckig einseitig}
  \begin{tabular}{S[table-format=2.0] S[table-format=2.0] S[table-format=2.0]}
    \toprule
    {$x$ [cm]} & {$D_{0}(x)$ [mm]} & {$D_{G}(x)$ [mm] ($750\,\symup{g}$)} \\
    \midrule
     3 & 8.00 & 7.95 \\
     6 & 7.99 & 7.88 \\
     9 & 7.98 & 7.77 \\
    12 & 7.93 & 7.60 \\
    15 & 7.88 & 7.41 \\
    18 & 7.80 & 7.14 \\
    21 & 7.79 & 6.92 \\
    24 & 7.72 & 6.63 \\
    27 & 7.61 & 6.29 \\
    30 & 7.51 & 5.92 \\
    33 & 7.41 & 5.57 \\
    36 & 7.37 & 5.22 \\
    39 & 7.21 & 4.80 \\
    42 & 7.12 & 4.37 \\
    45 & 6.96 & 3.89 \\
    48 & 6.82 & 3.44 \\
    \bottomrule
  \end{tabular}
\end{table}

\begin{figure} [H]
  \centering
  \includegraphics[height=8cm]{build/plot_eckig_einseitig.pdf}
  \caption{Durchbiegung eckiger Stab einseitig eingespannt.}
  \label{fig:eckig_einseitig}
\end{figure}

\begin{table} [H]
  \centering
  \caption{Durchbiegung eckiger Stab beidseitig aufliegend}
  \label{tab:eckig beidseitig}
  \begin{tabular}{S[table-format=2.0] S[table-format=2.0] S[table-format=2.0]}
    \toprule
    {$x$ [cm]} & {$D_{0}(x)$ [mm]} & {$D_{G}(x)$ [mm] ($1750\,\symup{g}$)} \\
    \midrule
     3 & 8.00 & 7.91 \\
     6 & 8.02 & 7.85 \\
     9 & 8.04 & 7.79 \\
    12 & 8.09 & 7.69 \\
    15 & 8.08 & 7.62 \\
    18 & 8.08 & 7.55 \\
    21 & 8.15 & 7.60 \\
    24 & 8.20 & 7.60 \\
    27 & 8.23 & 7.63 \\
    30 & 7.42 & 6.75 \\
    33 & 7.44 & 6.89 \\
    36 & 7.54 & 6.88 \\
    39 & 7.62 & 7.14 \\
    42 & 7.68 & 7.26 \\
    45 & 7.75 & 7.41 \\
    48 & 7.67 & 7.55 \\
    51 & 7.89 & 7.77 \\
    54 & 8.00 & 7.84 \\ 
    \bottomrule
  \end{tabular}
\end{table}

\begin{figure} [H]
  \centering
  \includegraphics[height=8cm]{build/plot_eckig_beidseitig.pdf}
  \caption{Durchbiegung eckiger Stab beidseitig aufliegend.}
  \label{fig:eckig_beidseitig}
\end{figure}

Die linearen Regressionen werden analog zu \autoref{sec:Elastizitätsmodul eckig} durchgeführt und es ergeben sich die Werte
\begin{align}
  m_{\symup{einseitig}} &= 0,04516\pm0,00014 & b_{\symup{einseitig}} &= 0,0000\pm0,0001 \notag \\
  m_{\symup{beidseitig,1}} &= 0,00404\pm0,00015 & b_{\symup{beidseitig,1}} &= 0,0000\pm0,0001 \notag \\
  m_{\symup{beidseitig,2}} &= 0,0038\pm0,0005 & b_{\symup{beidseitig,2}} &= 0,0000\pm0,0005. \notag
\end{align}

Es resultiert für das Elastizitätsmodul des eckigen Stabs mit \autoref{eq:Elastizitätsmodul aus Steigung}:

\begin{align}
  E_{\symup{einseitig}}&=\qty{98+-4}{\giga\pascal} \notag \\
  E_{\symup{beidseitig,1}}&=\qty{106+-6}{\giga\pascal}\notag \\
  E_{\symup{beidseitig,2}}&=\qty{113+-16}{\giga\pascal}. \notag
\end{align}

Der Mittelwert beträgt hierbei
\begin{equation}
  \bar{E}_{\symup{eckig}} = \qty{106+-7}{\giga\pascal}. \notag
\end{equation}

%Elastizitätsmodul rund einseitig: (1.11+/-0.09)e+11
%Elastizitätsmodul rund beidseitig 1: (1.29+/-0.11)e+11
%Elastizitätsmodul rund beidseitig 2: (1.29+/-0.11)e+11
%
%Elastizitätsmodul eckig einseitig: (9.8+/-0.4)e+10
%Elastizitätsmodul eckig beidseitig 1: (1.06+/-0.06)e+11
%Elastizitätsmodul eckig beidseitig 2: (1.13+/-0.16)e+11