\section{Diskussion}
\label{sec:Diskussion}
Ein Literaturwert \cite[S. 829 / E77]{czichos} für das Elastizitätsmodul von Kupfer lautet
\begin{gather}
    E_{\symup{lit}} = \qty{125}{\giga\pascal}. \notag
\end{gather}
Für den runden Stab ergibt sich das Elastizitätsmodul $E$ zu
\begin{gather}
    \bar{E}_{\symup{rund}} = \qty{123+-10}{\giga\pascal}. \notag
\end{gather}
Der gemessene Wert lässt sich also hierbei mit dem Literaturwert bestätigen.

Das Elastizitätsmodul $E$ des eckigen Stabes wird allerdings zu 
\begin{gather}
    \bar{E}_{\symup{eckig}} = \qty{106+-7}{\giga\pascal} \notag
\end{gather}
bestimmt. Der Literaturwert liegt hierbei außerhalb der Fehlertoleranz und es wird eine Abweichung 
$\Delta \bar{E}_{\symup{eckig}} \approx \qty{18}\percent$ festgestellt.

Generell ist zu beachten, dass die zur Bestimmung der Auslenkung $D(x)$ verwendeten Messuhren mitunter große Ungenauigkeiten
verursachen könnten. Es ist zu beobachten, dass selbst kleine Erschütterungen des Versuchsaufbaus große Schwankungen in den gemessenen Werten ergeben.
Außerdem sind die Werte, die mit der linken Messuhr abgelesen werden, nicht annähernd kongruent zu Messwerten der rechten Messuhr, 
auch wenn an ein und derselben Stelle gemessen wird. Dies lässt darauf schließen, dass gegebenenfalls ein systematischer Fehler
in der Methode der Messung besteht, da die Messuhren nicht in ihrem angegebenen Gütebereich messen.

Auch die vergleichsweise geringe maximale Auslenkung des eckigen Stabes verursacht Ungenauigkeiten, weil so nur ein kleinerer
Messbereich der Messuhren genutzt werden kann.
