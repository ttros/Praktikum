\section{Diskussion}
\label{sec:Diskussion}
Die dynamische Viskosität von destilliertem Wasser bei Raumtemperatur $T=20\unit{\celsius}$ wird in der
Literatur mit $\eta = 1,002\unit{\milli\pascal\second}$ \cite{dichte} angegeben. Die Messung unter dem 
angegebenen Versuchsaufbau liefert eine dynamische Viskosität von 
\begin{equation}
  \eta_{\symup{oben}} = \eta_{\symup{unten}} = (1,23\pm0,05)\unit{\milli\pascal\second}. \notag
\end{equation}
Dieser Wert lässt sich auch unter 
Betrachtung der angegebenen Messunsicherheiten nicht mit dem Literaturwert in Einklang bringen. Die große Diskrepanz liegt
eventuell darin begründet, dass für die Berechnung der dynamischen Viskosität $\eta$ zunächst mehrere andere Größen 
bestimmt werden müssen. Zur Berechnung der Dichte $\rho_{\symup{kl}}$ der kleinen Glaskugel muss auf eine ohne Messfehler 
angegebene Masse $m_{\symup{kl}}$ zurückgegriffen werden. Auch kann es durchaus sein, dass die Strömung nicht in dem Maße
laminar ist, wie die Reynoldszahl $Re_{\symup{kl}}$ vermuten lässt. Dies ließe sich mit etwaigen Unebenheiten auf der Kugeloberfläche
begründen. 

Auch die Werte der dynamischen Viskosität bei verschiedenen anderen Temperaturen liegen außerhalb des Toleranzbereichs, wie in 
\autoref{tab:Viskositäten mit Literaturwerten} zu sehen ist. Dabei fällt auf, dass die Messwerte in allen Fällen über den Literaturwerten
liegen. Dies lässt Vermuten, dass die gemessenen Temperaturen $T$ nicht den tatsächlichen Temperaturen in dem Fallrohr entsprechen.
Die Temperatur im Fallrohr wird durch das Erhitzen eines Wasserbades, welches das Fallrohr umspült, reguliert. Möglicherweise weicht die
Wassertemperatur im Fallrohr zum Zeitpunkt der Messung um mehr als den betrachteten Fehler $\Delta T = 1\unit{\celsius}$ ab.
\begin{table} [H]
    \centering
    \caption{Gemessene dynamische Viskositäten im Vergleich zu Literaturwerten}
    \label{tab:Viskositäten mit Literaturwerten}
    \begin{tabular}{S[table-format=2.0] c c S[table-format=1.3]}
      \toprule
      {$T$ [°C]} & {$\eta_{\symup{oben}}$ [$\unit{\milli\pascal\second}$]} & {$\eta_{\symup{unten}}$ [$\unit{\milli\pascal\second}$]}%
      & {$\eta_{\symup{lit}}$ [$\unit{\milli\pascal\second}$]\cite{dichte}} \\
      \midrule
      22 & 1,125 \pm 0,050 & 1,137 \pm 0,050 & 0,954\\
      27 & 1,014 \pm 0,045 & 1,040 \pm 0,046 & 0,851\\
      30 & 0,934 \pm 0,042 & 0,958 \pm 0,043 & 0,797\\
      33 & 0,890 \pm 0,040 & 0,886 \pm 0,040 & 0,749\\
      36 & 0,846 \pm 0,038 & 0,835 \pm 0,038 & 0,705\\
      39 & 0,769 \pm 0,035 & 0,767 \pm 0,035 & 0,665\\
      42 & 0,748 \pm 0,034 & 0,748 \pm 0,035 & 0,629\\
      45 & 0,700 \pm 0,033 & 0,696 \pm 0,033 & 0,596\\
      48 & 0,681 \pm 0,032 & 0,672 \pm 0,032 & 0,566\\
      51 & 0,642 \pm 0,030 & 0,646 \pm 0,031 & 0,538\\
      \bottomrule
    \end{tabular}
  \end{table}
