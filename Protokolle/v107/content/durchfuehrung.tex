\section{Durchführung}
\label{sec:Durchführung}

\subsection{Bestimmung der Viskositaet und Reinoldszahl}
\label{sec:Bestimmung der Viskositaet und Reinoldszahl}

Zu Beginn werden der Durchmesser $d$ und die Masse $m$ der kleinen sowie der großen Glaskugel bestimmt,
sodass anschließend deren Dichten berechnet werden können. Die Apperaturkonstante der kleinen Kugel ist bekannt.

Das Höppler-Viskosimeter wird nun mit destilliertem Wasser befüllt, wobei alle eingeschlossenen Luftblasen entfernt werden.
Für diesen Versuchsteil besitzt das Wasser eine konstante Temperatur von 18 °C.
Zur Bestimmung der Reinoldszahl wird die Viskositaet des destillierten Wassers bestimmt, indem
die Fallzeit der kleinen Kugel im Viskosimeter gemessen wird. Als Orientierung dafür werden zwei rote 
Markierungen mit einem Abstand von 5cm gewählt.

Da die Fallzeit von der Oberseite zur Unterseite nicht genau identisch ist mit der von der Unterseite zur Oberseite
(nach Drehung um 180° des Viskosimeters) werden diese Fälle seperat gemessen und ausgewertet.

Es werden 10 Werte je Richtung gemessen und anschließend gemittelt. Aus den Ergebnissen wird in der Auswertung die Viskositaet
sowie die Reinoldszahl berechnet.

\subsection{Bestimmung der Apperaturkonstante der großen Kugel}
\label{sec:Bestimmung der Apperaturkonstante der großen Kugel}
Als nächstes wird die kleine Kugel aus dem Höppler-Viskosimeter entfernt und gegen die Große getauscht.
Auch hier besitzt das Wasser eine konstante Temperatur von 18 °C.
Da die Viskositaet des destillierten Wassers bereits aus \autoref{sec:Bestimmung der Viskositaet und Reinoldszahl} bekannt ist,
kann nun die Apperaturkonstante der großen Kugel bestimmt werden.

Dafür wird deren Fallzeit wie in \autoref{sec:Bestimmung der Viskositaet und Reinoldszahl} für beide Richtungen getrennt gemessen.
Aufgrund der längeren Fallzeit ist die Messung genauer und 5 Werte je Richtung genügen.

\subsection{Temperaturabhängigkeit der Fallzeit}
\label{sec:Temperaturabhängigkeit der Fallzeit}
Für den letzten Teil des Versuchs wird erneut die große Kugel verwendet.
Die Temperatur des destillierten Wassers im Höppler-Viskosimeter wird nun durch Heizen des Wasserbades variert.
Ausgehend von 22°C wird die Temperatur in ca. 3°C Schritten erhöht bis 50°C erreicht sind, um ein Sieden des
Wassers zu vermeiden wird dieser Wert nicht überschritten.

Für jeden Temperaturwert wird die Fallzeit der großen Kugel 2 mal gemessen,
und dies wie in \autoref{sec:Bestimmung der Viskositaet und Reinoldszahl}
und \autoref{sec:Bestimmung der Apperaturkonstante der großen Kugel} jeweils getrennt für beide Richtungen.