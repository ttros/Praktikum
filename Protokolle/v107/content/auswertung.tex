\section{Auswertung}
\label{sec:Auswertung}

% O_k_mean: 6.58+/-0.16
% U_k_mean: 6.58+/-0.16
% O_g_mean: 43.17+/-0.22
% U_g_mean: 42.70+/-0.22
% Dichte kl. Kugel 2.22+/-0.04 in g/cm^3
% Dichte gr. Kugel 2.38+/-0.04 in g/cm^3
% Viskositaet kl. Kugel oben: 1.23+/-0.05
% Viskositaet kl. Kugel unten: 1.23+/-0.05
% Apperaturkonstante gr. Kugel oben: 0.0103+/-0.0006
% Apperaturkonstante gr. Kugel unten: 0.0104+/-0.0006
% Reinoldszahl kl. Kugel oben: 0.0218+/-0.0011
% Reinoldszahl kl. Kugel unten: 0.0218+/-0.0011
% Reinoldszahl gr. Kugel oben: 0.00356+/-0.00016
% Reinoldszahl gr. Kugel unten: 0.00360+/-0.00016
% Geradegleichug Oben: 22.4+/-1.6*1/T + 2.7+/-1.6
% Geradegleichug Unten: 23.4+/-1.6*1/T + 2.7+/-1.6

\subsection{Berechnung der Viskosität von destilliertem Wasser bei Raumtemperatur}
\label{sec:Berechnung der Viskosität von destilliertem Wasser bei Raumtemperatur}
\begin{figure} [H]
  Aus dem gemessenen Durchmesser der kleinen Glaskugel $d_{kl}=(1,565\pm0.01) \unit{\centi\metre}$ und der angegebenen Masse
  $m_{\symup{kl}}=4.4531\unit{\gram}$ wird die Dichte berechnet:
  \begin{equation}
    \rho_{\symup{kl}}=\frac{\emph{m}_{\symup{kl}}}{V_{\symup{kl}}}=%
    \frac{m_{\symup{kl}}}{\frac{4}{3}\symup{\pi}\left(\frac{d}{2}\right)^3}=(2,22\pm0,04) \unit{\gram\per\cubic\centi\metre}
  \end{equation}
  Ein Literaturwert für die Dichte von Wasser bei Raumtemperatur $T=20 \unit{\celsius}$ lautet
  $\rho_{\symup{Wasser}}=0,99823 \unit{\gram\per\cubic\centi\metre}$\cite[551/C23]{czichos}
  
\end{figure}

Die Fallzeiten der kleinen Kugel sind in folgender Tabelle aufgeführt:

\begin{table} [H]
  \centering
  \caption{Fallzeiten der kleinen Kugel bei Start von oben bzw. unten}
  \label{tab:Fallzeiten kleine Kugel}
  \begin{tabular}{S[table-format=2.0] S[table-format=1.1]}
    \toprule
    {$t$ oben [s]} & {$t$ unten [s]} \\
    \midrule
    6.62 &	6.76 \\
    6.62 &	6.50 \\
    6.61 &	6.43 \\
    6.68 &	6.50 \\
    6.56 &	6.56 \\
    6.49 &	6.68 \\
    6.56 &	6.64 \\
    6.44 &	6.50 \\
    6.50 &	6.49 \\
    6.69 &	6.69 \\
    \bottomrule
  \end{tabular}
\end{table}

Dabei wird der Messfehler der Fallzeit mit $\Delta t = 0.5\symup{s}$ angesetzt.
Durch Mittelwertbildung erhält man abschließend folgende Ergebnisse:

\begin{gather}
  \overline{t}_{oben} = (6,58\pm0,16) \symup{mPat} \notag \\
  \overline{t}_{unten} = (6,58\pm0,16) \symup{mPat} \notag .
\end{gather}

Setzt man diese Werte zusammen mit der gegebenen Apperaturkonstante $K=0.07640\frac{\symup{mPacm^{^{3}}}}{\symup{g}}$
und den ermittelten Dichten $\rho_{kl}=(2,22\pm0,04) \unit{\gram\per\cubic\centi\metre}$ und
$\rho_{Fl}=\rho_{\symup{Wasser}}=0,99823 \unit{\gram\per\cubic\centi\metre}$ in
\autoref{eqn:Viskositaet} ein, erhält man folgende Ergebnisse für die Viskositaet des destillierten Wassers:

\begin{gather}
  \eta_{oben} = (1,23\pm0,05) \symup{s} \notag \\
  \eta_{unten} = (1,23\pm0,05) \symup{s} \notag .
\end{gather}

\subsection{Berechnung der Apperaturkonstante der großen Kugel}
\label{sec:Berechnung der Apperaturkonstante der großen Kugel}

\begin{table} [H]
  \centering
  \caption{Fallzeiten der großen Kugel bei Start von oben bzw. unten}
  \label{tab:Fallzeiten große Kugel}
  \begin{tabular}{S[table-format=2.0] S[table-format=2.0]}
    \toprule
    {$t$ oben [s]} & {$t$ unten [s]} \\
    \midrule
    43.05 &	42.63 \\
    43.33 &	42.47 \\
    43.01 &	42.92 \\
    43.47 &	42.68 \\
    42.98 &	42.79 \\
    \bottomrule
  \end{tabular}
\end{table}

Auch hier wird der Messfehler der Fallzeit mit $\Delta t = 0.5\symup{s}$ angesetzt.

\subsection{Berechnung der Reynoldszahl}
\label{sec:Berechnung der Reynoldszahl}

Hier Formel aus Theorie referenzieren!!
Daraus resultiert für die Reynoldszahl der kleinen Kugel:

\begin{gather}
  Re_{oben,kl} = (0.0218\pm0.0011) \notag \\
  Re_{unten,kl} = (0.0218\pm0.0011) \notag .
\end{gather}

Durch gleiches vorgehen erhält man für die große Kugel:

\begin{gather}
  Re_{oben,gr} = (0.00356\pm0.00016) \notag \\
  Re_{unten,gr} = (0.00360\pm0.00016) \notag .
\end{gather}

\subsection{Temperaturabhängigkeit der Viskusität}
\label{sec:Temperaturabhängigkeit der Viskusität}

\begin{table} [H]
  \centering
  \caption{Fallzeiten der großen Kugel bei variabler Temperatur von oben}
  \label{tab:Temperaturabhängigkeit oben}
  \begin{tabular}{S[table-format=2.0] S[table-format=2.0] S[table-format=2.0]}
    \toprule
    {$T$ [°C]} & {$t$ [s] 1.Messung} & {$t$ [s] 2.Messung} \\
    \midrule
    22 &	39.76 &	39.46 \\
    27 &	35.49 &	35.83 \\
    30 &	33.33 &	32.33 \\
    33 &	31.48 &	31.01 \\
    36 &	29.77 &	29.62 \\
    39 &	27.28 &	26.63 \\
    42 &	26.36 &	26.03 \\
    45 &	24.26 &	24.73 \\
    48 &	23.88 &	23.74 \\
    51 &	22.43 &	22.43 \\
    \bottomrule
  \end{tabular}
\end{table}

\begin{figure} [H]
  \centering
  \includegraphics[height=10cm]{build/plot_oben.pdf}
  \caption{Temperaturabhängigkeit der Fallzeit bei Start von oben.}
  \label{fig:Plot oben}
\end{figure}

\begin{table} [H]
  \centering
  \caption{Fallzeiten der großen Kugel bei variabler Temperatur von unten}
  \label{tab:Temperaturabhängigkeit oben}
  \begin{tabular}{S[table-format=2.0] S[table-format=2.0] S[table-format=2.0]}
    \toprule
    {$T$ [°C]} & {$t$ [s] 1.Messung} & {$t$ [s] 2.Messung} \\
    \midrule
    22 &	40.12 &	39.06 \\
    27 &  36.82 &	35.53 \\
    30 &	33.51 &	33.12 \\
    33 &	30.88 &	30.69 \\
    36 &	29.12 &	28.86 \\
    39 &	26.24 &	26.95 \\
    42 &	25.83 &	26.02 \\
    45 &	24.09 &	24.09 \\
    48 &	23.17 &	23.35 \\
    51 &	22.18 &	22.48 \\
    \bottomrule
  \end{tabular}
\end{table}

\begin{figure} [H]
  \centering
  \includegraphics[height=10cm]{build/plot_unten.pdf}
  \caption{Temperaturabhängigkeit der Fallzeit bei Start von unten.}
  \label{fig:Plot unten}
\end{figure}
