\section{Auswertung}
\label{sec:Auswertung}

% O_k_mean: 6.58\pm0.16
% U_k_mean: 6.58\pm0.16
% O_g_mean: 43.17\pm0.22
% U_g_mean: 42.70\pm0.22
% Dichte kl. Kugel 2.22\pm0.04 in g/cm^3
% Dichte gr. Kugel 2.38\pm0.04 in g/cm^3
% Viskositaet kl. Kugel oben: 1.23\pm0.05
% Viskositaet kl. Kugel unten: 1.23\pm0.05
% Apperaturkonstante gr. Kugel oben: 0.0103\pm0.0006
% Apperaturkonstante gr. Kugel unten: 0.0104\pm0.0006
% Reinoldszahl kl. Kugel oben: 98\pm6
% Reinoldszahl kl. Kugel unten: 98\pm6
% Reinoldszahl gr. Kugel oben: 14.9\pm0.6
% Reinoldszahl gr. Kugel unten: 15.1\pm0.7
% Geradegleichug Oben: 22.1\pm1.6*1/T + -1.5\pm1.6
% Geradegleichug Unten: 23.4\pm1.6*1/T + 2.7\pm1.6


\subsection{Berechnung der Viskosität von destilliertem Wasser bei Raumtemperatur}
\label{sec:Berechnung der Viskosität von destilliertem Wasser bei Raumtemperatur}

Aus dem gemessenen Durchmesser der kleinen Glaskugel $d_{\symup{kl}}=(1,565\pm0.01) \unit{\centi\metre}$ und der angegebenen Masse
$m_{\symup{kl}}=4.4531\unit{\gram}$ wird die Dichte berechnet:

\begin{equation}
  \rho_{\symup{kl}}=\frac{\emph{m}_{\symup{kl}}}{V_{\symup{kl}}}=%
  \frac{m_{\symup{kl}}}{\frac{4}{3}\symup{\pi}\left(\frac{d}{2}\right)^3}=(2,22\pm0,04) \unit{\gram\per\cubic\centi\metre}
\end{equation}
Ein Literaturwert für die Dichte von Wasser bei Raumtemperatur $T=20 \unit{\celsius}$ lautet
$\rho_{\symup{Wasser}}=0,99823 \unit{\gram\per\cubic\centi\metre}$.\cite[551/C23]{czichos}

Die Fallzeiten der kleinen Kugel sind in folgender Tabelle aufgeführt:

\begin{table} [H]
  \centering
  \caption{Fallzeiten der kleinen Kugel bei Start von oben bzw. unten}
  \label{tab:Fallzeiten kleine Kugel}
  \begin{tabular}{S[table-format=2.0] S[table-format=1.1]}
    \toprule
    {$t$ oben [s]} & {$t$ unten [s]} \\
    \midrule
    6.62 &	6.76 \\
    6.62 &	6.50 \\
    6.61 &	6.43 \\
    6.68 &	6.50 \\
    6.56 &	6.56 \\
    6.49 &	6.68 \\
    6.56 &	6.64 \\
    6.44 &	6.50 \\
    6.50 &	6.49 \\
    6.69 &	6.69 \\
    \bottomrule
  \end{tabular}
\end{table}

Dabei wird der Messfehler der Fallzeit mit $\Delta t = 0.5\symup{s}$ angesetzt.
Durch Mittelwertbildung erhält man abschließend folgende Ergebnisse:

\begin{gather}
  \overline{t}_{\symup{oben}} = (6,58\pm0,16) \symup{s} \notag \\
  \overline{t}_{\symup{unten}} = (6,58\pm0,16) \symup{s}  \notag .
\end{gather}

Setzt man diese Werte zusammen mit der gegebenen Apperaturkonstante $K=0.07640\frac{\symup{mPacm^{^{3}}}}{\symup{g}}$
und den ermittelten Dichten $\rho_{\symup{kl}}=(2,22\pm0,04) \unit{\gram\per\cubic\centi\metre}$ und
$\rho_{Fl}=\rho_{\symup{Wasser}}=0,99823 \unit{\gram\per\cubic\centi\metre}$ in
\autoref{eqn:Viskosität} ein, erhält man folgende Ergebnisse für die Viskosität des destillierten Wassers:

\begin{gather}
  \eta_{\symup{oben}} = (1,23\pm0,05) \symup{mPas}  \notag \\
  \eta_{\symup{unten}} = (1,23\pm0,05) \symup{mPas}  \notag .
\end{gather}

\subsection{Berechnung der Apperaturkonstante der großen Kugel}
\label{sec:Berechnung der Apperaturkonstante der großen Kugel}

Die Fallzeiten der großen Kugel sind in folgender Tabelle aufgeführt:

\begin{table} [H]
  \centering
  \caption{Fallzeiten der großen Kugel bei Start von oben bzw. unten}
  \label{tab:Fallzeiten große Kugel}
  \begin{tabular}{S[table-format=2.0] S[table-format=2.0]}
    \toprule
    {$t$ oben [s]} & {$t$ unten [s]} \\
    \midrule
    43.05 &	42.63 \\
    43.33 &	42.47 \\
    43.01 &	42.92 \\
    43.47 &	42.68 \\
    42.98 &	42.79 \\
    \bottomrule
  \end{tabular}
\end{table}

Auch hier wird der Messfehler der Fallzeit mit $\Delta t = 0.5\symup{s}$ angesetzt.
Durch Mittelwertbildung erhält man abschließend folgende Fallzeiten:

\begin{gather}
  \overline{t}_{\symup{oben}} = (43.17\pm0,22) \symup{s} \notag \\
  \overline{t}_{\symup{unten}} = (42.70\pm0,22) \symup{s}  \notag .
\end{gather}

Aus dem gemessenen Durchmesser der großen Glaskugel $d_{gr}=(1,585\pm0.01) \unit{\centi\metre}$ und der angegebenen Masse
$m_{\symup{gr}}=4.9528\unit{\gram}$ wird die Dichte berechnet:

\begin{equation}
  \rho_{\symup{gr}}=\frac{\emph{m}_{\symup{gr}}}{V_{\symup{gr}}}=%
  \frac{m_{\symup{gr}}}{\frac{4}{3}\symup{\pi}\left(\frac{d}{2}\right)^3}=(2,38\pm0,04) \unit{\gram\per\cubic\centi\metre}
\end{equation}

Für die Dichte des Wassers wird der selbe Wert wie in \autoref{sec:Berechnung der Viskosität von destilliertem Wasser bei Raumtemperatur}
verwendet.

\autoref{eqn:Viskosität} wird nun nach der Apperaturkonstante $K$ aufgelöst:

\begin{equation}
  K=\frac{\eta}{(\rho_{\symup{K}}-\rho_{\symup{Fl}})\cdot t}
\end{equation}

Nun werden alle Werte eingesetzt und die Ergebnisse verdoppelt, damit sich die Apperaturkonstante wie bei der kleinen
Kugel auf 10cm bezieht. Es ergibt sich:

\begin{gather}
  K_{\symup{oben}} = (0,0103\pm0,0006) \frac{\symup{mPacm^{^{3}}}}{\symup{g}}  \notag \\
  K_{\symup{unten}} = (0,0104\pm0,0006) \frac{\symup{mPacm^{^{3}}}}{\symup{g}}  \notag .
\end{gather}

\subsection{Berechnung der Reynoldszahl}
\label{sec:Berechnung der Reynoldszahl}

In \autoref{eqn:Reynoldszahl} kann die Strömungsgeschwindigkeit $v_{\symup{S}}$ der  Kugel ersetzt werden durch
ihre Fallgeschwindigkeit:

\begin{equation}
  v_{\symup{s}}=\frac{\Delta s}{\overline{t}}
\end{equation}

Dabei ist $\Delta s=5\symup{cm}$ und die Fallzeiten $\overline{t}$ wurden in
\autoref{sec:Berechnung der Viskosität von destilliertem Wasser bei Raumtemperatur} und
\autoref{sec:Berechnung der Apperaturkonstante der großen Kugel} emittelt.

Setzt man alles ein resultiert für die Reynoldszahl der kleinen Kugel:

\begin{gather}
  Re_{\symup{oben,kl}} = (98\pm6) \notag \\
  Re_{\symup{unten,kl}} = (98\pm6) \notag .
\end{gather}

Durch gleiches vorgehen erhält man für die große Kugel:

\begin{gather}
  Re_{\symup{oben,gr}} = (14,9\pm0,6) \notag \\
  Re_{\symup{unten,gr}} = (15,1\pm0,7) \notag .
\end{gather}

Da alle Reynoldszahlen deutlich kleiner als die kritische Reynoldszahl $R_{\symup{crit}}\approx 1200$ sind, liegt für alle 
betrachteten Fälle eine laminare Strömung vor.

\subsection{Temperaturabhängigkeit der Viskosität}
\label{sec:Temperaturabhängigkeit der Viskosität}
Die \textit{Andradesche Gleichung} aus \autoref{eqn:Andradesche Gleichung} wird logarithmiert, sodass sich ein linearer Zusammenhang
zwischen dem Kehrwert der Temperatur und dem natürlichen Logarithmus der dynamischen Viskosität $\eta$ ergibt:

\begin{gather}
  \eta (T)=A\cdot e^{\frac{B}{T}} \\
  \Rightarrow \symup{ln}(\eta(T)) = B \cdot \frac{1}{T} + \symup{ln}(A) \label{eqn:Andradesche Gleichung umgestellt}
\end{gather}
Als Messfehler werden für die Temperatur eine Unsicherheit von $\Delta T = 1 \unit{\celsius}$ und für die Zeit 
eine Messunsicherheit von $\Delta t = 0,5 \unit{\second}$ angesetzt.
Die Messung wird jeweils für beide Richtung getrennt ausgewertet. Zunächst wird der Start von oben berachtet.
Die dynamische Viskosität $\eta$ wird für die jeweiligen Temperaturen $T$ und verschiedenen Dichten $\rho_{\symup{Wasser}}$ 
des destillierten Wassers berechnet und tabellarisch dargestellt (\autoref{tab:Temperaturabhängigkeit oben} 
und \autoref{tab:Temperaturabhängigkeit unten}). Anschließend wird eine lineare Regression mit den Wertepaaren 
\{$\frac{1}{T}, \symup{ln}(\eta)$\} durchgeführt. Für die \autoref{eqn:Andradesche Gleichung umgestellt} ergeben sich so die Parameter
\begin{align}
  B_{\symup{oben}}&=1860\pm50 \\
  \symup{ln}(A_{\symup{oben}}) &= -6,19\pm0,16 &\Leftrightarrow A_{\symup{oben}}&=0,00206\pm0,00033 .
\end{align}

Analog ergeben sich für den Start von unten die Werte 
\begin{align}
  B_{\symup{unten}} &= 1940\pm60 \\
  \symup{ln}(A_{\symup{unten}}) &= -6,44\pm0,2 &\Leftrightarrow A_{\symup{unten}}&=0,000159\pm0.00032 .
\end{align}

\begin{table} [H]
  \centering
  \caption{Fallzeiten der großen Kugel bei variabler Temperatur von oben}
  \label{tab:Temperaturabhängigkeit oben}
  \begin{tabular}{S[table-format=2.0] S[table-format=2.0] S[table-format=2.0] S[table-format=1.5] c}
    \toprule
    {$T$ [°C]} & {$t$ [s] 1.Messung} & {$t$ [s] 2.Messung}%
    & {$\rho_{\symup{Wasser}}$ [$\unit{\gram\per\cubic\centi\metre}$]\cite{dichte}} & {$\eta$ [$\unit{\milli\pascal\second}$]} \\
    \midrule
    22 &	39.76 &	39.46 & 0.99777 &  1,125 \pm 0,050 \\
    27 &	35.49 &	35.83 & 0.99651 &  1,014 \pm 0,045 \\
    30 &	33.33 &	32.33 & 0.99565 &  0,934 \pm 0,042 \\
    33 &	31.48 &	31.01 & 0.99470 &  0,890 \pm 0,040 \\
    36 &	29.77 &	29.62 & 0.99369 &  0,846 \pm 0,038 \\
    39 &	27.28 &	26.63 & 0.99260 &  0,769 \pm 0,035 \\
    42 &	26.36 &	26.03 & 0.99144 &  0,748 \pm 0,034 \\
    45 &	24.26 &	24.73 & 0.99021 &  0,700 \pm 0,033 \\
    48 &	23.88 &	23.74 & 0.98893 &  0,681 \pm 0,032 \\
    51 &	22.43 &	22.43 & 0.98758 &  0,642 \pm 0,030 \\
    
    \bottomrule
  \end{tabular}
\end{table}
	
\begin{figure} [H]
  \centering
  \includegraphics[height=10cm]{build/plot_oben.pdf}
  \caption{Temperaturabhängigkeit der Fallzeit bei Start von oben.}
  \label{fig:Plot oben}
\end{figure}

\begin{table} [H]
  \centering
  \caption{Fallzeiten der großen Kugel bei variabler Temperatur von unten}
  \label{tab:Temperaturabhängigkeit unten}
  \begin{tabular}{S[table-format=2.0] S[table-format=2.0] S[table-format=2.0] S[table-format=1.5] c}
    \toprule
    {$T$ [°C]} & {$t$ [s] 1.Messung} & {$t$ [s] 2.Messung}%
    & {$\rho_{\symup{Wasser}}$ [$\unit{\gram\per\cubic\centi\metre}$]\cite{dichte}} & {$\eta$ [$\unit{\milli\pascal\second}$]} \\
    \midrule
    22 &	40.12 &	39.06 & 0.99777 &  1,137 \pm 0,050\\
    27 &  36.82 &	35.53 & 0.99651 &  1,040 \pm 0,046\\
    30 &	33.51 &	33.12 & 0.99565 &  0,958 \pm 0,043\\
    33 &	30.88 &	30.69 & 0.99470 &  0,886 \pm 0,040\\
    36 &	29.12 &	28.86 & 0.99369 &  0,835 \pm 0,038\\
    39 &	26.24 &	26.95 & 0.99260 &  0,767 \pm 0,035\\
    42 &	25.83 &	26.02 & 0.99144 &  0,748 \pm 0,035\\
    45 &	24.09 &	24.09 & 0.99021 &  0,696 \pm 0,033\\
    48 &	23.17 &	23.35 & 0.98893 &  0,672 \pm 0,032\\
    51 &	22.18 &	22.48 & 0.98758 &  0,646 \pm 0,031\\
    \bottomrule
  \end{tabular}
\end{table}

\begin{figure} [H]
  \centering
  \includegraphics[height=10cm]{build/plot_unten.pdf}
  \caption{Temperaturabhängigkeit der Fallzeit bei Start von unten.}
  \label{fig:Plot unten}
\end{figure}
