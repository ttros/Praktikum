\section{Auswertung}
\label{sec:Auswertung}

% O_k_mean: 6.58+/-0.16
% U_k_mean: 6.58+/-0.16
% O_g_mean: 43.17+/-0.22
% U_g_mean: 42.70+/-0.22
% Dichte kl. Kugel 2.22+/-0.04 in g/cm^3
% Dichte gr. Kugel 2.38+/-0.04 in g/cm^3
% Viskositaet kl. Kugel oben: 1.23+/-0.05
% Viskositaet kl. Kugel unten: 1.23+/-0.05
% Apperaturkonstante gr. Kugel oben: 0.0103+/-0.0006
% Apperaturkonstante gr. Kugel unten: 0.0104+/-0.0006
% Reinoldszahl kl. Kugel oben: 0.0218+/-0.0011
% Reinoldszahl kl. Kugel unten: 0.0218+/-0.0011
% Reinoldszahl gr. Kugel oben: 0.00356+/-0.00016
% Reinoldszahl gr. Kugel unten: 0.00360+/-0.00016
% Geradegleichug Oben: 22.4+/-1.6*1/T + 2.7+/-1.6
% Geradegleichug Unten: 23.4+/-1.6*1/T + 2.7+/-1.6

\subsection{Bestimmung der Viskosität von destilliertem Wasser bei Raumtemperatur}
\begin{figure}
  Aus dem gemessenen Durchmesser der kleinen Glaskugel $d_{kl}=(1,565\pm0.01) \unit{\centi\metre}$ und der angegebenen Masse
  $m_{\symup{kl}}=4.4531\unit{\gram}$ wird die Dichte berechnet:
  \begin{equation}
    \rho_{\symup{kl}}=\frac{\emph{m}_{\symup{kl}}}{V_{\symup{kl}}}=%
    \frac{m_{\symup{kl}}}{\frac{4}{3}\symup{\pi}\left(\frac{d}{2}\right)^3}=(2,22\pm0,04) \unit{\gram\per\cubic\centi\metre}
  \end{equation}
  Ein Literaturwert für die Dichte von Wasser bei Raumtemperatur $T=20 \unit{\celsius}$ lautet
  $\rho_{\symup{Wasser}}=0,99823 \unit{\gram\per\cubic\centi\metre}$\cite[551/C23]{czichos}
  
\end{figure}


