\section{Diskussion}
\label{sec:Diskussion}
Die experimentell ermittelte Wellenlänge des Lasers, sowie der Brechungsindex von Luft unter Normalbedingungen 
werden in \autoref{tab:Diskussion} mit Literaturwerten verglichen.
\begin{table}[H]
    \centering
    \caption{Experimentell ermittelte Größen im Vergleich zu Literaturwerten (\cite{v401}, \cite{Brechungsindex_Wien}).}
    \label{tab:Diskussion}
    \begin{tabular}{l S S S}
        \toprule
        {Größe} & {Exp.} & {Lit.} & {relative Abweichung} \\
        \midrule
        {$\lambda$} & $\qty{644+-7}{\nano\metre}$ & $\qty{635}{\nano\metre}$  & $\qty{1.465}{\percent}$ \\
        {$n$}       & $\num{1.00028+-0.00002}$    & $\num{1,0003}$            & $\qty{0.002}{\percent}$ \\
        \bottomrule
    \end{tabular}
  \end{table}

Es fällt auf, dass sich die experimentell bestimmte Wellenlänge $\lambda$ des Lasers auch mit den Messunsicherheiten nicht
mit der theoretischen Wellenlänge vereinen lässt. Dennoch ist die relative Abweichung mit 
$\symup{\Delta}\lambda = \qty{1.465}{\percent}$ sehr gering. Mögliche Fehlerquellen hierbei sind eventuelle andere Lichtquellen,
die Messsignale an der Photodiode auslösen könnten. 
Außerdem denkbar ist, dass die Kalibrierung der Spiegel nicht perfekt ist,
was zur Folge hat, dass nicht alle Maxima und Minima detektiert werden können.
Ebenfalls denkbar ist es, dass der Synchronmotor nicht absolut gleichmäßig läuft, sodass auch hier Maxima und Minima übersprungen
werden.

Der experimentelle Brechungsindex von Luft unter Normalbedingungen stimmt mit der geringen Messunsicherheit mit dem Theoriewert
überein. Dennoch sind Ungenauigkeiten, aus denselben Gründen wie bereits beschreiben, nicht auszuschließen.