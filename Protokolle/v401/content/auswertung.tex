\section{Auswertung}
\label{sec:Auswertung}
In der Auswertung werden alle Mittelwerte und deren Unsicherheiten mithilfe der \textit{python}-Erweiterung 
\textit{uncertainties} \cite{uncertainties} berechnet.

\subsection{Messung der Wellenlänge des Lasers}
Die Messwerte zur Bestimmung der Wellenlänge des Lasers sind in \autoref{tab:Wellenlänge} dargestellt.
\begin{table} [H]
    \centering
    \caption{Messwerte zur Bestimmung der Wellenlänge des Lasers.}
    \label{tab:Wellenlänge}
    \begin{tabular}{S S}
      \toprule
      {Messung} & {Zählrate} \\
      \midrule
      1  & 3111 \\
      2  & 3138 \\
      3  & 3117 \\
      4  & 3096 \\
      5  & 3047 \\
      6  & 3117 \\
      7  & 3040 \\
      8  & 3119 \\
      9  & 3033 \\
      10 & 3118 \\
      \midrule
      {Mittelwert} & $\num{3090+-40}$ \\  
      \bottomrule
    \end{tabular}
\end{table}

Aufgrund des Übersetzungsverhältnises von $u = 1:5,017$ ergibt die Verstellung der Stellschraube um $\qty{5}{\milli\metre}$
eine Veränderung der Armlänge von $\symup{\Delta}s=\qty{0.997}{\milli\metre}$. Mit Formel \eqref{eq:Lambda} folgt für die 
Wellenlänge des Laserlichts
\begin{equation*}
    \lambda_{\text{exp}} = \qty{644+-7}{\nano\metre}.
\end{equation*}

\subsection{Messung des Brechungsindex von Luft}
Die Zählraten der Photodiode für diesen Teil des Versuches sind in \autoref{tab:Brechungsindex} dargestellt.
 
\begin{table} [H]
    \centering
    \caption{Messwerte zur Bestimmung des Brechungsindex von Luft.}
    \label{tab:Brechungsindex}
    \begin{tabular}{S S}
      \toprule
      {Messung} & {Zählrate} \\
      \midrule
      1  & 36 \\
      2  & 33 \\
      3  & 34 \\
      4  & 33 \\
      5  & 30 \\
      6  & 32 \\
      \midrule
      {Mittelwert} & $\num{33.0+-1.8}$ \\  
      \bottomrule
    \end{tabular}
\end{table}

Die Länge der Kammer, die evakuiert werden kann, beträgt $b = \qty{5}{\centi\metre}$. Mit der Wellenlänge des Lasers von
$\lambda_{\text{lit}}=\qty{635}{\nano\metre}$ und dem Zusammenhang \eqref{eq:Delta N} ergibt sich eine Veränderung des 
Brechungsindex von
\begin{equation*}
    \symup{\Delta}n = \num{0.000210+-0.000012}.
\end{equation*}
Der Druck in der Kammer wird auf $p'=\qty{20}{\kilo\pascal}$ abgesenkt und die Temperatur wird zu 
$T=\qty{23}{\celsius}=\qty{296}{\kelvin}$ bestimmt.
Mithilfe von Formel \eqref{eq:Brechung} ergibt sich der Brechungsindex von Luft unter Normalbedingungen zu 
\begin{equation*}
    n_\text{Luft} = \num{1.00028+-0.00002}.
\end{equation*}