\section{Diskussion}
\label{sec:Diskussion}
Die experimentell ermittelten Werte für die Elementarladung und die Avogadrokonstante sind in \autoref{tab:relative fehler} dargestellt
und werden mit Literaturwerten verglichen.
\begin{table}[H]
    \centering
    \caption{Experimentell ermittelte Größen im Vergleich zu Literaturwerten \cite{czichos}.}
    \label{tab:relative fehler}
    \begin{tabular}{l S S S}
        \toprule
        {Größe} & {Exp.} & {Lit.} & {relative Abweichung} \\
        \midrule
        {$e_0$}                 & $\qty{1.198(1.038)e-19}{\coulomb}$    & $\qty{1,602e-19}{\coulomb}$        & $\qty{25,22}{\percent}$ \\
        {$e_{0\text{ ,korr}}$}  & $\qty{1.255(0.749)e-19}{\coulomb}$    & $\qty{1,602e-19}{\coulomb}$        & $\qty{21,66}{\percent}$ \\
        {$N_{\symup{A}}$}       & $\qty{7.69(4.59)e23}{\per\mol}$       & $\qty{6,022e23}{\per\mol}$         & $\qty{27,70}{\percent}$ \\
        \bottomrule
    \end{tabular}
  \end{table}
Es fällt auf, dass sich sich die experimentellen Werte nicht mit den Literaturwerten vereinen lassen und die relativen Abweichungen 
sehr groß sind. Die Unsicherheiten der experimentellen Werte sind besonders groß. Dies könnte daran liegen, dass alle Werte allein aus den
Steig- und Sinkzeiten der Öltröpfchen bestimmt werden und bereits diese Werte mit nicht zu vernachlässigenden Unsicherheiten
behaftet sind. Diese Unsicherheiten entstehen, da die Steig- und Sinkzeiten der Öltröpfchen nicht besonders genau zu bestimmen sind, weil
sich die Beobachtung der Tröpfchen als schwierig herausstellt. Dies liegt darin begründet, dass die Messapparatur aufgrund mangelhafter
Beleuchtung nur eine Beobachtung der Tröpfchen in einem eingeschränkten Bereich zulässt.
Durch eine bessere Methode zur Bestimmung der Geschwindigkeiten der Öltröpfchen ließen sich die Unsicherheiten verringern.

Außerdem ist es möglich, dass die gewählte Methode zur Bestimmung der Elementarladung aus den Ladungen der einzelnen Öltröpfchen nicht optimal
ist und verbessert werden sollte.