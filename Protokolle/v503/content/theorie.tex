\section{Zielsetzung}

Ziel dieses Versuches ist, mithilfe der Steig- und Sinkgeschwindigkeit von Öltröpfchen in einem
elektrischen Feld die Elementarladung $e_0$ zu bestimmen.

\section{Theorie}
\label{sec:Theorie}
Um die besagte Elementarladung $e_0$ zu bestimmen, wird die von Milikan benutzte Öltröpfchenmethode angewendet.
Mithilfe eines Zerstäubers werden sehr kleine Öltröpchen in ein vertikales E-Feld eines Plattenkondensators gesprüht.
Aufgrund der dabei auftretenden Reibung werden die Tröpchen geladen, diese Ladung muss einem ganzzähligen
Vielfachen der Elementarladung entsprechen.

Es werden nun drei mögliche Fälle und die dort herrschenden Kräfte betrachtet, um daraus abschließend die
Ladung eines Tröpchens bestimmen zu können.

\subsection{Freies Fallen}
Zuerst wird kein Feld im Kondensator erzeugt und das Verhalten des freien Tröpchens betrachtet.
In diesem Fall wirkt auf das Öltröpfchen der Masse $m$ und Radius $r$ die Gewichtskraft $F_{\symup{G}}=mg$, sodass
es beginnt sich nach unten zu bewegen.
Dem entgegen wirkt dann aufgrund der Luft mit der Viskosität $\eta_{L}$ im Kondensator
die Stokesche Reibungskraft $F_{\symup{R}}=6\pi r\eta_{L}v$.
Nach kurzer Zeit stellt sich ein Kräftegleichgewicht ein, das Tröpchen sinkt dann mit konstanter
Geschwindigkeit $v_0$.

\subsection{Sinken und Steigen}
Nun wird eine Spannung an den Kondensator angelegt, sodass ein elektrisches Feld zwischen den Plattenkondensators
existiert. Auf das Öltröpchen wirkt nun zusätzlich die elektrostatische Kraft $F_{\symup{el}}=qE$, deren Richtung
durch Umpolen der Spannung entweder nach oben oder nach unten gewählt werden kann.
\begin{figure} [H]
    \centering
    \includegraphics[height=4cm]{content/pics/Kräfte.jpg}
    \caption{Kräftegleichgewicht für einen fallenden bzw. steigenden Tropfen in homogenem elektrischen Feld \cite{v503}.}
    \label{fig:Kraefte}
\end{figure}
In \autoref{fig:Kraefte} sind für beide Szenarien alle wirkenden Kräfte samt Richtung eingezeichnet.

Wählt man eine elektrische Kraft nach unten, so bewegt sich das Tröpfchen nach kurzer Zeit mit einer konstanten
Geschwindigkeit $v_{\symup{ab}}$ abwärts. Berücksichtigt man den Auftrieb in der Luft, so ergibt sich
aus dem Kräftegleichgewicht die Gleichung
\begin{equation}
    \label{eq:v_ab}
    \frac{4\symup{\pi}}{3}r^3(\rho_\text{Oel}-\rho_\text{L})g - 6\symup{\pi}\eta_\text{L}rv_{\text{ab}} = -qE.
\end{equation}
Bei umgekehrten elektrischen Feld stellt sich eine konstante Aufwärtsbewegung ein mit einer Geschwindigkeit $v_{\symup{auf}}$,
hier lautet die Gleichung
\begin{equation}
    \label{eq:v_auf}
    \frac{4\symup{\pi}}{3}r^3(\rho_\text{Oel}+\rho_\text{L})g + 6\symup{\pi}\eta_\text{L}rv_{\text{ab}} = qE.
\end{equation}
Die Ladung $q$ sowie der Radius $r$ des Öltröpfchens lassen sich zusammen aus \eqref{eq:v_ab} und \eqref{eq:v_auf}
bestimmen und es ergeben sich die Zusammenhänge:
\begin{equation}
    \label{eq:Ladung}
    q = \frac{9}{2} \symup{\pi} \sqrt{\frac{\eta_\text{L}^3(v_\text{ab} - v_\text{auf})}{g(\rho_\text{Oel}- \rho_\text{L})}} \cdot \frac{v_\text{ab} + v_\text{auf}}{E}.
\end{equation}
\begin{equation}
    \label{eq:Radius}
    r = 3\sqrt{\frac{\eta_\text{L}(v_\text{ab} - v_\text{auf})}{2g(\rho_\text{Oel}- \rho_\text{L})}}
\end{equation}
Für die Geschwindigkeiten muss dabei gelten
\begin{equation}
    \label{eq:Geschwindigkeiten}
    2v_0 = v_\text{ab} - v_\text{auf}.
\end{equation}

\subsection{Cunningham-Korrektur}
In diesem Versuch sind die Öltröpchen so klein, dass ihr Durchmesser kleiner als die mittlere freie Weglänge
in Luft ist. Daher muss die Viskosität der Luft $\eta_{\text{L}}$ mit dem
\textit{Cunningham-Korrekturterm} versehen werden und die effektive Viskosität lautet
\begin{equation*}
    \eta_\text{eff} = \eta_\text{L} \left(\frac{1}{1 + B \frac{1}{pr}}\right).
\end{equation*}
Dabei ist $B=6.17\cdot10^{-3}\,\text{Torr}\cdot\text{cm}$ \cite{v503} mit $1\,\text{Torr}\approx\qty{133.322}{\pascal}$ \cite{Torr}.
Mit dieser Korrektur ergibt sich für die Ladung
\begin{equation}
\label{eqn:q_korrigiert}
    q_\text{korr} = q_0 \left(1+ \frac{B}{pr}\right)^{3/2}
\end{equation}