\section{Auswertung}
\label{sec:Auswertung}
\subsection{Bestimmung der mittleren Geschwindigkeiten der Öltröpfchen}
Aus den aufgenommenen Steig- und Sinkgeschwindigkeiten wird die mittlere Geschwindigkeit eines jeden Tröpfchens
mit dazugehöriger Abweichung mithilfe der \textit{python}-Erweiterung \textit{uncertainties}\,\cite{uncertainties}
bestimmt. Die daraus resultierenden Geschwindigkeiten werden in \autoref{tab:v_m} dargestellt.\footnote{Anm: Die 
Versuchsanleitung fordert eigentlich die Bestimmung der Fallgeschwindigkeit $v_0$ der Öltröpfchen bei ausgeschaltetem
E-Feld. Bereits während des Versuchs stellte sich heraus, dass diese Werte unplausibel und nicht verwertbar sind, 
weshalb auf die Aufnahme und Betrachtung von $v_0$ verzichtet wird.}

\begin{table}[H]
    \centering
    \caption{Mittleren Geschwindigkeiten der Öltröpfchen bei verschiedenen Spannungen.}
    \label{tab:v_m}
    \begin{tabular}{S S S S}
        \toprule
        {Spannung} & {Tröpfchen} & {$v_{\text{auf}} \mathbin{/} 10^{-5}\,\frac{\unit{\metre}}{\unit{\second}}$} %
        & {$v_{\text{ab}} \mathbin{/} 10^{-5}\,\frac{\unit{\metre}}{\unit{\second}}$}\\
        \midrule
        {150\,V}    & 1 &   $\num{3.628+-0.000}$ & $\num{4.677+-0.000}$ \\
        {}          & 2 &   $\num{3.454+-0.345}$ & $\num{4.385+-0.770}$ \\   
        {}          & 3 &   $\num{5.223+-0.040}$ & $\num{6.504+-0.538}$ \\ 
        {}          & 4 &   $\num{5.101+-0.268}$ & $\num{7.435+-0.090}$ \\ 
        {}          & 5 &   $\num{8.217+-0.548}$ & $\num{7.114+-0.543}$ \\
        {175\,V}    & 1 &   $\num{7.828+-0.299}$ & $\num{8.091+-0.139}$ \\   
        {}          & 2 &   $\num{23.12+-1.33 }$ & $\num{23.15+-0.62 }$ \\
        {}          & 3 &   $\num{6.657+-0.215}$ & $\num{5.911+-0.053}$ \\
        {}          & 4 &   $\num{12.57+-0.30 }$ & $\num{12.11+-0.53 }$ \\
        {}          & 5 &   $\num{12.34+-0.32 }$ & $\num{11.76+-0.72 }$ \\
        {200\,V}    & 1 &   $\num{13.67+-0.29 }$ & $\num{13.70+-0.04 }$ \\
        {}          & 2 &   $\num{5.274+-0.188}$ & $\num{1.026+-0.040}$ \\
        {}          & 3 &   $\num{4.250+-0.168}$ & $\num{8.425+-0.558}$ \\
        {}          & 4 &   $\num{6.884+-0.275}$ & $\num{7.365+-0.356}$ \\
        {}          & 5 &   $\num{6.517+-0.223}$ & $\num{6.933+-0.121}$ \\
        {225\,V}    & 1 &   $\num{7.483+-0.391}$ & $\num{7.995+-0.291}$ \\
        {}          & 2 &   $\num{6.126+-0.068}$ & $\num{7.190+-0.355}$ \\
        {}          & 3 &   $\num{3.514+-0.222}$ & $\num{4.652+-0.395}$ \\
        {}          & 4 &   $\num{10.90+-0.10 }$ & $\num{10.50+-0.44 }$ \\
        {}          & 5 &   $\num{8.738+-0.233}$ & $\num{8.115+-0.507}$ \\
        {250\,V}    & 1 &   $\num{11.26+-0.57 }$ & $\num{10.92+-0.75 }$ \\
        {}          & 2 &   $\num{11.81+-0.84 }$ & $\num{11.69+-0.20 }$ \\
        {}          & 3 &   $\num{9.996+-0.123}$ & $\num{9.904+-0.526}$ \\
        {}          & 4 &   $\num{20.05+-0.89 }$ & $\num{20.91+-0.41 }$ \\
        {}          & 5 &   $\num{7.316+-0.149}$ & $\num{8.580+-0.196}$ \\ 
        \bottomrule
    \end{tabular}
  \end{table}

\subsection{Bestimmung der Ladung und Radien der Öltröpchen}
Mithilfe der zuvor berechneten Steig- und Sinkgeschwindigkeiten der Öltröpfchen und der Formel \eqref{eq:Radius}
wird der Radius eines jeden Öltröpfchens bestimmt. Dafür wird außerdem die Viskosität der Luft $\eta_{\symup{L}}$ benötigt.
Diese ist abhängig von der Temperatur über einen Zusammenhang der über \autoref{fig:eta_L} gegeben ist. Die Temperatur in 
dem Kondensator wird allerdings nicht direkt gemessen, sondern wird über einen Thermowiderstand und die dazugehörige Tabelle
aus \autoref{fig:R_T} bestimmt.

Darüber hinaus lässt sich über Formel \eqref{eq:Ladung} die Ladung der Öltröpfchen bestimmen, welche von denselben Parametern
wie der Radius abhängt.

Aufgrund der Tatsache, dass die Öltröpfchen kleiner als die mittlere freie Weglänge in dem Plattenkondensators sind, muss die
Ladung nach Cunningham und somit nach Gleichung \eqref{eqn:q_korrigiert} korrigert werden.

Die Werte der Radien, sowie die unkorrigerten und korrigierten Ladungen der Öltröpfchen sind in \autoref{tab:r,q,q_korr} 
aufgelistet.

\begin{table}[H]
    \centering
    \caption{Radien, unkorrigierte und korrigierte Ladungen der Öltröpfchen bei verschiedenen Spannungen.}
    \label{tab:r,q,q_korr}
    \begin{tabular}{S S S S S}
        \toprule
        {Spannung} & {Tröpfchen} & {$r \mathbin{/} 10^{-7}\,\unit{\metre}$} %
        & {$q \mathbin{/} 10^{-19}\,\unit{\coulomb}$} & {$q_{\text{korr}} \mathbin{/} 10^{-19}\,\unit{\coulomb}$}\\
        \midrule
        {150\,V}    & 1 &   $\num{3.176+-0.000}$ & $\num{1.657+-0.001}$ & $\num{2.086+-0.001}$\\
        {}          & 2 &   $\num{2.992+-1.357}$ & $\num{1.473+-0.783}$ & $\num{1.878+-0.817}$\\   
        {}          & 3 &   $\num{3.510+-0.740}$ & $\num{2.585+-0.663}$ & $\num{3.191+-0.690}$\\ 
        {}          & 4 &   $\num{4.739+-0.287}$ & $\num{3.731+-0.167}$ & $\num{4.379+-0.159}$\\ 
        {}          & 5 &   $\num{3.258+-1.140}$ & $\num{3.136+-1.110}$ & $\num{3.928+-1.117}$\\
        {175\,V}    & 1 &   $\num{1.595+-0.997}$ & $\num{1.368+-0.837}$ & $\num{2.074+-0.828}$\\   
        {}          & 2 &   $\num{0.047+-1.504}$ & $\num{0.117+-3.748}$ & $\num{0.322+-3.744}$\\
        {}          & 3 &   $\num{2.683+-0.397}$ & $\num{1.817+-0.298}$ & $\num{2.374+-0.307}$\\
        {}          & 4 &   $\num{2.105+-1.406}$ & $\num{2.799+-1.834}$ & $\num{3.893+-1.821}$\\
        {}          & 5 &   $\num{2.376+-1.608}$ & $\num{3.086+-2.022}$ & $\num{4.154+-2.000}$\\
        {200\,V}    & 1 &   $\num{0.492+-2.896}$ & $\num{0.634+-3.722}$ & $\num{1.693+-3.711}$\\
        {}          & 2 &   $\num{6.933+-0.310}$ & $\num{5.066+-0.339}$ & $\num{5.667+-0.354}$\\
        {}          & 3 &   $\num{6.346+-0.443}$ & $\num{3.784+-0.420}$ & $\num{4.275+-0.442}$\\
        {}          & 4 &   $\num{2.154+-1.007}$ & $\num{1.444+-0.688}$ & $\num{1.995+-0.694}$\\
        {}          & 5 &   $\num{2.003+-0.612}$ & $\num{1.267+-0.374}$ & $\num{1.788+-0.370}$\\
        {225\,V}    & 1 &   $\num{2.224+-1.058}$ & $\num{1.447+-0.677}$ & $\num{1.983+-0.674}$\\
        {}          & 2 &   $\num{3.204+-0.544}$ & $\num{1.794+-0.350}$ & $\num{2.255+-0.362}$\\
        {}          & 3 &   $\num{3.315+-0.660}$ & $\num{1.138+-0.265}$ & $\num{1.421+-0.276}$\\
        {}          & 4 &   $\num{1.966+-1.100}$ & $\num{1.768+-0.957}$ & $\num{2.508+-0.943}$\\
        {}          & 5 &   $\num{2.453+-1.098}$ & $\num{1.739+-0.742}$ & $\num{2.322+-0.731}$\\
        {250\,V}    & 1 &   $\num{1.811+-2.525}$ & $\num{1.520+-2.104}$ & $\num{2.211+-2.097}$\\
        {}          & 2 &   $\num{1.095+-3.807}$ & $\num{0.974+-3.418}$ & $\num{1.705+-3.442}$\\
        {}          & 3 &   $\num{0.944+-2.765}$ & $\num{0.711+-2.065}$ & $\num{1.330+-2.050}$\\
        {}          & 4 &   $\num{2.880+-1.650}$ & $\num{4.464+-2.490}$ & $\num{5.739+-2.471}$\\
        {}          & 5 &   $\num{3.494+-0.340}$ & $\num{2.102+-0.216}$ & $\num{2.597+-0.219}$\\ 
        \bottomrule
    \end{tabular}
  \end{table}

\subsection{Bestimmung der Elementarladung}
Die korrigierten Ladungen der Öltröpfchen aus \autoref{tab:r,q,q_korr} werden in \autoref{fig:plot} veranschaulicht.
  \begin{figure}[H]
    \centering
    \includegraphics{build/plot.pdf}
    \caption{Veranschaulichung der korrigierten Ladungen der Öltröpfchen mit Unsicherheiten bei verschiedenen Spannungen.}
    \label{fig:plot}
\end{figure}
Es fällt auf, dass viele der Ladungen eine große Unsicherheit aufweisen auf die in der Diskussion in Abschnitt \ref{sec:Diskussion} 
eingegangen wird.

Die Elementarladung wird bestimmt, indem der minimale Abstand der einzelnen Ladungen einer Messreihe berechnet wird. Dabei 
werden nur Abstände berücksichtigt, die größer als $10^{-19}\,\unit{\coulomb}$ sind, um außzuschließen, dass Tröpfchen selber
Ladung das Ergebnis verfälschen.\footnote{Anm: Dies hat bei der Messreihe für 225\,V zur Folge, dass keine Elementarladung
bestimmt werden kann. Es ist davon auszugehen, dass die hier beobachteten Tröpfen dieselbe Ladungsmenge hatten. \\Der Wert von 
$e_{0\text{ ,korr}}$ für die Spannung von 175\,V wird aufgrund der extrem großen Unsicherheit ebenfalls nicht bei der Berechnung 
des Mittelwerts betrachet.}
\begin{table}[H]
    \centering
    \caption{Berechneten Elementarladungen bei verschiedenen Spannungen und deren Mittelwert.}
    \label{tab:e_0}
    \begin{tabular}{S S S}
        \toprule
        {Spannung} & {$e_0 \mathbin{/} 10^{-19}\,\unit{\coulomb}$} %
        & {$e_{0\text{ ,korr}} \mathbin{/} 10^{-19}\,\unit{\coulomb}$}\\
        \midrule
        {150\,V} & $\num{1.112+-1.025}$ & $\num{1.105+-0.690}$ \\
        {175\,V} & $\num{1.269+-2.044}$ & {$(\num{1.15+-37.45})$} \\
        {200\,V} & $\num{1.282+-0.540}$ & $\num{1.393+-0.566}$ \\
        {225\,V} & {}                   & {}                   \\
        {250\,V} & $\num{1.128+-3.425}$ & $\num{1.267+-2.062}$ \\
        \midrule
        {Mittelwert} & $\num{1.198+-1.038}$ & $\num{1.255+-0.749}$ \\
        \bottomrule
    \end{tabular}
  \end{table}
  
  \subsection{Berechnung der Avogadrokonstante}
  Der Zusammenhang zwischen Elementarladung und Avogadrokonstante ist gegeben durch
  \begin{equation*}
      N_{\symup{A}} = \frac{F}{e_0}.
  \end{equation*}
Hierbei bezeichnet $F$ die Faraday-Konstante, welche den Wert $F=\qty{96485,3399 +- 0,0024}{\coulomb\per\mol}$\,\cite{czichos} hat.
Mit dem Mittelwert für $e_{0\text{ ,korr}}$ ergibt sich die Avogadrokonstante zu
\begin{equation*}
    N_{\symup{A}} = \qty{7.69(4.59)e23}{\per\mol}.
\end{equation*}



